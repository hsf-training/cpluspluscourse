\subsection[range]{Ranges}

\begin{frame}[fragile]
  \frametitlecpp[20]{Ranges}
  \begin{block}{The range concept}
    \begin{itemize}
    \item A range is a sequence of items that you can iterate over
    \item It must provide iterators to its beginning and end, obtainable via \mintinline{cpp}{std::range::begin} and \mintinline{cpp}{std::range::end}
    \item Range concepts are located in \mintinline{cpp}{std::ranges} namespace
    \item All STL containers are ranges
    \end{itemize}
  \end{block}
  \begin{exampleblock}{Variations on ranges}
    There are actually different types of ranges
    \begin{itemize}
    \item \mintinline{cpp}{input_range}, \mintinline{cpp}{output_range}, \mintinline{cpp}{forward_range}, \mintinline{cpp}{bidirectional_range}, \mintinline{cpp}{random_access_range}, \mintinline{cpp}{contiguous_range}
    \item corresponding to the different iterator categories
    \end{itemize}
  \end{exampleblock}
\end{frame}

\begin{frame}[fragile]
  \frametitlecpp[20]{Views}
  \begin{block}{The view concept}
    \begin{itemize}
    \item A \mintinline{cpp}{view} is a non-owning object built from a range or another view after applying some range adaptor(s)
    \item The time complexity to copy, move, assign is constant
    \item Range adaptors can be applied/chained with \mintinline{cpp}{|} and are lazy
    \item They live in the \mintinline{cpp}{std::views} namespace
    \item They introduce easy functional programming to the STL
    \end{itemize}
  \end{block}
  \begin{exampleblock}{Example \godboltLink{https://godbolt.org/z/d1vTv4TMa}}
    { \small
      \begin{cppcode*}{gobble=4}
        auto const numbers = std::views::iota(0, 6);
        auto even = [](int i) { return 0 == i % 2; };
        auto square = [](int i) { return i * i; };
        auto results = numbers | std::views::filter(even)
                               | std::views::transform(square);
        for (auto a : results) { ... } // 0, 4, 16
      \end{cppcode*}
    }
  \end{exampleblock}
\end{frame}

\begin{frame}[fragile]
  \frametitlecpp[20]{Many range adaptors are provided by the STL}
  \begin{block}{Useful subset}
    \begin{description}[keys/values]
    \item[all] a view on all elements
    \item[filter] applies a filter
    \item[transform] applies a transformation
    \item[take] takes only first n elements
    \item[drop] skips first n elements
    \item[join/split] joins ranges or splits into subranges
    \item[reverse] reverses view order
    \item[elements] for tuple views, extract nth elements of each tuple
    \item[keys/values] for pair like views, takes 1st/2nd element of each pair
    \end{description}
  \end{block}
  Remember you can combine them via (\mintinline{cpp}{|})
\end{frame}

\begin{frame}[fragile]
  \frametitlecpp[23]{Back to laziness}
  \begin{block}{Views are lazy}
    \begin{itemize}
    \item Computation is only triggered when iterating
    \item And can be stopped by e.g. \mintinline{cpp}{take}
    \item Here, the minimal number of iterations is performed
    \end{itemize}
  \end{block}
  \begin{exampleblock}{Example \godboltLink{https://godbolt.org/z/bWe6W69oE}}
    \begin{cppcode*}{gobble=2}
      // print first 20 prime numbers above 1000000
      for (int i: std::views::iota(1000000)
                  | std::views::filter(odd)
                  | std::views::filter(isPrime)
                  | std::views::take(20)) {
        std::cout << i << " ";
      }
    \end{cppcode*}
  \end{exampleblock}
\end{frame}
