\subsection[except]{Exceptions}

\begin{frame}[fragile]
  \frametitlecpp[98]{Exceptions}
  \begin{block}{The concept}
    \begin{itemize}
      \item to handle \textit{exceptional} events that happen rarely
      \item and cleanly jump to a place where the error can be handled
    \end{itemize}
  \end{block}
  \begin{block}{In practice}
    \begin{itemize}
      \item add an exception handling block with \mintinline{cpp}{try} ... \mintinline{cpp}{catch}
      \begin{itemize}
        \item when exceptions are possible \textit{and can be handled}
      \end{itemize}
      \item throw an exception using \mintinline{cpp}{throw}
      \begin{itemize}
        \item when a function cannot proceed or recover internally
      \end{itemize}
    \end{itemize}
  \end{block}
  \begin{multicols}{2}
    \begin{cppcode*}{fontsize=\scriptsize,gobble=2}
      #include <stdexcept>
      ...
      try {
        process_stream_data(s);
      } catch (const range_error& e) {
        cerr << e.what() << endl;
      }
    \end{cppcode*}
    \columnbreak
    \begin{cppcode*}{fontsize=\scriptsize,gobble=2}
      void process_stream_data(stream &s) {
        ...
        if (data_location >= buffer.length()) {
          throw range_error("buf overflow");
        }
        ...
      }
    \end{cppcode*}
  \end{multicols}
\end{frame}

\begin{frame}[fragile]
  \frametitlecpp[98]{Exceptions}
  \begin{block}{Rules and behavior}
    \begin{itemize}
      \item objects of any type can be thrown
      \begin{itemize}
        \item prefer standard exception types from the \texttt{<stdexcept>} header
        \item define your own subclass of \mintinline{cpp}{std::exception} if needed
      \end{itemize}
      \item an exception will be caught if the type in the catch clause matches or is a base class of the thrown object's static type
      \begin{itemize}
          \item if no one catches an exception then \mintinline{cpp}{std::terminate} is called
      \end{itemize}
      \item you can have multiple catch clauses, will be matched in order
      \item all objects on the stack between the \mintinline{cpp}{throw} and the \mintinline{cpp}{catch} are destructed automatically during stack unwinding
      \begin{itemize}
        \item this should cleanly release intermediate resources
        \item make sure you are using the RAII idiom for your own classes
      \end{itemize}
    \end{itemize}
  \end{block}
\end{frame}

\begin{frame}[fragile]
  \frametitlecpp[17]{Exceptions}
  \begin{block}{Advice}
    \begin{itemize}
      \item throw exceptions by value, catch them by (const) reference
      \item use exceptions for \textit{unlikely} runtime errors outside the program's control
      \begin{itemize}
        \item  bad inputs, files unexpectedly not found, DB connection, \ldots
      \end{itemize}
      \item \textit{don't} use exceptions for logic errors in your code
      \begin{itemize}
        \item consider \mintinline{cpp}{assert} and tests
      \end{itemize}
      \item \textit{don't} use exceptions to provide alternative return values (or to skip them)
      \begin{itemize}
        \item you can use \mintinline{cpp}{std::optional} or \mintinline{cpp}{std::variant}
        \item avoid using the global C-style \mintinline{cpp}{errno}
      \end{itemize}
      \item See also the \href{https://isocpp.github.io/CppCoreGuidelines/CppCoreGuidelines#S-errors}{\cpp core guidelines} and the \href{https://isocpp.org/wiki/faq/exceptions}{ISO \cpp FAQ}
    \end{itemize}
  \end{block}
\end{frame}

\begin{frame}[fragile]
  \frametitlecpp[98]{Exceptions}
  \begin{block}{A more illustrative example}
    \begin{itemize}
      \item exceptions are very powerful when there is much code between the error and where the error is handled
      \item they can also rather cleanly handle different types of errors
      \item \mintinline{cpp}{try}/\mintinline{cpp}{catch} statements can also be nested
    \end{itemize}
  \end{block}
  \begin{multicols}{2}
    \begin{cppcode*}{fontsize=\scriptsize,gobble=2}
      try {
        for (File const &f : files) {
          try {
            process_file(f);
          }
          catch (const bad_file &e) {
            ... // loop continues
          }
        }
      } catch (const bad_db &e) {
        ... // loop aborted
      }
    \end{cppcode*}
    \columnbreak
    \begin{cppcode*}{fontsize=\scriptsize,gobble=2}
      void process_file(const File &file) {
        ...
        if (handle = open_file(file))
          throw bad_file(file.status());
        while (!handle) {
          line = read_line(handle);
          database.insert(line); // can throw
                                 // bad_db
        }
      }
    \end{cppcode*}
  \end{multicols}
\end{frame}

\begin{frame}[fragile]
  \frametitlecpp[98]{Exceptions}
  \begin{block}{Catching everything}
    \begin{itemize}
    \item sometimes we need to catch all possible exceptions
    \item e.g. in \mintinline{cpp}{main}, a thread, a destructor, interfacing with C, \ldots
    \end{itemize}
  \end{block}
  \begin{cppcode}

    try {
      callUnknownFramework();
    } catch(const std::exception& e) {
      // catches std::exception and all derived types
      std::cerr << "Exception: " << e.what() << std::endl;
    } catch(...) {
      // catches everything else
      std::cerr << "Unknown exception type" << std::endl;
    }
  \end{cppcode}
\end{frame}

\begin{frame}[fragile]
  \frametitlecpp[98]{Error Handling and Exceptions}
  \begin{block}{}
    \begin{itemize}
      \item exceptions have little cost if no exception is thrown
      \begin{itemize}
        \item they are recommended to report \textit{exceptional} errors
      \end{itemize}
      \item for performance, when error raising and handling are close, or errors occur often, prefer error codes or a dedicated class
      \item when in doubt about which error strategy is better, profile!
   \end{itemize}
  \end{block}
  \begin{multicols}{2}
    \begin{minipage}{5cm}
      \begin{alertblock}{Avoid}
        \begin{cppcode*}{fontsize=\scriptsize,gobble=6,linenos=false}
          for (string const &num: nums) {
            try {
              int i = convert(num); // can
                                    // throw
              process(i);
            } catch (not_an_int const &e) {
              ... // log and continue
            }
          }
        \end{cppcode*}
      \end{alertblock}
    \end{minipage}
    \columnbreak
    \begin{minipage}{5cm}
      \begin{exampleblock}{Prefer}
        \begin{cppcode*}{fontsize=\scriptsize,gobble=6,linenos=false}
          for (string const &num: nums) {
            optional<int> i = convert(num);
            if (i) {
              process(*i);
            } else {
              ... // log and continue
            }
          }
        \end{cppcode*}
      \end{exampleblock}
    \end{minipage}
  \end{multicols}
\end{frame}


\begin{frame}[fragile]
  \frametitlecpp[11]{noexcept specifier}
  \begin{block}{}
    \begin{itemize}
      \item a function with the \mintinline{cpp}{noexcept} specifier states that it guarantees to not throw an exception
      \begin{cppcode*}{gobble=2,linenos=false}
        int f() noexcept;
      \end{cppcode*}
      \begin{itemize}
        \item either no exceptions will be thrown or they are handled internally
        \item checked at compile time, so it allows the compiler to optimise around that knowledge
      \end{itemize}
      \item a function with \mintinline{cpp}{noexcept(expression)} is only \mintinline{cpp}{noexcept} when \mintinline{cpp}{expression} evaluates to \mintinline{cpp}{true} at compile-time
      \begin{cppcode*}{gobble=2,linenos=false}
        int safe_if_8B_long() noexcept(sizeof(long)==8);
      \end{cppcode*}
      \item Use \mintinline{cpp}{noexcept} on leaf functions where you know the behaviour
      \item C++11 destructors are \mintinline{cpp}{noexcept} - never throw from them
    \end{itemize}
  \end{block}
\end{frame}


\begin{frame}[fragile]
  \frametitlecpp[11]{noexcept operator}
  \begin{block}{}
    \begin{itemize}
      \item the \mintinline{cpp}{noexcept(expression)} operator checks at compile-time whether an expression can throw exceptions
      \item it returns a \mintinline{cpp}{bool}, which is \mintinline{cpp}{true} if no exceptions can be thrown
    \end{itemize}
  \end{block}
  \begin{block}{}
    \begin{cppcode*}{gobble=2, linenos=false}
      constexpr bool callCannotThrow = noexcept(f());
      if constexpr (callCannotThrow) { ... }
    \end{cppcode*}
  \end{block}
  \begin{block}{}
    \begin{cppcode*}{gobble=2, linenos=false}
      template <typename Function>
      void g(Function f) noexcept(noexcept(f())) {
        ...
        f();
      }
    \end{cppcode*}
  \end{block}
\end{frame}
