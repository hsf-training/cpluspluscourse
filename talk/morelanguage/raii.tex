\subsection[RAII]{RAII and smart pointers}

\begin{frame}[fragile]
  \frametitlecpp[98]{Pointers: why are they error prone?}
  \begin{exampleblock}{They need initialization
      \hfill \onslide<2->{\textcolor{orange}{\bf Seg Fault}}}
    \begin{cppcode*}{xleftmargin=20pt}
      char *s;
      try {
        foo(); // may throw
        s = new char[100];
        read_line(s);
      } catch (...) { ... }
      process_line(s);
    \end{cppcode*}
  \end{exampleblock}
  \pause
  \pause
  \vspace{-2cm}
  \begin{exampleblock}{They need to be released
      \hfill \onslide<4->{\textcolor{orange}{\bf Memory leak}}}
    \begin{cppcode*}{xleftmargin=20pt}
      char *s = new char[100];
      read_line(s);
      if (s[0] == '#') return;
      process_line(s);
      delete[] s;
    \end{cppcode*}
  \end{exampleblock}
  \pause
  \pause
  \vspace{-2cm}
  \begin{exampleblock}{They need clear ownership
      \hfill \onslide<6->{\textcolor{orange}{\bf Who should release ?}}}
    \begin{cppcode*}{xleftmargin=20pt}
      char *s = new char[100];
      read_line(s);
      vec.push_back(s);
      set.add(s);
      std::thread t1{func1, vec};
      std::thread t2{func2, set};
    \end{cppcode*}
  \end{exampleblock}
\end{frame}

\begin{frame}[fragile]
  \frametitlecpp[11]{This problem exists for any resource}
  \begin{exampleblock}{For example with a file}
    \begin{cppcode*}{}
      std::FILE *handle = std::fopen(path, "w+");
      if (nullptr == handle) { throw ... }
      std::vector v(100, 42);
      write(handle, v);
      if (std::fputs("end", handle) == EOF) {
        return;
      }
      std::fclose(handle);
    \end{cppcode*}
  \end{exampleblock}
  \begin{block}{}
    Which problems do you spot in the above snippet?
  \end{block}
\end{frame}

\begin{frame}
  \frametitlecpp[98]{Resource Acquisition Is Initialization (RAII)}
  \begin{block}{Practically}
    Use variable construction/destruction and scope semantics:
    \begin{itemize}
    \item wrap the resource inside a class
    \item acquire resource in constructor
    \item release resource in destructor
    \item create an instance on the stack
    \begin{itemize}
      \item automatically destructed when leaving the scope
      \item including in case of exception
    \end{itemize}
    \item use move semantics to pass the resource around
    \end{itemize}
  \end{block}
\end{frame}

\begin{frame}[fragile]
  \frametitlecpp[98]{RAII in practice}
  \begin{exampleblock}{An RAII File class}
    \small
    \begin{cppcode*}{gobble=2}
      class File {
      public:
        // constructor: acquire resource
        File(const char* filename)
          : m_handle(std::fopen(filename, "w+")) {
          // abort constructor on error
          if (m_handle == nullptr) { throw ... }
        }
        // destructor: release resource
        ~File() { std::fclose(m_handle); }
        void write (const char* str) {
          ...
        }
      private:
        std::FILE* m_handle; // wrapped resource
      };
    \end{cppcode*}
  \end{exampleblock}
\end{frame}

\begin{frame}[fragile]
  \frametitlecpp[98]{RAII usage}
  \begin{exampleblock}{Usage of File class}
    \begin{cppcode*}{}
      void log_function() {
        // file opening, aka resource acquisition
        File logfile("logfile.txt");

        // file usage
        logfile.write("hello logfile!"); // may throw

        // file is automatically closed by the call to
        // its destructor, even in case of exception!
      }
    \end{cppcode*}
  \end{exampleblock}
  \begin{goodpracticeWithShortcut}{Use \texttt{std::fstream} for file handling}{Use \texttt{std::fstream}}
     The standard library provides \cppinline{std::fstream} to handle files, use it!
  \end{goodpracticeWithShortcut}
\end{frame}

\begin{frame}[fragile]
  \frametitlecpp[11]{\texttt{std::unique\_ptr}}
  \begin{block}{A RAII pointer}
    \begin{itemize}
    \item wraps and behaves like a regular pointer
    \item get underlying pointer using \cppinline{get()}
    \item when destroyed, deletes the object pointed to
    \item has move-only semantic
      \begin{itemize}
      \item the pointer has unique ownership
      \item copying will result in a compile error
      \end{itemize}
    \end{itemize}
  \end{block}
  \pause
  \begin{exampleblock}{}
    \begin{cppcode*}{}
      #include <memory>
      void f(std::unique_ptr<Foo> ptr) {
        ptr->bar();
      } // deallocation when f exits

      std::unique_ptr<Foo> p{ new Foo{} }; // allocation
      f(std::move(p)); // transfer ownership
      assert(p.get() == nullptr);
    \end{cppcode*}
  \end{exampleblock}
\end{frame}

\begin{frame}[fragile]
  \frametitlecpp[11]{Quiz}
  \begin{exampleblock}{What do you expect?}
    \begin{cppcode*}{}
      void f(std::unique_ptr<Foo> ptr);
      std::unique_ptr<Foo> uptr(new Foo{});
      f(uptr); // transfer of ownership
    \end{cppcode*}
  \end{exampleblock}
  \pause
  \begin{alertblock}{Compilation Error - \godboltLink{https://godbolt.org/z/jfqKjocnh}{\texttt{unique\_ptr} copy}}
    \begin{minted}{text}
test.cpp:15:5: error: call to deleted constructor
of 'std::unique_ptr<Foo>'
  f(uptr);
    ^~~~
/usr/include/c++/4.9/bits/unique_ptr.h:356:7: note:
 'unique_ptr' has been explicitly marked deleted here
 unique_ptr(const unique_ptr&) = delete;
 ^
    \end{minted}
  \end{alertblock}
\end{frame}

\begin{frame}[fragile]
  \frametitlecpp[14]{\texttt{std::make\_unique}}
  \begin{block}{\texttt{std::make\_unique}}
    \begin{itemize}
      \item allocates and constructs an object with arguments\\
            and wraps it with \cppinline{std::unique_ptr} in one step
      \item no \cppinline{new} or \cppinline{delete} calls anymore!
      \item no memory leaks if used consistently
    \end{itemize}
  \end{block}
  \pause
  \begin{exampleblock}{\texttt{std::make\_unique} usage}
    \begin{cppcode*}{}
      {
        // calls new File("logfile.txt") internally
        auto f = std::make_unique<File>("logfile.txt");
        f->write("hello logfile!");
      } // deallocation at end of scope
    \end{cppcode*}
  \end{exampleblock}
\end{frame}

\begin{advanced}

\begin{frame}[fragile]
  \frametitlecpp[11]{Arrays and \texttt{unique\_ptr}}
  \begin{block}{Dynamic arrays}
    \begin{itemize}
      \item \cppinline{unique_ptr} can wrap arrays
      \item Use \cppinline{T[]} as template parameter
      \item This will enable the subscript operator
      \item The default constructor is called for each element
      \item If size known at compile time, prefer \cppinline{std::array}
      \item If size might change, prefer \cppinline{std::vector}
    \end{itemize}
  \end{block}
  \begin{exampleblock}{}
    \begin{cppcode*}{}
      auto b = std::make_unique<Foo[]>(10);
      b[3] = ...;
      b[4].someFunction();

      // deallocations at end of scope
    \end{cppcode*}
  \end{exampleblock}
\end{frame}

\end{advanced}

\begin{frame}[fragile]
  \frametitlecpp[11]{RAII or raw pointers}
  \begin{block}{When to use what?}
    \begin{itemize}
    \item Always use RAII for resources, in particular allocations
    \begin{itemize}
      \item You thus never have to release / deallocate yourself
    \end{itemize}
    \item Use raw pointers as non-owning, re-bindable observers
    \item Remember that \cppinline{std::unique_ptr} is move only
    \end{itemize}
  \end{block}
  \pause
  \begin{exampleblock}{A question of ownership}
    \small
    \begin{cppcode*}{}
      std::unique_ptr<T> produce();
      void observe(const T&);
      void modifyRef(T&);
      void modifyPtr(T*);
      void consume(std::unique_ptr<T>);
      std::unique_ptr<T> pt{produce()}; // Receive ownership
      observe(*pt);                     // Keep ownership
      modifyRef(*pt);                   // Keep ownership
      modifyPtr(pt.get());              // Keep ownership
      consume(std::move(pt));           // Transfer ownership
    \end{cppcode*}
  \end{exampleblock}
\end{frame}

\begin{frame}[fragile]
  \frametitlecpp[11]{\texttt{std::unique\_ptr} usage summary}
  \begin{goodpractice}{\texttt{std::unique\_ptr}}
    \begin{itemize}
      \item \cppinline{std::unique_ptr} is about lifetime management
      \begin{itemize}
        \item use it to tie the lifetime of an object to a unique RAII owner
        \item use raw pointers/references to refer to another object\\
              without owning it or managing its lifetime
      \end{itemize}
      \item use \cppinline{std::make_unique} for creation
      \item strive for having no \cppinline{new}/\cppinline{delete} in your code
      \item for dynamic arrays, \cppinline{std::vector} may be more useful
    \end{itemize}
  \end{goodpractice}
\end{frame}

\begin{frame}[fragile]
  \frametitlecpp[11]{\texttt{std::shared\_ptr}}
  \begin{block}{\mintinline{cpp}{std::shared_ptr} : a reference counting pointer}
    \begin{itemize}
    \item wraps a regular pointer similar to \cppinline{unique_ptr}
    \item has move and copy semantic
    \item uses reference counting internally
      \begin{itemize}
      \item "Would the last person out, please turn off the lights?"
      \end{itemize}
    \item reference counting is thread-safe, therefore a bit costly
    \end{itemize}
  \end{block}
  \begin{block}{\texttt{std::make\_shared} : creates a \texttt{std::shared\_ptr}}
    \begin{cppcode*}{}
      {
        auto sp = std::make_shared<Foo>(); // #ref = 1
        vector.push_back(sp);              // #ref = 2
        set.insert(sp);                    // #ref = 3
      } // #ref 2
    \end{cppcode*}
  \end{block}
\end{frame}

\begin{advanced}

\begin{frame}[fragile]
  \frametitlecpp[11]{\texttt{weak\_ptr}}
  \begin{block}{\texttt{weak\_ptr}: a non-owning observer}
    \small
    \begin{itemize}
    \item Sometimes want to observe resources without keeping them alive
    \begin{itemize}
      \item \cppinline{shared_ptr}? Resource stays alive
      \item Raw pointer? Risk of dangling pointer
    \end{itemize}
    \item The solution is to construct a \cppinline{weak_ptr} from a shared pointer
    \item To access the resource, convert the weak into a \cppinline{shared_ptr}
    \end{itemize}
  \end{block}
  \begin{exampleblock}{}
    \small
    \begin{cppcode*}{}
      std::shared_ptr<Cache> getSharedCache();
      std::weak_ptr<Cache> weakPtr{ getSharedCache() };
      // ... shared cache may be invalidated here
      if (std::shared_ptr<Cache> cache = weakPtr.lock()) {
        // Cache is alive, we actively extend its lifetime
        return cache->findItem(...);
      } else {
        // Cache is nullptr, we need to do something
        weakPtr = recomputeCache(...);
    \end{cppcode*}
  \end{exampleblock}
\end{frame}

\end{advanced}

\begin{frame}[fragile]
  \frametitlecpp[11]{Quiz: \texttt{std::shared\_ptr} in use}
  \begin{exampleblockGB}{What is the output of this code?}{https://godbolt.org/z/vM35Y6qEW}{\texttt{std::shared\_ptr} quiz}
    \small
    \begin{cppcode*}{gobble=2}
      auto shared = std::make_shared<int>(100);
      auto print = [shared](){
        std::cout << "Use: " << shared.use_count() << " "
                  << "value: " << *shared << "\n";
      };
      print();
      {
        auto ptr{ shared };
        (*ptr)++;
        print();
      }
      print();
    \end{cppcode*}
  \end{exampleblockGB}
  \pause
  \begin{block}{}
    \small
    \begin{minted}[autogobble=true]{bash}
      Use: 2 value: 100
      Use: 3 value: 101
      Use: 2 value: 101
    \end{minted}
  \end{block}
\end{frame}

\begin{frame}[fragile]
  \frametitlecpp[11]{Quiz: \texttt{std::shared\_ptr} in use}
  \begin{exampleblock}{What is the output of this code?}
    \small
    % escapeinside seems to break gobble, so need to un-indent manually
    \begin{cppcode*}{escapeinside=@@}
auto shared = std::make_shared<int>(100);
auto print = [@\textcolor{red}{&}@shared](){
  std::cout << "Use: " << shared.use_count() << " "
            << "value: " << *shared << "\n";
};
print();
{
  auto ptr{ shared };
  (*ptr)++;
  print();
}
print();
      \end{cppcode*}
  \end{exampleblock}
  \begin{block}{}
    \small
    \begin{minted}[autogobble=true]{bash}
      Use: 1 value: 100
      Use: 2 value: 101
      Use: 1 value: 101
    \end{minted}
  \end{block}
\end{frame}

\begin{advanced}

\begin{frame}[fragile]
  \frametitlecpp[11]{Quiz: \texttt{shared\_ptr} and \texttt{weak\_ptr} in use}
  \begin{exampleblockGB}{What is the output of this code?}{https://godbolt.org}{\texttt{shared\_ptr} vs.\ \texttt{weak\_ptr} quiz}
    \small
    \begin{cppcode*}{gobble=2}
      auto shared = std::make_shared<int>(100);
      std::weak_ptr<int> weak{ shared };
      print(); // with print as before

      auto ptr = weak.lock();
      (*ptr)++;       print();

      ptr = nullptr;  print();

      function(weak); print();
    \end{cppcode*}
  \end{exampleblockGB}
  \pause
  \begin{block}{}
    \small
    \begin{minted}[autogobble=true]{bash}
      Use: 1 value: 100
      Use: 2 value: 101
      Use: 1 value: 101
      Use: 1 (or more) value: ???
    \end{minted}
  \end{block}
\end{frame}

\end{advanced}

\begin{frame}[fragile]
    \frametitlecpp[11]{Rule of zero}
    \begin{goodpracticeWithShortcut}{Single responsibility principle (SRP)}{Single responsibility principle}
        Every class should have only one responsibility.
    \end{goodpracticeWithShortcut}
    \begin{goodpractice}{Rule of zero}
        \begin{itemize}
            \item If your class has any special member functions (except ctor.)
            \begin{itemize}
                \item Your class probably deals with a resource, use RAII
                \item Your class should only deal with this resource (SRP)
                \item Apply rule of 3/5: write/default/delete all special members
            \end{itemize}
            \item Otherwise: do not declare any special members (rule of zero)
            \begin{itemize}
                \item A constructor is fine, if you need some setup
                \item If your class holds a resource as data member:\\
                      wrap it in a smart pointer, container, or any other RAII class
            \end{itemize}
        \end{itemize}
    \end{goodpractice}
\end{frame}

\begin{frame}[fragile]
  \frametitlecpp[98]{smart pointers}
  \begin{exercise}{Smart pointers}
    \begin{itemize}
    \item go to \texttt{exercises/smartPointers}
    \item compile and run the program. It doesn't generate any output.
    \item Run with valgrind if possible to check for leaks
      { \scriptsize
      \begin{minted}[gobble=6]{shell-session}
        $ valgrind --leak-check=full --track-origins=yes ./smartPointers
      \end{minted}
      }
    \item In the \emph{essentials course}, go through {\ttfamily problem1()} and {\ttfamily problem2()} and fix the leaks using smart pointers.
    \item In the \emph{advanced course}, go through {\ttfamily problem1()} to {\ttfamily problem4()} and fix the leaks using smart pointers.
    \item {\ttfamily problem4()} is the most difficult. Skip if not enough time.
    \end{itemize}
  \end{exercise}
\end{frame}
