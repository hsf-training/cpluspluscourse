\subsection[RAII]{Pointers and RAII}

\begin{frame}[fragile]
  \frametitlecpp[98]{Pointers: why are they error prone?}
  \begin{exampleblock}{They need initialization
      \hfill \onslide<2->{\textcolor{orange}{\bf Seg Fault}}}
    \begin{cppcode*}{xleftmargin=20pt}
      char *s;
      try {
        foo(); // may throw
        s = new char[100];
        read_line(s);
      } catch (...) { ... }
      process_line(s);
    \end{cppcode*}
  \end{exampleblock}
  \pause
  \pause
  \vspace{-2cm}
  \begin{exampleblock}{They need to be released
      \hfill \onslide<4->{\textcolor{orange}{\bf Memory leak}}}
    \begin{cppcode*}{xleftmargin=20pt}
      char *s = new char[100];
      read_line(s);
      if (s[0] == '#') return;
      process_line(s);
      delete[] s;
    \end{cppcode*}
  \end{exampleblock}
  \pause
  \pause
  \vspace{-2cm}
  \begin{exampleblock}{They need clear ownership
      \hfill \onslide<6->{\textcolor{orange}{\bf Who should release ?}}}
    \begin{cppcode*}{xleftmargin=20pt}
      char *s = new char[100];
      read_line(s);
      vec.push_back(s);
      set.add(s);
      std::thread t1(func1, vec);
      std::thread t2(func2, set);
    \end{cppcode*}
  \end{exampleblock}
\end{frame}

\begin{frame}[fragile]
  \frametitlecpp[11]{This problem exists for any resource}
  \begin{exampleblock}{For example with a file}
    \begin{cppcode*}{}
      std::FILE *handle = std::fopen(path, "w+");
      if (nullptr == handle) { throw ... }
      std::vector v(100, 42);
      write(handle, v);
      if (std::fputs("end", handle) == EOF) {
        return;
      }
      std::fclose(handle);
    \end{cppcode*}
  \end{exampleblock}
  \begin{block}{}
    Which problems do you spot in the above snippet?
  \end{block}
\end{frame}

\begin{frame}
  \frametitlecpp[98]{Resource Acquisition Is Initialization (RAII)}
  \begin{block}{Practically}
    Use object semantic to acquire/release resources
    \begin{itemize}
    \item wrap the resource inside an object
    \item acquire resource in constructor
    \item release resource in destructor
    \item create this object on the stack so that it is automatically destructed when leaving the scope, including in case of exception
    \item use move semantics to pass the resource around
    \end{itemize}
  \end{block}
\end{frame}

\begin{frame}[fragile]
  \frametitlecpp[98]{RAII in practice}
  \begin{exampleblock}{File class}
    \begin{cppcode*}{}
      class File {
      public:
        File(const char* filename) :
          m_handle(std::fopen(filename, "w+")) {
          if (m_handle == nullptr) { throw ... }
        }
        ~File() { std::fclose(m_handle); }
        void write (const char* str) {
          if (std::fputs(str, m_handle) == EOF) {
            throw ...
          }
        }
      private:
        std::FILE* m_handle;
      };
    \end{cppcode*}
  \end{exampleblock}
\end{frame}

\begin{frame}[fragile]
  \frametitlecpp[98]{RAII usage}
  \begin{exampleblock}{Usage of File class}
    \begin{cppcode*}{}
      void log_function() {
        // file opening, aka resource acquisition
        File logfile("logfile.txt") ;

        // file usage
        logfile.write("hello logfile!") ;

        // file is automatically closed by the call to
        // its destructor, even in case of exception !
      }
    \end{cppcode*}
  \end{exampleblock}
  \begin{goodpracticeWithShortcut}{Use \texttt{std::fstream} for file handling}{Use \texttt{std::fstream}}
     The standard library provides \mintinline{cpp}{std::fstream} to handle files, use it!
  \end{goodpracticeWithShortcut}
\end{frame}

\begin{frame}[fragile]
  \frametitlecpp[11]{std::unique\_ptr}
  \begin{block}{A RAII pointer}
    \begin{itemize}
    \item wraps a regular pointer
    \item retrieve it using \mintinline{cpp}{get()}
    \item has move-only semantic
      \begin{itemize}
      \item the pointer has unique ownership
      \item copying will result in a compile error
      \end{itemize}
    \item in \mintinline{cpp}{<memory>} header
    \end{itemize}
  \end{block}
  \pause
  \begin{exampleblock}{Usage}
    \begin{cppcode*}{}
      std::unique_ptr<Foo> p{ new Foo{} }; // allocation
      void f(std::unique_ptr<Foo> ptr);

      f(std::move(p)); // transfer ownership
      // deallocation when exiting f
      assert(p.get() == nullptr);
    \end{cppcode*}
  \end{exampleblock}
\end{frame}

\begin{frame}[fragile]
  \frametitlecpp[11]{Quiz}
  \begin{exampleblock}{}
    \begin{cppcode*}{}
      Foo *p = new Foo{};  // allocation
      std::unique_ptr<Foo> uptr(p);
      void f(std::unique_ptr<Foo> ptr);
      f(uptr); // transfer of ownership
    \end{cppcode*}
    What do you expect ?
  \end{exampleblock}
  \pause
  \begin{alertblock}{Compilation Error - \godboltLink{https://godbolt.org/z/jfqKjocnh}{\texttt{unique\_ptr} copy}}
    \begin{minted}{text}
test.cpp:15:5: error: call to deleted constructor
of 'std::unique_ptr<Foo>'
  f(uptr);
    ^~~~
/usr/include/c++/4.9/bits/unique_ptr.h:356:7: note:
 'unique_ptr' has been explicitly marked deleted here
 unique_ptr(const unique_ptr&) = delete;
 ^
    \end{minted}
  \end{alertblock}
\end{frame}

\begin{frame}[fragile]
  \frametitlecpp[14]{std::make\_unique}
  \begin{block}{}
    \begin{itemize}
      \item directly allocates the object managed by a \mintinline{cpp}{unique_ptr}
      \item no \mintinline{cpp}{new} or \mintinline{cpp}{delete} calls anymore!
      \item no memory leaks if used everywhere
    \end{itemize}
  \end{block}
  \pause
  \begin{exampleblock}{make\_unique usage}
    \begin{cppcode*}{}
      // allocation of one Foo object,
      // calls new Foo(arg1, arg2) internally
      auto a = std::make_unique<Foo>(arg1, arg2);
      std::cout << a.get() << " points to "
                << a->someMember << '\n';

      // deallocations at end of scope
    \end{cppcode*}
  \end{exampleblock}
  \begin{block}{}
    \begin{minted}[autogobble=true]{text}

      0x6000005fe9a0 points to 5
    \end{minted}
  \end{block}
\end{frame}

\begin{advanced}

\begin{frame}[fragile]
  \frametitlecpp[11]{Arrays and unique\_ptr}
  \begin{block}{Dynamic arrays}
    \begin{itemize}
      \item \mintinline{cpp}{unique_ptr} can wrap arrays
      \item Use \mintinline{cpp}{T[]} as template parameter
      \item This will enable the subscript operator
      \item The default constructor is called for each element
      \item If size known at compile time, prefer \mintinline{cpp}{std::array}
      \item If size might change, prefer \mintinline{cpp}{std::vector}
    \end{itemize}
  \end{block}
  \begin{exampleblock}{}
    \begin{cppcode*}{}
      auto b = std::make_unique<Foo[]>(10);
      b[3] = ...;
      b[4].someFunction();

      // deallocations at end of scope
    \end{cppcode*}
  \end{exampleblock}
\end{frame}

\end{advanced}

\begin{frame}[fragile]
  \frametitlecpp[11]{RAII or raw pointers}
  \begin{block}{When to use what ?}
    \begin{itemize}
    \item Always use RAII for resources, in particular allocations
    \item You thus never have to release / deallocate yourself
    \item Use raw pointers as non-owning, re-bindable observers
    \item Remember that \mintinline{cpp}{unique_ptr} is move only
    \end{itemize}
  \end{block}
  \pause
  \begin{exampleblock}{A question of ownership}
    \begin{cppcode*}{}
      unique_ptr<T> producer();
      void observer(const T&);
      void modifier(T&);
      void consumer(unique_ptr<T>);
      unique_ptr<T> pt{producer()}; // Receive ownership
      observer(*pt);                // Keep ownership
      modifier(*pt);                // Keep ownership
      consumer(std::move(pt));      // Transfer ownership
    \end{cppcode*}
  \end{exampleblock}
\end{frame}

\begin{frame}[fragile]
  \frametitlecpp[11]{unique\_ptr usage summary}
  \begin{block}{It's about lifetime management}
    \begin{itemize}
    \item Use \mintinline{cpp}{unique_ptr} in functions taking part in lifetime management
    \item Otherwise use raw pointers or references
    \end{itemize}
  \end{block}
\end{frame}

\begin{frame}[fragile]
  \frametitlecpp[11]{shared\_ptr, make\_shared}
  \begin{block}{shared\_ptr : a reference counting pointer}
    \begin{itemize}
    \item wraps a regular pointer similar to \mintinline{cpp}{unique_ptr}
    \item has move and copy semantic
    \item uses reference counting internally
      \begin{itemize}
      \item "Would the last person out, please turn off the lights?"
      \end{itemize}
    \item reference counting is thread-safe, therefore a bit costly
    \end{itemize}
  \end{block}
  \begin{block}{make\_shared : creates a shared\_ptr}
    \begin{cppcode*}{}
      {
        auto sp = std::make_shared<Foo>(); // #ref = 1
        vector.push_back(sp);              // #ref = 2
        set.insert(sp);                    // #ref = 3
      } // #ref 2
    \end{cppcode*}
  \end{block}
\end{frame}

\begin{frame}[fragile]
  \frametitlecpp[11]{weak\_ptr}
  \begin{block}{weak\_ptr: a non-owning observer}
    \small
    \begin{itemize}
    \item Sometimes want to observe resources without keeping them alive
    \begin{itemize}
      \item \mintinline{cpp}{shared_ptr}? Resource stays alive
      \item Raw pointer? Risk of dangling pointer
    \end{itemize}
    \item The solution is to construct a \mintinline{cpp}{weak_ptr} from a shared pointer
    \item To access the resource, convert the weak into a \mintinline{cpp}{shared_ptr}
    \end{itemize}
  \end{block}
  \begin{exampleblock}{}
    \small
    \begin{cppcode*}{}
      std::shared_ptr<Cache> getSharedCache();
      std::weak_ptr<Cache> weakPtr{ getSharedCache() };
      // ... shared cache may be invalidated here
      if (std::shared_ptr<Cache> cache = weakPtr.lock()) {
        // Cache is alive, we actively extend its lifetime
        return cache->findItem(...);
      } else {
        // Cache is nullptr, we need to do something
        weakPtr = recomputeCache(...);
    \end{cppcode*}
  \end{exampleblock}
\end{frame}

\begin{frame}[fragile]
  \frametitlecpp[11]{Quiz: shared\_ptr and weak\_ptr}
  \begin{exampleblockGB}{What is the output of this code?}{https://godbolt.org/z/vM35Y6qEW}{shared\_ptr vs weak\_ptr}
    \small
    \begin{cppcode*}{gobble=2}
      auto shared = std::make_shared<int>(100);
      auto print = [shared](){
        std::cout << "Use: " << shared.use_count() << " "
                  << "value: " << *shared << "\n";
      };
      print();
      {
        auto ptr{ shared };
        (*ptr)++;
        print();
      }
      print();
    \end{cppcode*}
  \end{exampleblockGB}
  \pause
  \begin{block}{}
    \small
    \begin{minted}[autogobble=true]{bash}
      Use: 2 value: 100
      Use: 3 value: 101
      Use: 2 value: 101
    \end{minted}
  \end{block}
\end{frame}

\begin{frame}[fragile]
  \frametitlecpp[11]{Quiz: shared\_ptr and weak\_ptr}
  \begin{exampleblockGB}{What is the output of this code?}{https://godbolt.org/z/vM35Y6qEW}{shared\_ptr vs weak\_ptr}
    \small
    % escapeinside seems to break gobble, so need to un-indent manually
    \begin{cppcode*}{escapeinside=@@}
auto shared = std::make_shared<int>(100);
auto print = [@\textcolor{red}{&}@shared](){
  std::cout << "Use: " << shared.use_count() << " "
            << "value: " << *shared << "\n";
};
print();
{
  auto ptr{ shared };
  (*ptr)++;
  print();
}
print();
      \end{cppcode*}
  \end{exampleblockGB}
  \begin{block}{}
    \small
    \begin{minted}[autogobble=true]{bash}
      Use: 1 value: 100
      Use: 2 value: 101
      Use: 1 value: 101
    \end{minted}
  \end{block}
\end{frame}

\begin{frame}[fragile]
  \frametitlecpp[11]{Quiz: shared\_ptr and weak\_ptr}
  \begin{exampleblockGB}{What is the output of this code?}{https://godbolt.org}{shared\_ptr vs weak\_ptr}
    \small
    \begin{cppcode*}{gobble=2}
      auto shared = std::make_shared<int>(100);
      std::weak_ptr<int> weak{ shared };
      print(); // with print as before

      auto ptr = weak.lock();
      (*ptr)++;       print();

      ptr = nullptr;  print();

      function(weak); print();
    \end{cppcode*}
  \end{exampleblockGB}
  \pause
  \begin{block}{}
    \small
    \begin{minted}[autogobble=true]{bash}
      Use: 1 value: 100
      Use: 2 value: 101
      Use: 1 value: 101
      Use: 1 (or more) value: ???
    \end{minted}
  \end{block}
\end{frame}

\begin{frame}[fragile]
  \frametitlecpp[98]{smart pointers}
  \begin{exercise}{Smart pointers}
    \begin{itemize}
    \item go to code/smartPointers
    \item compile and run the program. It doesn't generate any output.
    \item Run with valgrind to check for leaks
      { \scriptsize
      \begin{minted}[gobble=6]{shell-session}
        $ valgrind --leak-check=full --track-origins=yes ./smartPointers
      \end{minted}
      }
    \item Go through {\ttfamily problem1()} to {\ttfamily problem3()} and fix the leaks using smart pointers.
    \item {\ttfamily problem4()} is the most difficult. Skip if not enough time.
    \end{itemize}
  \end{exercise}
\end{frame}
