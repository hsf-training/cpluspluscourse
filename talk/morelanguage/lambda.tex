\subsection[$\lambda$]{Lambdas}

\begin{frame}[fragile]
  \frametitlecpp[11]{Function return type}
  \begin{block}{A new way to specify function's return type}
    \begin{cppcode*}{linenos=false}
      ReturnType fn_name(ArgType1, ArgType2);  //old
      auto fn_name(ArgType1, ArgType2) -> ReturnType;
    \end{cppcode*}
  \end{block}
  \pause
  \begin{block}{Advantages}
    \begin{itemize}
    \item Allows to simplify inner type definition
      \begin{cppcode*}{gobble=4}
        class TheClass {
          using inner_type = int;
          inner_type func();
        }
        TheClass::inner_type TheClass::func() {...}
        auto TheClass::func() -> inner_type {...}
      \end{cppcode*}
    \item C++14: ReturnType is not mandatory, compiler can deduce it
    \item will be used for lambdas
    \end{itemize}
  \end{block}
\end{frame}


\begin{frame}[fragile]
  \frametitlecpp[11]{Lambdas}
  \begin{block}{Definition}
    a lambda is a function with no name
  \end{block}
  \pause
  \begin{exampleblock}{Python example}
    \begin{pythoncode*}{}
      data = [1,9,3,8,3,7,4,6,5]

      # without lambdas
      def isOdd(n):
        return n%2 == 1
      print(filter(isOdd, data))

      # with lambdas
      print(filter(lambda n:n%2==1, data))
    \end{pythoncode*}
  \end{exampleblock}
\end{frame}

\begin{frame}[fragile]
  \frametitlecpp[11]{\cpp Lambdas}
  \begin{block}{Simplified syntax}
    \begin{cppcode*}{gobble=2}
      [] (args) -> type {
        code;
      }
    \end{cppcode*}
    The type specification is optional
  \end{block}
  \begin{exampleblock}{Usage example}
    \begin{cppcode*}{firstnumber=4,gobble=2}
      std::vector<int> data{1,2,3,4,5};
      for_each(begin(data), end(data),
               [](int i) {
                 std::cout << "The square of " << i
                           << " is " << i*i << std::endl;
               });
    \end{cppcode*}
  \end{exampleblock}
\end{frame}


\begin{frame}[fragile]
  \frametitlecpp[11]{Capture}
  \begin{block}{Python code}
    \begin{pythoncode*}{}
      increment = 3
      data = [1,9,3,8,3,7,4,6,5]
      map(lambda x : x + increment, data)
    \end{pythoncode*}
  \end{block}
  \pause
  \begin{block}{First attempt in \cpp}
    \begin{cppcode*}{firstnumber=4}
      int increment = 3;
      std::vector<int> data{1,9,3,8,3,7,4,6,5};
      transform(begin(data), end(data), begin(data),
                [](int x) { return x+increment; });
    \end{cppcode*}
  \end{block}
  \pause
  \begin{alertblock}{Error}
    \begin{minted}[gobble=6]{text}
        error: 'increment' is not captured
          [](int x) { return x+increment; });
                                     ^
    \end{minted}
  \end{alertblock}
\end{frame}

\begin{frame}[fragile]
  \frametitlecpp[11]{Capture}
  \begin{block}{Variable capture}
    \begin{itemize}
    \item external variables need to be explicitly captured
    \item captured variables are listed within initial []
    \end{itemize}
  \end{block}
  \pause
  \begin{exampleblock}{Example}
    \begin{cppcode*}{}
      int increment = 3;
      std::vector<int> data{1,9,3,8,3,7,4,6,5};
      transform(begin(data), end(data), begin(data),
                [increment](int x) {
                  return x+increment;
                });
    \end{cppcode*}
  \end{exampleblock}
\end{frame}

\begin{frame}[fragile]
  \frametitlecpp[11]{Default capture is by value}
  \begin{exampleblock}{Code example}
    \begin{cppcode}
      int sum = 0;
      std::vector<int> data{1,9,3,8,3,7,4,6,5};
      for_each(begin(data), end(data),
              [sum](int x) { sum += x; });
    \end{cppcode}
  \end{exampleblock}
  \pause
  \begin{alertblock}{Error}
    \begin{minted}[gobble=4]{text}
      error: assignment of read-only variable 'sum'
               [sum](int x) { sum += x; });
    \end{minted}
  \end{alertblock}
  \pause
  \begin{block}{Explanation}
    By default, variables are captured by value
  \end{block}
\end{frame}

\begin{frame}[fragile]
  \frametitlecpp[11]{Capture by reference}
  \begin{exampleblock}{Simple example}
    In order to capture by reference, add '\&' before the variable
    \begin{cppcode*}{}
      int sum = 0;
      std::vector<int> data{1,9,3,8,3,7,4,6,5};
      for_each(begin(data), end(data),
              [&sum](int x) { sum += x; });
    \end{cppcode*}
  \end{exampleblock}
  \pause
  \begin{exampleblock}{Mixed case}
    One can of course mix values and references
    \begin{cppcode*}{firstnumber=5}
      int sum = 0, offset = 1;
      std::vector<int> data{1,9,3,8,3,7,4,6,5};
      for_each(begin(data), end(data),
              [&sum, offset](int x) {
                sum += x + offset;
              });
    \end{cppcode*}
  \end{exampleblock}
\end{frame}

\begin{frame}[fragile]
  \frametitlecpp[11]{Capture list}
  \begin{block}{all by value}
    \begin{cppcode*}{linenos=false}
      [=](...) { ... };
    \end{cppcode*}
  \end{block}
  \pause
  \begin{block}{all by reference}
    \begin{cppcode*}{linenos=false}
      [&](...) { ... };
    \end{cppcode*}
  \end{block}
  \pause
  \begin{block}{mix}
    \begin{cppcode*}{linenos=false}
      [&, b](...) { ... };
      [=, &b](...) { ... };
    \end{cppcode*}
  \end{block}
\end{frame}

\begin{frame}[fragile]
  \frametitlecpp[11]{Lambdas rather than functors}
  \begin{exampleblock}{Example}
    \begin{cppcode*}{}
      auto build_incrementer = [](int inc) {
        return [inc](int value) { return value + inc; };
      };
      auto inc1 = build_incrementer(1);
      auto inc10 = build_incrementer(10);
      int i = 0;
      i = inc1(i);   // i = 1
      i = inc10(i);  // i = 11
    \end{cppcode*}
  \end{exampleblock}
  \begin{block}{How it works}
    \begin{itemize}
      \item build\_incrementer returns a function object
      \item this function's behavior depends on a parameter
      \item note how \mintinline{cpp}{auto} is useful here!
    \end{itemize}
  \end{block}
\end{frame}

\begin{frame}[fragile]
  \frametitlecpp[11]{{\texttt lambda}s make the STL usable}
  \begin{block}{Before lambdas}
    \begin{cppcode*}{}
      struct Incrementer {
        int m_inc;
        Incrementer(int inc) : m_inc(inc) {}
        int operator() (int value) {
          return value + m_inc;
        };
      };
      std::vector<int> v{1, 2, 3};
      std::transform(begin(v), end(v), begin(v),
                     Incrementer(1));
      for (auto a : v) std::cout << a << " ";
      \end{cppcode*}
    \end{block}
\end{frame}

\begin{frame}[fragile]
  \frametitlecpp[11]{{\texttt lambda}s make the STL usable}
  \begin{exampleblock}{With lambdas}
    \begin{cppcode*}{}
      std::vector<int> v{1, 2, 3};
      std::transform(begin(v), end(v), begin(v),
                     [](int value) {
                       return value + 1;
                     });
      for (auto a : v) std::cout << a << " ";
    \end{cppcode*}
  \end{exampleblock}
  \pause
  \begin{alertblock}{Conclusion}
    Use the STL (with lambdas)!
  \end{alertblock}
\end{frame}

\begin{frame}[fragile]
  \frametitlecpp[11]{Lambdas}
  \begin{alertblock}{Exercise Time}
    \begin{itemize}
    \item go to code/lambdas
    \item look at the code (it's the solution to the stl exercise)
    \item use lambdas to simplify it
    \end{itemize}
  \end{alertblock}
\end{frame}
