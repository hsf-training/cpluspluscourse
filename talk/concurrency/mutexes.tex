\subsection[mutex]{Mutexes}

\begin{frame}[fragile]
  \frametitlecpp[11]{Races}
  \begin{exampleblock}{Example code \godboltLink{https://godbolt.org/z/oGz61Pn19}}
    \begin{cppcode*}{}
      int a = 0;
      void inc() { a++; };
      void inc100() {
        for (int i=0; i < 100; i++) inc();
      };
      int main() {
        std::thread t1{inc100};
        std::thread t2{inc100};
        for (auto t: {&t1,&t2}) t->join();
        std::cout << a << "\n";
      }
    \end{cppcode*}
  \end{exampleblock}
  \pause
  \begin{block}{What do you expect? (Exercise code/race)}
    \pause
    Anything between 100 and 200 !!!
  \end{block}
\end{frame}

\begin{frame}[fragile]
  \frametitlecpp[11]{Atomicity}
  \begin{exampleblock}{Definition (wikipedia)}
    \begin{itemize}
    \item an operation (or set of operations) is atomic if it appears to the rest of the system to occur instantaneously
    \end{itemize}
  \end{exampleblock}
  \begin{block}{Practically}
    \begin{itemize}
    \item an operation that won't be interrupted by other concurrent operations
    \item an operation that will have a stable environment during execution
    \end{itemize}
  \end{block}
  \pause
  \begin{alertblock}{Is \texttt{++} operator atomic ?}
    \pause
    Usually not. It behaves like :
    \begin{cppcode*}{}
      eax = a       // memory to register copy
      increase eax  // increase (atomic CPU instruction)
      a = eax       // copy back to memory
    \end{cppcode*}
  \end{alertblock}
\end{frame}

\begin{frame}[fragile]
  \frametitlecpp[11]{Timing}
  \begin{exampleblock}{Code}
    \begin{cppcode*}{}
      eax = a       // memory to register copy
      increase eax  // increase (atomic CPU instruction)
      a = eax       // copy back to memory
    \end{cppcode*}
  \end{exampleblock}
  \begin{block}{For 2 threads}
    \begin{tikzpicture}
      \begin{umlseqdiag}
        \umlobject[x=0, class=eax]{Thread 1}
        \umlobject[x=3, class=a, fill=blue!20]{Memory}
        \umlobject[x=6, class=eax]{Thread 2}
        \begin{umlcall}[op=read, type=synchron, return=0]{Thread 1}{Memory}
        \end{umlcall}
        \begin{umlcall}[padding=3, op=read, type=synchron, return=0]{Thread 2}{Memory}
        \end{umlcall}
        \begin{umlcallself}[op=incr, type=synchron]{Thread 1}
        \end{umlcallself}
        \begin{umlcallself}[op=incr, type=synchron]{Thread 2}
        \end{umlcallself}
        \begin{umlcall}[op=write 1]{Thread 2}{Memory}
        \end{umlcall}
        \begin{umlcall}[padding=3, op=write 1]{Thread 1}{Memory}
        \end{umlcall}
      \end{umlseqdiag}
      \draw[-triangle 60](8.5,0) -- (8.5,-4) node[right, pos=0.5]{time};
    \end{tikzpicture}
  \end{block}
\end{frame}

\begin{frame}[fragile]
  \frametitlecpp[17]{Mutexes and Locks}
  \begin{block}{Concept}
    \begin{itemize}
    \item Use locks to serialize access to a non-atomic piece of code
    \end{itemize}
  \end{block}
  \pause
  \begin{block}{The objects}
    \begin{description}[labelwidth=1.8cm]
    \item[std::mutex] in the \mintinline{cpp}{mutex} header. \textbf{Mut}ual \textbf{ex}clusion
    \item[std::scoped\_lock] RAII to lock and unlock automatically
    \item[std::unique\_lock] same, but can be released/relocked explicitly
    \end{description}
  \end{block}
  \pause
  \begin{exampleblock}{Practically \godboltLink{https://godbolt.org/z/a5TaaPrad}}
    \begin{cppcode*}{}
      int a = 0;
      std::mutex m;
      void inc() {
        std::scoped_lock lock{m};
        a++;
      }
    \end{cppcode*}
  \end{exampleblock}
\end{frame}

\begin{frame}[fragile]
  \frametitlecpp[17]{Mutexes and Locks}
  \begin{goodpractice}{Locking}
    \begin{itemize}
      \item Generally, use \mintinline{cpp}{std::scoped_lock}. Before \cpp17 use \mintinline{cpp}{std::lock_guard}.
      \item Hold as short as possible, consider wrapping critical section in block statement \mintinline{cpp}|{ }|
      \item Only if manual control needed, use \mintinline{cpp}{std::unique_lock}
    \end{itemize}
  \end{goodpractice}
  \begin{exampleblock}{}
    \begin{cppcode*}{gobble=2}
      void function(...) {
        // uncritical work ...
        {
          std::scoped_lock myLocks{mutex1, mutex2, ...};
          // critical section
        }
        // uncritical work ...
      }
    \end{cppcode*}
  \end{exampleblock}
\end{frame}

\begin{frame}[fragile]
  \frametitle{Mutexes and Locks}
  \begin{exerciseWithShortcut}{Mutexes and Locks}{Mutexes/Locks}
    \begin{itemize}
    \item Go to code/race
    \item Look at the code and try it\\
      See that it has a race condition
    \item Use a mutex to fix the issue
    \item See the difference in execution time
    \end{itemize}
  \end{exerciseWithShortcut}
\end{frame}

\begin{frame}[fragile]
  \frametitlecpp[11]{Dead locks}
  \begin{block}{Scenario}
    \begin{itemize}
    \item 2 mutexes, 2 threads
    \item locking order different in the 2 threads
    \end{itemize}
  \end{block}
  \pause
  \begin{block}{Sequence diagram}
    \begin{tikzpicture}
      \begin{umlseqdiag}
        \umlobject[x=0]{Thread 1}
        \umlobject[x=2.5, fill=blue!20]{Mutex A}
        \umlobject[x=5, fill=blue!20]{Mutex B}
        \umlobject[x=7.5]{Thread 2}
        \begin{umlcall}[op=lock]{Thread 1}{Mutex A}
        \end{umlcall}
        \begin{umlcall}[op=lock, dt=6]{Thread 2}{Mutex B}
        \end{umlcall}
        \begin{umlcall}[op=lock (block), dt=6]{Thread 1}{Mutex B}
        \end{umlcall}
        \begin{umlcall}[op=lock (block), dt=12]{Thread 2}{Mutex A}
        \end{umlcall}
      \end{umlseqdiag}
      \draw[-triangle 60](9,0) -- (9,-4) node[right, pos=0.5]{time};
    \end{tikzpicture}
  \end{block}
\end{frame}

\begin{frame}[fragile]
  \frametitlecpp[11]{How to avoid dead locks}
  \begin{block}{Possible solutions}
    \begin{itemize}
    \item \cpp17: \mintinline{cpp}{std::scoped_lock lock{m1, m2};} comes with deadlock-avoidance algorithm
    \item Never take several locks
      \begin{itemize}
      \item Or add master lock protecting the locking phase
      \end{itemize}
    \item Respect a strict locking order across all threads
    \item Do not use locks
      \begin{itemize}
      \item Use other techniques, e.g.\ lock-free queues
      \end{itemize}
    \end{itemize}
  \end{block}
\end{frame}

\begin{frame}[fragile]
  \frametitlecpp[17]{Shared mutex / locks}
  \begin{block}{Sharing a mutex}
    \begin{itemize}
      \item Normal \mintinline{cpp}{std::mutex} objects cannot be shared
      \item \mintinline{cpp}{std::shared_mutex} to the rescue, but can be slower
      \item It locks \emph{either} in exclusive \emph{or} in shared mode; never both
    \end{itemize}
  \end{block}
  \begin{exampleblock}{}
    \begin{cppcode*}{gobble=2}
      Data data; std::shared_mutex mutex;
      auto reader = [&](){
        std::shared_lock lock{mutex};
        read(data); // Many can read
      };
      std::thread r1{reader}, r2{reader}, ...;

      std::thread writer([&](){
        std::scoped_lock lock{mutex}; // exclusive
        modify(data); // Only one can write
      });
    \end{cppcode*}
  \end{exampleblock}
\end{frame}
