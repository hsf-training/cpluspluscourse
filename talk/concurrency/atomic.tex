\subsection[atomic]{Atomic types}

\begin{frame}[fragile]
  \frametitlecpp[11]{Atomic types in \cpp}
  \begin{block}{\texttt{std::atomic} template}
    \begin{itemize}
      \item Any trivially copyable type can be made atomic in C++
      \item Most useful for integral types
      \item May internally use locks for custom types
    \end{itemize}
  \end{block}
  \begin{exampleblock}{}
    \begin{cppcode*}{}
      std::atomic<int> a{0};
      std::thread t1([&](){ a++; });
      std::thread t2([&](){ a++; });
      a += 2;
      t1.join(); t2.join();
      assert( a == 4 ); // Guaranteed to succeed
    \end{cppcode*}
  \end{exampleblock}
\end{frame}

\begin{frame}[fragile]
  \begin{alertblock}{Warning: Expressions using an atomic type are \textit{not} atomic!}
    \begin{cppcode*}{}
      std::atomic<int> a{0};
      std::thread t1([&]{ a = a + 2; });
      std::thread t2([&]{ a = a + 2; });
    \end{cppcode*}
    \begin{itemize}
      \item Atomic load; value+2; atomic store
    \end{itemize}
  \end{alertblock}
  \begin{block}{Sequence diagram}
    \begin{tikzpicture}
      \begin{umlseqdiag}
        \umlobject[x=0]{Thread 1}
        \umlobject[x=3, fill=blue!20]{atomic}
        \umlobject[x=6]{Thread 2}
        \begin{umlcall}[op=load]{Thread 1}{atomic}
        \end{umlcall}
        \begin{umlcall}[op=+2]{Thread 1}{Thread 1}
        \end{umlcall}
        \begin{umlcall}[op=load]{Thread 2}{atomic}
        \end{umlcall}
        \begin{umlcall}[op=+2]{Thread 2}{Thread 2}
        \end{umlcall}
        \begin{umlcall}[op=store 2]{Thread 1}{atomic}
        \end{umlcall}
        \begin{umlcall}[op=store 2]{Thread 2}{atomic}
        \end{umlcall}
      \end{umlseqdiag}
      \draw[-triangle 60](9,0) -- (9,-4) node[right, pos=0.5]{time};
    \end{tikzpicture}
  \end{block}
\end{frame}

\begin{frame}[fragile]
  \begin{block}{Solution: Use atomic member functions}
    \begin{itemize}
      \item The member functions of \cppinline{std::atomic} are thread safe
      \item \cppinline{fetch_add} and \cppinline{operator+=}: atomic \{ load; add; store \}
      \item But don't confuse ``\cppinline{a += 2}'' and ``\cppinline{a = a + 2}''
    \end{itemize}
  \end{block}
  \begin{exampleblock}{}
    \begin{cppcode*}{}
      std::atomic<int> a{0};
      std::thread t1([&]{ a.fetch_add(2); });
      std::thread t2([&]{ a += 2; });
    \end{cppcode*}
  \end{exampleblock}
  \begin{block}{Sequence diagram}
    \begin{tikzpicture}
      \begin{umlseqdiag}
        \umlobject[x=0]{Thread 1}
        \umlobject[x=3, fill=blue!20]{atomic}
        \umlobject[x=6]{Thread 2}
        \begin{umlcall}[op=fetch\_add]{Thread 1}{atomic}
        \end{umlcall}
        \begin{umlcall}[op=\texttt{+=}, dt=9]{Thread 2}{atomic}
        \end{umlcall}
      \end{umlseqdiag}
      \draw[-triangle 60](9,0) -- (9,-2) node[right, pos=0.5]{time};
    \end{tikzpicture}
  \end{block}
\end{frame}

\begin{frame}[fragile]
  \frametitlecpp[20]{Atomic references}
  \begin{block}{\texttt{std::atomic\_ref} template}
    \begin{itemize}
      \item Wraps a \cppinline{T&} and makes access to it atomic
      \item Like \cppinline{std::atomic<T>}, but does not contain the \cppinline{T}
    \end{itemize}
  \end{block}
  \begin{exampleblock}{}
    \begin{cppcode*}{gobble=2}
      int a{0};
      std::thread t1([&]{ std::atomic_ref<int> r{a}; r++;});
      std::thread t2([&]{ std::atomic_ref{a}++; });
      t1.join(); t2.join();
      a += 2; // non-atomic (fine, threads joined before)
      assert( a == 4 ); // Guaranteed to succeed
    \end{cppcode*}
  \end{exampleblock}
  \begin{alertblock}{Don't mix concurrent atomic and non-atomic access}
    \begin{cppcode*}{gobble=2}
      std::thread t3([&]{ std::atomic_ref{a}++; });
      a += 2; // data race
      t3.join();
    \end{cppcode*}
  \end{alertblock}
\end{frame}

\begin{frame}[fragile]
  \frametitle{Atomic types in \cpp}
  \begin{exercise}{Atomics}
    \begin{itemize}
      \item Go to \texttt{exercises/atomic}
      \item You'll find a program with the same race condition as in \texttt{race}
      \item Fix it using \cppinline{std::atomic}
    \end{itemize}
  \end{exercise}
\end{frame}
