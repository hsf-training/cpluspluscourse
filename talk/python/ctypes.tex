\subsection[ctypes]{The ctypes module}

\begin{frame}[fragile]
  \frametitle{The ctypes python module}
  \begin{block}{From the documentation}
    \begin{itemize}
    \item provides C compatible data types
    \item allows calling functions in DLLs or shared libraries
    \item can be used to wrap these libraries in pure Python
    \end{itemize}
  \end{block}
\end{frame}

\begin{frame}[fragile]
  \frametitle{ctypes: usage example}
  \begin{block}{\cpp code : mandel.hpp}
    \begin{cppcode*}{}
      int mandel(Complex const & a);
    \end{cppcode*}
  \end{block}
  \begin{block}{``C'' code : mandel\_cwrapper.hpp}
    \begin{cppcode*}{}
      extern "C" {
        int mandel(float r, float i) {
          return mandel(Complex(r, i));
        };
      }
    \end{cppcode*}
  \end{block}
  \begin{exampleblock}{calling the mandel library}
    \begin{minted}[gobble=4]{python}
      from ctypes import *
      libmandel = CDLL('libmandelc.so')
      v = libmandel.mandel(c_float(0.3), c_float(1.2))
    \end{minted}
  \end{exampleblock}
\end{frame}

\begin{frame}
  \frametitle{Marrying \cpp and python}
  \begin{exercise}{\cpp and python}
    \begin{itemize}
    \item go to \texttt{exercises/python}
    \item look at the original python code mandel.py
    \item time it (`time python3 mandel.py`)
    \item look at the code in mandel.hpp/cpp
    \item look at the python module mandel\_module.cpp
    \item compile and modify mandel.py to use it
    \item see the gain in time
    \item look at the C wrapper in mandel\_cwrapper.cpp
    \item modify mandel.py to use libmandelc directly with ctypes
    \end{itemize}
  \end{exercise}
  \tiny Note : you may have to add '.' to LD\_LIBRARY\_PATH and PYTHONPATH
\end{frame}
