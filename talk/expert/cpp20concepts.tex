\subsection[concepts]{Concepts}

\begin{frame}[fragile]
  \frametitlecpp[20]{Requirements and concepts}
  \begin{block}{Motivation}
    \begin{itemize}
    \item Generic programming is made of variable, function and class templates which can be instantiated with different types.
    \item It is frequent to instantiate them with \textbf{unsuited types}, and the resulting compilation errors are cryptic.
    \item As a last resort, authors provide \textbf{documentation}, and practice tricky \textbf{template meta-programming}.
    \item C++20 brings \textbf{simpler ways to define constraints} on template parameters.
    \end{itemize}
  \end{block}
\end{frame}

\begin{frame}[fragile]
  \frametitlecpp[17]{The world before concepts}
  \begin{block}{\cpp17 work around : SFINAE}
    \begin{itemize}
    \item Unsuited arguments can be avoided by inserting fake template arguments,
    leading to a substitution failure
    \end{itemize}
  \end{block}
  \begin{exampleblockGB}{Practical code}{https://godbolt.org/z/e4T9q1jfh}{SFINAE failure}
    \scriptsize
    \begin{cppcode*}{}
    template
    <typename T,
     typename = std::enable_if_t<std::is_floating_point_v<T>>>
    bool equal( T e1, T e2 ) {
      return std::abs(e1-e2)<std::numeric_limits<T>::epsilon();
    }
    ... equal(10,5+5) ...
    \end{cppcode*}
    \pause
    \tiny
    \begin{Verbatim}[commandchars=\|\[\]]
|fvtextcolor[blue][<source>:11:12:] |fvtextcolor[red][error:] |fvtextcolor[blue][no matching function for call to 'equal(int, int)']
   11 |   if (|fvtextcolor[red][equal(10,5+5)]) { std::cout << "FAILURE\n"; }
|fvtextcolor[blue][<source>:7:6:] |fvtextcolor[blue!50!green][note:] |fvtextcolor[blue][candidate: 'template<class T, class> bool equal(T, T)']
    7 | bool |fvtextcolor[blue!50!green][equal]( T e1, T e2 )
|fvtextcolor[blue][<source>:7:6:] |fvtextcolor[blue!50!green][note:]   |fvtextcolor[blue][template argument deduction/substitution failed:]
In file included from <source>:1:
.../type_traits: In substitution of 'template<bool _Cond, class _Tp>
using enable_if_t = typename std::enable_if::type [with bool _Cond = false; _Tp = void]':
|fvtextcolor[blue][<source>:6:14:   required from here]
.../type_traits:2514:11: |fvtextcolor[red][error:] no type named 'type' in 'struct std::enable_if<false, void>'
 2514 |     using |fvtextcolor[red][enable_if_t] = typename enable_if<_Cond, _Tp>::type;
    \end{Verbatim}
  \end{exampleblockGB}
\end{frame}

\begin{frame}[fragile]
  \frametitlecpp[20]{Basic requirements}
  \begin{block}{A new keyword}
    \begin{itemize}
      \item The keyword \mintinline{cpp}{requires} lets us define various constraints.
    \end{itemize}
  \end{block}
  \begin{exampleblockGB}{With concepts}{https://godbolt.org/z/dWvM9Pvee}{concept failure}
    \scriptsize
    \begin{cppcode*}{}
    template<typename T>
    requires std::is_floating_point_v<T>
    bool equal( T e1, T e2 ) {
      return std::abs(e1-e2)<std::numeric_limits<T>::epsilon();
    }
    ... equal(10,5+5) ...
    \end{cppcode*}
    \pause
    \tiny
    \begin{Verbatim}[commandchars=\|\[\]]
|fvtextcolor[blue][<source>:11:12:] |fvtextcolor[red][error:] |fvtextcolor[blue][no matching function for call to 'equal(int, int)']
   11 |   if (|fvtextcolor[red][equal(10,5+5)]) { std::cout << "FAILURE\n"; }
|fvtextcolor[blue][<source>:7:6:] |fvtextcolor[blue!50!green][note:] |fvtextcolor[blue][candidate: 'template<class T, class> bool equal(T, T)']
    7 | bool |fvtextcolor[blue!50!green][equal]( T e1, T e2 )
|fvtextcolor[blue][<source>:7:6:] |fvtextcolor[blue!50!green][note:]   |fvtextcolor[blue][template argument deduction/substitution failed:]
|fvtextcolor[blue][<source>:7:6:] |fvtextcolor[blue!50!green][note:] |fvtextcolor[blue][constraints not satisfied]
<source>: In substitution of 'template<class T> ... bool equal(T, T) [with T = int]':
|fvtextcolor[blue][<source>:11:12:   required from here]
|fvtextcolor[blue][<source>:7:6:   required by the constraints of 'template<class T> ... bool equal(T, T)']
|fvtextcolor[blue][<source>:6:15:] |fvtextcolor[blue!50!green][note:] |fvtextcolor[blue][the expression 'is_floating_point_v<T> [with T = int]' evaluated to 'false']
    6 | requires |fvtextcolor[blue!50!green][std::is_floating_point_v<T>]
    \end{Verbatim}
  \end{exampleblockGB}
\end{frame}

\begin{frame}[fragile]
  \frametitlecpp[20]{Requirements and overloads}
  \begin{exampleblock}{Example of several competing templates}
    \small
    \begin{cppcode*}{}
    template<typename T>
    bool equal( T e1, T e2 ) { return (e1==e2); }

    template< typename T>
    requires std::is_floating_point_v<T>
    bool equal( T e1, T e2 )
    {return std::abs(e1-e2)<std::numeric_limits<T>::epsilon();}
    \end{cppcode*}
  \end{exampleblock}
  \begin{block}{Requirements affect overload resolution}
    \begin{itemize}
      \item Overload resolution considers the second function as a better match when the requirements are fulfilled.
    \end{itemize}
  \end{block}
\end{frame}

\begin{frame}[fragile]
    \frametitlecpp[20]{Concepts}
    \begin{block}{Definition}
      \begin{itemize}
      \item a \textbf{concept} gives a name to a given set of requirements
      \item useful when the same requirements are reused frequently
      \end{itemize}
    \end{block}
    \begin{exampleblock}{A new keyword: \texttt{concept}}
      \small
      \begin{cppcode*}{gobble=2}
      template< typename T>
      concept MyFloatingPoint =
        std::is_floating_point_v<T> &&
        std::numeric_limits<T>::epsilon()>0;

      template<typename T>
      requires MyFloatingPoint<T>
      bool equal( T e1, T e2 )
      {return std::abs(e1-e2)<std::numeric_limits<T>::epsilon();}
      \end{cppcode*}
    \end{exampleblock}
\end{frame}

\begin{frame}[fragile]
    \frametitlecpp[20]{Some usages of concepts}
    \begin{block}{Constrained template parameters}
      \begin{itemize}
        \item a concept can replace \mintinline{cpp}{typename} in a template parameter lists
      \end{itemize}
    \end{block}
    \begin{exampleblock}{}
      \small
      \begin{cppcode*}{gobble=2}
      template<MyFloatingPoint T>
      bool equal( T e1, T e2 ) {
        return std::abs(e1-e2)<std::numeric_limits<T>::epsilon();
      }
      \end{cppcode*}
    \end{exampleblock}
    \begin{block}{Constrained variables}
      \begin{itemize}
        \item a concept can be used together with \mintinline{cpp}{auto} to impose requirements on the type of a variable
      \end{itemize}
    \end{block}
    \begin{exampleblock}{Example code}
      \begin{cppcode*}{firstnumber=4,gobble=2}
      MyFloatingPoint auto f = 3.14f;
      MyFloatingPoint auto d = 3.14;
      MyFloatingPoint auto i = 3; // compile error
      \end{cppcode*}
    \end{exampleblock}
\end{frame}


\begin{frame}[fragile]
    \frametitlecpp[20]{Some usages of concepts}
    \begin{block}{Abbreviated function templates}
      \begin{itemize}
        \item function parameters can be constrained as well
        \begin{itemize}
          \item similar to variables
        \end{itemize}
        \item the template head becomes obsolete
        \item the function is still a template though!
        \item unconstrained \mintinline{cpp}{auto} parameters are allowed as well
      \end{itemize}
    \end{block}
    \begin{exampleblock}{}
      \small
      \begin{cppcode*}{gobble=2}
      // no template <...> here!
      bool equal( MyFloatingPoint auto e1,
                  MyFloatingPoint auto e2 ) {
        return std::abs(e1-e2) <
               std::numeric_limits<decltype(e1)>::epsilon();
      }

      // unconstrained abbreviated function template:
      void equal(auto e1, auto e2) { ... }
      \end{cppcode*}
    \end{exampleblock}
\end{frame}

\begin{frame}[fragile]
    \frametitlecpp[20]{Standard concepts}
    \begin{block}{Never reinvent the wheel}
      \begin{itemize}
        \item Writing a bug-proof concept is an expert task
        \item Prefer the ones provided by the standard library
      \end{itemize}
    \end{block}
    \begin{exampleblock}{E.g.: the floating point concept}
      \small
      \begin{cppcode*}{gobble=2}
      #include <concepts>
      bool equal( std::floating_point auto e1,
                  std::floating_point auto e2 ) {
        return std::abs(e1-e2) <
               std::numeric_limits<decltype(e1)>::epsilon();
      }
      \end{cppcode*}
    \end{exampleblock}
\end{frame}

\begin{frame}[fragile]
    \frametitlecpp[20]{Checking concepts}
    \begin{block}{Concepts as Boolean operators}
      \begin{itemize}
        \item Concepts can be used wherever a \mintinline{c}{bool} is expected
        \item they can appear in \mintinline{cpp}{if constexpr} conditions
          \begin{itemize}
          \item since they are evaluated at compile-time
          \end{itemize}
      \end{itemize}
    \end{block}
    \begin{exampleblock}{Using concepts with \texttt{if constexpr}}
      \small
      \begin{cppcode*}{gobble=2}
      template<typename T>
      bool equal( T e1, T e2 )
      {
        if constexpr (std::floating_point<T>) {
          return std::abs(e1-e2)
                   < std::numeric_limits<T>::epsilon();
        } else {
          return (e1==e2);
        }
      }
      \end{cppcode*}
    \end{exampleblock}
\end{frame}

\begin{frame}[fragile]
    \frametitlecpp[20]{Advanced requirements overview}
    \begin{block}{\texttt{requires} as an expression}
        \mintinline{cpp}{requires} can express more than basic requirements. It can
        \begin{itemize}
            \item include other basic requirements
            \item list expressions that must be valid
            \item check the return type of some expressions
        \end{itemize}
    \end{block}
    \begin{exampleblock}{Practically}
      \small
      \begin{cppcode*}{gobble=2}
      template<typename T>
      concept StreamableAndComparableNumber =
      requires( T v1, T v2, std::ostream os ) {
        requires std::integral<T> || std::floating_point<T>;
        os<<v1<<v2;
        { equal(v1,v2) } -> std::convertible_to<bool>;
      };
      \end{cppcode*}
    \end{exampleblock}
    \begin{block}{}
      Remember : use standard concepts first
    \end{block}
\end{frame}

\begin{frame}
    \frametitlecpp[20]{Requirements and concepts}
    \begin{block}{Summary}
      \begin{itemize}
      \item A template can now \emph{require} properties of its parameters
      \item Compiler error messages clearly state which argument does not fulfill which requirement
      \item A set of requirements can be gathered in a \emph{concept}
      \item Overload resolution takes requirements into account
      \item The standard library provides many ready-to-use concepts
      \item Writing a new good concept is an expert task
      \end{itemize}
    \end{block}
\end{frame}

\begin{frame}
  \frametitlecpp[20]{Exercise time}
  \begin{exercise}{Concepts}
    \begin{itemize}
    \item go to code/concepts
    \item you will use concepts to optimize some loop on iterators
    \item follow the instructions in the source code
    \end{itemize}
  \end{exercise}
\end{frame}
