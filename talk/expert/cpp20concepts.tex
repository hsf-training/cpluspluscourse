\subsection[tmpl]{C++20 Requirements and Concepts}

\begin{frame}[fragile]
  \frametitlecpp[20]{C++20 Requirements and Concepts}
  \begin{block}{Motivation}
    Generic programming is made of functions and class templates
    which can be instantiated with different types. It is frequent
    to instantiate them with {\em unsuited types}, and the resulting
    compilation errors are generally very long and hardly understandable.
    As a last resort, the template authors are providing {\em documentation}
    about the relevant parameters, and practice some tricky
    {\em template meta-programmation}. C++20 finally brings simpler ways to
    define constraints on template parameters ! Among different proposals,
    the ISO committee has validated the flavor known as {\em Concepts Lite}.
  \end{block}
\end{frame}

\begin{frame}[fragile]
  \frametitlecpp[20]{C++20 Requirements and Concepts}
  \begin{block}{Requirements and concepts in a nutshell}
    \begin{itemize}
    \item A template can define requirements on some of its type parameters.
    \item Compiler error messages clearly state which parameter value does not fulfill which expected requirement.
    \item A typical set of requirements can be gathered in a reusable concept.
    \item Overload resolution takes those requirements and concepts into account.
    \item The standard library now provides many concepts, easy to use.
    \item Writing a new perfect concept stays an expert topic.
    \end{itemize}
  \end{block}
\end{frame}

\begin{frame}[fragile]
  \frametitlecpp[20]{C++20 Requirements and Concepts}
  \begin{block}{Old-style SFINAE example}
    The old way to forbid some type arguments is the definition of
    fake template arguments, which lead to a substitution failure when
    the type does not fulfill the condition.
  \end{block}
  \begin{exampleblock}{Original C++17 code}
    \begin{cppcode*}{}
    template< typename T, std::enable_if_t<!std::is_floating_point_v<T>> * = nullptr>
    bool equal( T e1, T e2 )
    { return (e1==e2) ; }

    template< typename T, std::enable_if_t<std::is_floating_point_v<T>> * = nullptr>
    bool equal( T e1, T e2 )
    { return abs(e1-e2)<std::numeric_limits<T>::epsilon() ; }
    \end{cppcode*}
  \end{exampleblock}
\end{frame}

\begin{frame}[fragile]
    \frametitlecpp[20]{C++20 Requirements and Concepts}
    \begin{block}{Basic requirements}
      The new keyword {\it requires} let us define many kind of requirements
      on the template parameters, with a syntax a lot more natural than previously.
    \end{block}
    \begin{exampleblock}{With {\it requires}}
      \begin{cppcode*}{}
      template< typename T>
      bool equal( T e1, T e2 )
      { return (e1==e2) ; }

      template< typename T>
      requires
        (std::is_floating_point_v<T>) &&
        (std::numeric_limits<T>::epsilon()>0)
      bool equal( T e1, T e2 )
      { return abs(e1-e2)<std::numeric_limits<T>::epsilon() ; }
      \end{cppcode*}
    \end{exampleblock}
  \end{frame}

  \begin{frame}[fragile]
    \frametitlecpp[20]{C++20 Requirements and Concepts}
    \begin{block}{Concepts}
      When a given set of requirements may be reused often, one should gather them in a concept.
    \end{block}
    \begin{exampleblock}{With {\it concept}}
      \begin{cppcode*}{}
      template< typename T>
      concept MyFloatingPoint =
        (std::is_floating_point_v<T>) &&
        (std::numeric_limits<T>::epsilon()>0) ;

      template<typename T>
      requires MyFloatingPoint<T>
      bool equal( T e1, T e2 )
      { return abs(e1-e2)<std::numeric_limits<T>::epsilon() ; }
      \end{cppcode*}
    \end{exampleblock}
  \end{frame}

  \begin{frame}[fragile]
    \frametitlecpp[20]{C++20 Requirements and Concepts}
    \begin{block}{Within the template head}
        Actually, when relevant, the concept can be directly used instead of {it typename} within the template head.
    \end{block}
    \begin{exampleblock}{Instead of {\it typename}}
      \begin{cppcode*}{}
      template< typename T>
      concept MyFloatingPoint =
        (std::is_floating_point_v<T>) &&
        (std::numeric_limits<T>::epsilon()>0) ;

      template<MyFloatingPoint T>
      bool equal( T e1, T e2 )
      { return abs(e1-e2)<std::numeric_limits<T>::epsilon() ; }
      \end{cppcode*}
    \end{exampleblock}
  \end{frame}

  \begin{frame}[fragile]
    \frametitlecpp[20]{C++20 Requirements and Concepts}
    \begin{block}{With abbreviated function templates}
        The concepts can be used together with {\it auto} in the abbreviated function templates.
    \end{block}
    \begin{exampleblock}{With {\it auto}}
      \begin{cppcode*}{}
      template< typename T>
      concept MyFloatingPoint =
        (std::is_floating_point_v<T>) &&
        (std::numeric_limits<T>::epsilon()>0) ;

      bool equal( MyFloatingPoint auto e1, MyFloatingPoint auto e2 )
      { return abs(e1-e2)<std::numeric_limits<T>::epsilon() ; }
      \end{cppcode*}
    \end{exampleblock}
  \end{frame}

  \begin{frame}[fragile]
    \frametitlecpp[20]{C++20 Requirements and Concepts}
    \begin{block}{Standard concepts}
        Writing a bug-proof concept is actually is really expert task.
        Whenever you can, use the ones provided by the standard library.
        Not surprinsgly, there is one for floating point numbers.
    \end{block}
    \begin{exampleblock}{With {\it std::}}
      \begin{cppcode*}{}
      #include <concepts>

      bool equal( std::floating_point auto e1, std::floating_point auto e2 )
      { return abs(e1-e2)<std::numeric_limits<T>::epsilon() ; }
      \end{cppcode*}
    \end{exampleblock}
  \end{frame}

  \begin{frame}[fragile]
    \frametitlecpp[20]{C++20 Requirements and Concepts}
    \begin{block}{Within a function body}
    Concepts are usable wherever a boolean is expected, including {\it if constexpr}, because they are evaluated at compile-time.
    \end{block}
    \begin{exampleblock}{With {\it if constexpr}}
      \begin{cppcode*}{}
      template< typename T>
      bool equal( T e1, T e2 )
      {
        if constexpr (std::floating_point<T>)
        {
          std::cout<<"(floating) " ;
          return abs(e1-e2)<std::numeric_limits<T>::epsilon() ;
        }
        else
        {
          std::cout<<"(default) " ;
          return (e1==e2) ;
        }
      }
      \end{cppcode*}
    \end{exampleblock}
  \end{frame}

  \begin{frame}[fragile]
    \frametitlecpp[20]{C++20 Requirements and Concepts}
    \begin{block}{Advanced requirements}
        More than the basic requirements seen before, one may use a **requires-expression**, able to:
        \begin{itemize}
            \item include basic requirements as before,
            \item list some expressions that must be valid,
            \item check the return type of some expressions.
        \end{itemize}
    \end{block}
    \begin{exampleblock}{Requires-expression}
      \begin{cppcode*}{}
      template< typename T>
      concept StreamableAndComparableNumber = requires( T v1, T v2 )
       {
        requires std::integral<T> || std::floating_point<T> ;
        std::cout<<v1<<v2 ;
        { equal(v1,v2) } -> std::convertible_to<bool> ;
       } ;
      \end{cppcode*}
    \end{exampleblock}
  \end{frame}
