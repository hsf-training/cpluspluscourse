\subsection[sfinae]{SFINAE}

%https://jguegant.github.io/blogs/tech/sfinae-introduction.html
\begin{frame}[fragile]
  \frametitlecpp[11]{Substitution Failure Is Not An Error (SFINAE)}
  \begin{block}{The main idea}
    \begin{itemize}
    \item substitution replaces template parameters with the provided types or values
    \item if it leads to an invalid code, do not fail but try other overloads
    \end{itemize}
  \end{block}
  \begin{exampleblock}{Example}
    \begin{cppcode*}{}
      template <typename T>
      void f(typename T::type arg) { ... }
      void f(int a) { ... }

      f(1); // Calls void f(int)
    \end{cppcode*}
  \end{exampleblock}
  Note : SFINAE will be largely superseded by concepts in \cpp20
\end{frame}

\begin{frame}[fragile]
  \frametitlecpp[11]{decltype}
  \begin{block}{The main idea}
    \begin{itemize}
    \item gives the type of the expression it will evaluate
    \item at compile time
    \end{itemize}
  \end{block}
  \begin{exampleblock}{Example}
    \begin{cppcode*}{}
      struct A { double x; };
      A a;
      decltype(a.x) y;       // double
      decltype((a.x)) z = y; // double& (lvalue)

      template<typename T, typename U>
      auto add(T t, U u) -> decltype(t + u);
      // return type depends on template parameters
    \end{cppcode*}
  \end{exampleblock}
\end{frame}

\begin{frame}[fragile]
  \frametitlecpp[11]{declval}
  \begin{block}{The main idea}
    \begin{itemize}
    \item gives you a ``fake reference'' to an object at compile time
    \item useful for types that cannot be easily constructed
    \end{itemize}
  \end{block}
  \begin{exampleblock}{Example}
    \begin{cppcode*}{}
      struct Default {
        int foo() const { return 1; }
      };
      struct NonDefault {
        NonDefault(int i) { }
        int foo() const { return 1; }
      };
      decltype(Default().foo()) n1 = 1;     // int
      decltype(NonDefault().foo()) n2 = n1; // error
      decltype(std::declval<NonDefault>().foo()) n2 = n1;
    \end{cppcode*}
  \end{exampleblock}
\end{frame}

\begin{frame}[fragile]
  \frametitlecpp[11]{true\_type and false\_type}
  \begin{block}{The main idea}
    \begin{itemize}
    \item encapsulate a constexpr boolean ``true'' and ``false''
    \item can be inherited
    \item constexpr
    \end{itemize}
  \end{block}
  \begin{exampleblock}{Example}
    \begin{cppcode*}{}
      struct testStruct : std::true_type { };
      constexpr bool testVar = testStruct();
      bool test = testStruct::value; // true
    \end{cppcode*}
  \end{exampleblock}
\end{frame}

\begin{frame}[fragile]
  \frametitlecpp[11]{Using SFINAE for introspection}
  \begin{block}{The main idea}
    \begin{itemize}
    \item use a template specialization
      \begin{itemize}
      \item that may or may not create valid code
      \end{itemize}
    \item use SFINAE to choose between them
    \item inherit from true/false\_type
    \end{itemize}
  \end{block}
  \begin{exampleblock}{Example}
    \small
    \begin{cppcode*}{}
      template <typename T, typename = void>
      struct hasFoo : std::false_type {};
      template <typename T>
      struct hasFoo<T, decltype(std::declval<T>().foo())>
        : std::true_type {};
      struct A{}; struct B{void foo();};
      static_assert(!hasFoo<A>::value, "A has no Foo()");
      static_assert(hasFoo<B>::value, "B has Foo()");
    \end{cppcode*}
  \end{exampleblock}
\end{frame}

\begin{frame}[fragile]
  \frametitlecpp[11]{Not so easy actually...}
  \begin{exampleblock}{Example}
    \small
    \begin{cppcode*}{}
      template <typename T, typename = void>
      struct hasFoo : std::false_type {};
      template <typename T>
      struct hasFoo<T, decltype(std::declval<T>().foo())>
        : std::true_type {};

      struct A{};
      struct B{void foo();};
      struct C{int foo();};

      static_assert(!hasFoo<A>::value, "A has no Foo()");
      static_assert(hasFoo<B>::value, "B has Foo()");
      static_assert(!hasFoo<C>::value, "C has Foo()");
      static_assert(hasFoo<C,int>::value, "C has Foo()");
    \end{cppcode*}
  \end{exampleblock}
\end{frame}

\begin{frame}[fragile]
  \frametitlecpp[17]{Using \texttt{void\_t}}
  \begin{block}{Concept}
    \begin{cppcode*}{gobble=2}
      #include <type_traits>
      template <typename... >
      using void_t = void;
    \end{cppcode*}
    \begin{itemize}
    \item Maps a sequence of given types to void
    \item Introduced in \cpp17 though trivial to copy to \cpp11
    \item Can thus be used in specialization to check the validity of an expression
    \end{itemize}
  \end{block}
\end{frame}

\begin{frame}[fragile]
  \frametitlecpp[17]{Previous example using \texttt{void\_t}}
    \begin{exampleblock}{Example}
      \begin{cppcode*}{}
      template <typename T, typename = void>
      struct hasFoo : std::false_type {};

      template <typename T>
      struct hasFoo<T,
         std::void_t<decltype(std::declval<T>().foo())>>
      : std::true_type {};

      struct A{}; struct B{ void foo(); };
      struct C{ int foo(); };

      static_assert(!hasFoo<A>::value,"Unexpected Foo()");
      static_assert(hasFoo<B>::value, "expected Foo()");
      static_assert(hasFoo<C>::value, "expected Foo()");
      \end{cppcode*}
    \end{exampleblock}
\end{frame}

\begin{frame}[fragile]
  \frametitle{SFINAE and the STL \hfill \cpp11/\cpp14/\cpp17}
  \begin{block}{enable\_if / enable\_if\_t}
    \begin{cppcode*}{gobble=2}
      template<bool B, typename T = void> struct enable_if {};
      template<typename T>
      struct enable_if<true, T> { using type = T; };
      template< bool B, typename T = void >
      using enable_if_t = typename enable_if<B,T>::type;
    \end{cppcode*}
    \begin{itemize}
    \item If B is true, has a alias \texttt{type} to type T
    \item otherwise, has no \texttt{type} alias
    \end{itemize}
  \end{block}
  \begin{block}{is\_*$<T>$/is\_*\_v$<T>$ (float/signed/object/final/abstract/...)}
    \begin{itemize}
    \item Checks whether T is ...
    \item At compile time
    \item Has member \texttt{value}, as boolean telling whether it was
    \end{itemize}
  \end{block}
\end{frame}

\begin{frame}[fragile]
  \begin{exampleblock}{Gaudi usage example}
    \begin{cppcode*}{}
      constexpr struct deref_t {
        template
          <typename In,
           typename = typename std::enable_if_t
                      <!std::is_pointer_v<In>>>
        In& operator()( In& in ) const { return in; }

        template <typename In>
        In& operator()( In* in ) const {
          assert(in!=nullptr); return *in;
        }
      } deref {};
    \end{cppcode*}
  \end{exampleblock}

\end{frame}


\begin{frame}[fragile]
  \frametitlecpp[11]{Back to variadic templated class}
  \begin{block}{The tuple get method}
    \begin{cppcode*}{}
      template <size_t k, typename TUPLE>
      struct elem_type_holder;

      template <typename T, typename... Ts>
      struct elem_type_holder<0, tuple<T, Ts...>> {
        using type = T;
      };

      template <size_t k, typename T, typename... Ts>
      struct elem_type_holder<k, tuple<T, Ts...>> {
        using type = typename elem_type_holder
           <k - 1, tuple<Ts...>>::type;
      };
    \end{cppcode*}
  \end{block}
\end{frame}

\begin{frame}[fragile]
  \frametitlecpp[11]{Back to variadic templated class}
  \begin{block}{The tuple get method}
    \begin{cppcode*}{}
      template <size_t k, typename... Ts>
      typename std::enable_if_t<k == 0,
        typename elem_type_holder
          <0, tuple<Ts...>>::type&>
      get(tuple<Ts...>& t) {
        return t.m_tail;
      }
      template <size_t k, typename T, typename... Ts>
      typename std::enable_if_t<k != 0,
        typename elem_type_holder
           <k-1, tuple<Ts...>>::type&>
      get(tuple<T, Ts...>& t) {
        tuple<Ts...>& base = t;
        return get<k - 1>(base);
      }
    \end{cppcode*}
  \end{block}
\end{frame}
