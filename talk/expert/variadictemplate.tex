\subsection[tmpl]{Variadic templates}

%http://eli.thegreenplace.net/2014/variadic-templates-in-c/

\begin{frame}[fragile]
  \frametitlecpp[11]{Basic variadic template}
  \begin{block}{The idea}
    \begin{itemize}
    \item a parameter accepting arbitrarily many arguments (i.e.\ pack)
    \item template parameter pack for e.g.\ types, function parameter packs for values, and expansions, details on \href{https://en.cppreference.com/w/cpp/language/parameter_pack}{cppreference}
    \end{itemize}
  \end{block}
  \begin{exampleblock}{Recursive example \cppinsightLink{https://cppinsights.io/s/1c9fa05f}}
    \begin{cppcode*}{}
      template<typename T>
      T sum(T v) { return v; }

      template<typename T,
               typename... Args>     // temp. param. pack
      T sum(T first, Args... args) { // func. param. pack
        return first + sum(args...); // pack expansion
      }
      int s = sum(1, 2, 3, 8, 7);
    \end{cppcode*}
  \end{exampleblock}
\end{frame}

\begin{frame}
  \frametitlecpp[11]{A couple of remarks}
  \begin{block}{About performance}
    \begin{itemize}
    \item do not be afraid of recursion
    \item everything is at compile time!
    \item unlike C-style variadic functions \\
          e.g.\ \mintinline{cpp}{printf(const char* fmt, ...);}
    \end{itemize}
  \end{block}
  \begin{block}{Why it is better than variadic functions}
    \begin{itemize}
    \item it's more performant
    \item type safety is included
    \item it applies to everything, including user-defined types
    \end{itemize}
  \end{block}
\end{frame}

\begin{frame}[fragile]
  \frametitlecpp[11]{Parameter packs}
  \begin{block}{Parameter packs}
    \begin{itemize}
      \item only a few operations are supported
      \begin{itemize}
        \item query the number of elements in a pack with \mintinline{cpp}{sizeof...}
        \item expand it with \mintinline{cpp}{x...}
      \end{itemize}
      \item can be empty
      \item can hold different types
    \end{itemize}
  \end{block}
  \begin{exampleblock}{}
    \begin{cppcode*}{}
      template<typename... Args>
      void f(Args... args) {
        constexpr std::size_t N = sizeof...(Args);
        std::tuple<Args...> t{args...};
      }
      f(0, 1, 2);       // Args = [int, int, int]
      f();              // Args = empty
      f(0, true, 3.14); // Args = [int, bool, double]
    \end{cppcode*}
  \end{exampleblock}
\end{frame}

\begin{frame}[fragile]
  \frametitlecpp[17]{Fold expressions}
  \begin{block}{The idea}
    \begin{itemize}
      \item reduces a parameter pack over a binary operator
      \item details on \href{https://en.cppreference.com/w/cpp/language/fold}{cppreference}
    \end{itemize}
  \end{block}
  \begin{exampleblock}{Example}
    \begin{cppcode*}{}
      template<typename... Args>
      auto sum1(Args... args) {
        return (args + ...);     // unary fold over +
      }                          // parentheses mandatory
      template<typename... Args>
      auto sum2(Args... args) {
        return (args + ... + 0); // binary fold over +
      }                          // parentheses mandatory
      int sum = sum1(); // error
      int sum = sum2(); // ok
    \end{cppcode*}
  \end{exampleblock}
\end{frame}

\begin{frame}[fragile]
  \frametitlecpp[17]{Fold expressions}
  \begin{exampleblock}{Example: call a function on all arguments - \cppinsightLink{https://cppinsights.io/s/23e7d474}}
    \begin{cppcode*}{}
      template<typename T>
      void print(const T& t) {
        std::cout << t;
      }
      template<typename... Ts>
      void printAll(const Ts&... ts) {
         (print(ts),...); // fold over comma operator
      }
    \end{cppcode*}
  \end{exampleblock}
  \begin{alertblock}{\cpp11 workaround (don't use anymore)}
    \begin{cppcode*}{}
      template<typename... Ts>
      void printAll(const Ts&... ts) {
         int dummy[]{(print(ts),0)...};
      }
    \end{cppcode*}
  \end{alertblock}
\end{frame}

\begin{frame}[fragile]
  \frametitlecpp[11]{Variadic class template}
  \begin{exampleblockGB}{The tuple example, simplified}{https://godbolt.org/z/nxjja4bob}{\texttt{tuple} code}
    \begin{cppcode*}{}
      template <typename... Ts>
      struct tuple {};

      template <typename T, typename... Ts>
      struct tuple<T, Ts...> : tuple<Ts...> {
        tuple(T head, Ts... tail) :
          tuple<Ts...>(tail...), m_head(head) {}
        T m_head;
      };

      tuple<double, uint64_t, const char*>
        t1(12.2, 42, "big");
    \end{cppcode*}
  \end{exampleblockGB}
\end{frame}

\begin{frame}[fragile]
  \frametitlecpp[14]{\texttt{std::integer\_sequence}}
  \begin{block}{Packs of values}
    \begin{itemize}
      \item We can also have non-type template parameter packs
      \item Useful to pass lists of compile-time constants around
      \item The standard library includes a few helpers
    \end{itemize}
  \end{block}
  \begin{block}{}
    \begin{cppcode*}{}
      template<typename T, T... Is>
      struct integer_sequence { ... };

      template<size_t... Is>
      using index_sequence =
        integer_sequence<std::size_t, Is...>;

      template<size_t N>
      using make_index_sequence =
        index_sequence</*0, 1, ..., N-1*/>;
    \end{cppcode*}
  \end{block}
\end{frame}

\begin{frame}[fragile]
  \frametitlecpp[14]{\texttt{std::integer\_sequence}}
  \begin{exampleblock}{Example - make\_from\_tuple with helper}
    \begin{cppcode*}{}
      template<typename T,
        typename... Args, std::size_t... Is>
      T helper(std::tuple<Args...> args,
        std::index_sequence<Is...>) {
        return T(std::get<Is>(args)...);
      }
      template<typename T, typename... Args>
      T make_from_tuple(std::tuple<Args...> args) {
         return helper<T>(args,
           std::make_index_sequence<sizeof...(Args)>{});
      }

      struct S { S(int, float, bool) { ... } };
      std::tuple t{42, 3.14, false};
      S s = make_from_tuple<S>(t);
    \end{cppcode*}
  \end{exampleblock}
\end{frame}

\begin{frame}[fragile]
  \frametitlecpp[20]{\texttt{std::integer\_sequence}}
  \begin{exampleblock}{Example - make\_from\_tuple with lambda}
    \begin{cppcode*}{}
      template<typename T, typename... Args>
      T make_from_tuple(std::tuple<Args...> args) {
         return [&]<std::size_t... Is>(
             std::index_sequence<Is...>){
           return T(std::get<Is>(args)...);
         }(std::make_index_sequence<sizeof...(Args)>{});
      }

      struct S { S(int, float, bool) { ... } };
      std::tuple t{42, 3.14, false};
      S s = make_from_tuple<S>(t);
    \end{cppcode*}
  \end{exampleblock}
\end{frame}

\begin{frame}[fragile]
  \frametitlecpp[11]{Variadic templates}
  \begin{exerciseWithShortcut}{Variadic templates}{Variadic tpl}
    \begin{itemize}
    \item go to code/variadic
    \item you will extend the \mintinline{cpp}{tuple} from the last slides
    \item follow the instructions in the source code
    \end{itemize}
  \end{exerciseWithShortcut}
\end{frame}
