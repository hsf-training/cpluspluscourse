\subsection[tmpl]{Variadic templates}

%http://eli.thegreenplace.net/2014/variadic-templates-in-c/

\begin{frame}[fragile]
  \frametitlecpp[11]{Basic variadic template}
  \begin{block}{The idea}
    \begin{itemize}
    \item a template parameter accepting arbitrary many arguments
    \item template parameter pack for e.g.\ types, function parameter packs for values, and expansions, details on \href{https://en.cppreference.com/w/cpp/language/parameter_pack}{cppreference}
    \end{itemize}
  \end{block}
  \begin{exampleblock}{Recursive example}
    \begin{cppcode*}{}
      template<typename T>
      T sum(T v) { return v; }

      template<typename T,
               typename... Args>     // temp. param. pack
      T sum(T first, Args... args) { // func. param. pack
        return first + sum(args...); // pack expansion
      }
      long sum = sum(1, 2, 3, 8, 7);
    \end{cppcode*}
  \end{exampleblock}
\end{frame}

\begin{frame}
  \frametitlecpp[11]{A couple of remarks}
  \begin{block}{About performance}
    \begin{itemize}
    \item do not be afraid of recursion
    \item everything is at compile time!
    \item unlike C-style variadic functions \\
          e.g.\ \mintinline{cpp}{printf(const char* fmt, ...);}
    \end{itemize}
  \end{block}
  \begin{block}{Why is it better than variadic functions}
    \begin{itemize}
    \item it's more performant
    \item type safety is included
    \item it applies to everything, including objects
    \end{itemize}
  \end{block}
\end{frame}

\begin{frame}[fragile]
  \frametitlecpp[17]{Fold expressions}
  \begin{block}{The idea}
    \begin{itemize}
      \item reduces a parameter pack over a binary operator
      \item details on \href{https://en.cppreference.com/w/cpp/language/fold}{cppreference}
    \end{itemize}
  \end{block}
  \begin{exampleblock}{Example}
    \begin{cppcode*}{}
      template<typename... Args>
      T sum1(Args... args) {
        return (args + ...);     // unary fold over +
      }
      template<typename... Args>
      T sum2(Args... args) {
        return (args + ... + 0); // binary fold over +
      }
      long sum = sum1(); // error
      long sum = sum2(); // ok
    \end{cppcode*}
  \end{exampleblock}
\end{frame}

\begin{frame}[fragile]
  \frametitlecpp[11]{Variadic class template}
  \begin{block}{The tuple example, simplified}
    \begin{cppcode*}{}
      template <typename... Ts>
      struct tuple {};

      template <typename T, typename... Ts>
      struct tuple<T, Ts...> : tuple<Ts...> {
        tuple(T head, Ts... tail) :
          tuple<Ts...>(tail...), m_head(head) {}
        T m_head;
      };

      tuple<double, uint64_t, const char*>
        t1(12.2, 42, "big");
    \end{cppcode*}
  \end{block}
\end{frame}

\begin{frame}[fragile]
  \frametitlecpp[14]{\texttt{std::integer\_sequence}}
  \begin{block}{Packs of values}
    \begin{itemize}
      \item We can also have non-type template parameter packs
      \item Useful to pass lists of compile-time constants around
      \item The standard library includes a few helpers
    \end{itemize}
  \end{block}
  \begin{block}{}
    \begin{cppcode*}{}
      template<typename T, T... Is>
      struct integer_sequence { ... };

      template<size_t... Is>
      using index_sequence =
        integer_sequence<std::size_t, Is...>;

      template<size_t N>
      using make_index_sequence =
        index_sequence</*0, 1, ..., N-1*/>;
    \end{cppcode*}
  \end{block}
\end{frame}

\begin{frame}[fragile]
  \frametitlecpp[14]{\texttt{std::integer\_sequence}}
  \begin{exampleblock}{Example - make\_from\_tuple}
    \begin{cppcode*}{}
      template<typename T,
        typename... Args, std::size_t... Is>
      T helper(std::tuple<Args...> args,
        std::index_sequence<Is...>) {
        return T(std::get<Is>(args)...);
      }
      template<typename T, typename... Args>
      T make_from_tuple(std::tuple<Args...> args) {
         return helper<T>(args,
           std::make_index_sequence<sizeof...(Args)>{});
      }

      struct S { S(int, float, bool) { ... } };
      std::tuple t{42, 3.14, false};
      S s = make_from_tuple<S>(t);
    \end{cppcode*}
  \end{exampleblock}
\end{frame}
