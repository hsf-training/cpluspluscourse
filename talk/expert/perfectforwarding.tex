\subsection[forward]{Perfect forwarding}

%http://eli.thegreenplace.net/2014/perfect-forwarding-and-universal-references-in-c/
\begin{frame}[fragile]
  \frametitlecpp[11]{The problem}
  Trying to write a generic wrapper function
  \begin{cppcode*}{}
    template <typename T>
    void wrapper(T arg) {
      func(arg);
    }
  \end{cppcode*}
  Example usage :
  \begin{itemize}
  \item emplace\_back
  \item make\_unique
  \end{itemize}
\end{frame}

\begin{frame}[fragile]
  \frametitlecpp[11]{Why is it not so simple?}
  \begin{cppcode*}{}
    template <typename T>
    void wrapper(T arg) {
      func(arg);
    }
  \end{cppcode*}
  \begin{alertblock}{What about references ?}
    what if func takes references to avoid copies ?\\
    wrapper would force a copy and we fail to use references
  \end{alertblock}
\end{frame}

\begin{frame}[fragile]
  \frametitlecpp[11]{Second try, second failure ?}
  \begin{cppcode*}{}
    template <typename T>
    void wrapper(T &arg) {
      func(arg);
    }
    wrapper(42);
    // invalid initialization of
    // non-const reference from
    // an rvalue
  \end{cppcode*}
  \begin{alertblock}{}
    const ref won't help : you may want to pass something non const\\
    and rvalue are not yet supported...
  \end{alertblock}
\end{frame}

\begin{frame}[fragile]
  \frametitlecpp[11]{The solution : cover all cases}
  \begin{cppcode*}{}
    template <typename T>
    void wrapper(T& arg) { func(arg); }

    template <typename T>
    void wrapper(const T& arg) { func(arg); }

    template <typename T>
    void wrapper(T&& arg) { func(arg); }
  \end{cppcode*}
\end{frame}

\begin{frame}[fragile]
  \frametitlecpp[11]{The new problem : scaling to n arguments}
  \begin{cppcode*}{}
    template <typename T1, typename T2>
    void wrapper(T1& arg1, T2& arg2)
    { func(arg1, arg2); }

    template <typename T1, typename T2>
    void wrapper(const T1& arg1, T2& arg2)
    { func(arg1, arg2); }

    template <typename T1, typename T2>
    void wrapper(T1& arg1, const T2& arg2)
    { func(arg1, arg2); }
    ...
  \end{cppcode*}
  \begin{alertblock}{Exploding complexity}
    3$^{n}$ complexity\\
    you do not want to try n = 5...
  \end{alertblock}
\end{frame}

\begin{frame}[fragile]
  \frametitlecpp[11]{Reference collapsing in \cpp98}
  \begin{block}{Reference to references}
    They are forbidden in \cpp\\
    But still may happen
    \begin{cppcode*}{}
      template <typename T>
      void foo(T t) {
        T& k = t;
        ...
      }
      int ii = 4;
      foo<int&>(ii);
    \end{cppcode*}
  \end{block}
  \begin{exampleblock}{Practically}
    all compilers were collapsing the 2 references
  \end{exampleblock}
\end{frame}

\begin{frame}
  \frametitlecpp[11]{Reference collapsing in \cpp11}
  \begin{block}{rvalues have been added}
    \begin{itemize}
    \item what about int\&\& \& ?
    \item and int \&\& \&\& ?
    \end{itemize}
  \end{block}
  \begin{exampleblock}{\cpp11 standardization}
    The rule is simple : \& always wins\\
    \&\& \&, \& \&\&, \& \& $\rightarrow$ \&\\
    \&\& \&\& $\rightarrow$ \&\&
  \end{exampleblock}
\end{frame}

\begin{frame}[fragile]
  \frametitlecpp[11]{rvalue in type-deducing context}
  \begin{cppcode*}{}
    template <typename T>
    void func(T&& t) {}
  \end{cppcode*}
  Next to a template parameter, \&\& is not an rvalue, but a ``universal reference''\\
  T\&\& actual type depends on the arguments passed to func
  \begin{itemize}
  \item if an lvalue of type U is given, T is deduced to U\&
  \item otherwise, collapse references normally
  \end{itemize}
  \begin{cppcode*}{firstnumber=3}
    func(4);        // rvalue -> T&& is int&&
    double d = 3.14;
    func(d);        // lvalue -> T&& is double&
    float f() {...}
    func(f());      // rvalue -> T&& is float&&
    int foo(int i) {
      func(i);      // lvalue -> T&& is int&
    }
  \end{cppcode*}
\end{frame}

\begin{frame}[fragile]
  \frametitlecpp[11]{std::remove\_reference}
  Some template trickery removing reference from a type
  \begin{cppcode*}{}
    template< typename T >
    struct remove_reference
    {using type = T;};

    template< typename T >
    struct remove_reference<T&>
    {using type = T;};

    template< typename T >
    struct remove_reference<T&&>
    {using type = T;};
  \end{cppcode*}
  If {\ttfamily T} is a reference type, {\ttfamily remove\_reference<T>::type} is the type referred to by T,
  otherwise it is T.
\end{frame}

\begin{frame}[fragile]
  \frametitlecpp[11]{std::forward}
  Another template trickery keeping references and mapping non reference types to rvalue references
  \begin{cppcode*}{}
    template<typename T>
    T&& forward(typename std::remove_reference<T>::type& t)
      noexcept {
      return static_cast<T&&>(t);
    }
  \end{cppcode*}
  \begin{block}{How it works}
    \begin{itemize}
    \item if T is int, it returns int \&\&
    \item if T is int\&, it returns int\& \&\& ie int\&
    \item if T is int\&\&, it returns int\&\& \&\& ie int\&\&
    \end{itemize}
  \end{block}
\end{frame}

\begin{frame}[fragile]
  \frametitlecpp[11]{Perfect forwarding}
  Putting it all together
  \begin{cppcode*}{}
    template <typename... T>
    void wrapper(T&&... args) {
      func(std::forward<T>(args)...);
    }
  \end{cppcode*}
  \begin{block}{How it works}
    \begin{itemize}
    \item if we pass an rvalue to wrapper (U\&\&)
      \begin{itemize}
      \item arg will be of type U\&\&
      \item func will be called with a U\&\&
      \end{itemize}
    \item if we pass an lvalue to wrapper (U\&)
      \begin{itemize}
      \item arg will be of type U\&
      \item func will be called with a U\&
      \end{itemize}
    \item if we pass a plain value (U)
      \begin{itemize}
      \item arg will be of type U\&\& (no copy in wrapper)
      \item func will be called with a U\&\&
      \item but func takes a U, so copy happens there, as expected
      \end{itemize}
    \end{itemize}
  \end{block}
\end{frame}

\begin{frame}[fragile]
  \frametitlecpp[11]{Real life example}
  \begin{cppcode*}{}
    template<typename T, typename... Args>
    unique_ptr<T> make_unique(Args&&... args) {
      return unique_ptr<T>
        (new T(std::forward<Args>(args)...));
    }
  \end{cppcode*}
\end{frame}
