\subsection[modules]{Modules}

\begin{frame}[fragile]
  \frametitlecpp[20]{Modules}
  \begin{block}{Motivation}
    \begin{itemize}
      \item Modules allow for sharing declarations and definitions across translation units
      \item Header files do the same, but modules have benefits:
      \begin{itemize}
        \item Better isolation. No cross-talk between multiple pieces of included code, or between included code and your code.
        \item Better control over public interface of your library. Avoid leaking library dependencies into user code.
        \item Faster compilation (\href{https://youtu.be/DJTEUFRslbI?si=ZRvH1wx0sVJVriLh&t=3259}{import fmt; \textgreater1500x faster})
      \end{itemize}
    \end{itemize}
  \end{block}
  \begin{alertblock}{Textual includes}
    \begin{cppcode}
      #include "math.h"
    \end{cppcode}
  \end{alertblock}
  \begin{exampleblock}{Module importation}
    \begin{cppcode}
      import math;
    \end{cppcode}
  \end{exampleblock}
\end{frame}

\begin{frame}[fragile]
  \frametitlecpp[20]{Writing modules}
  \begin{exampleblock}{ratio.cpp}
    \begin{cppcode*}{}
      export module math;   // module declaration
      import <string>;
      namespace utils { // ns. and module names orthogonal
        export struct Ratio { int num, denom; };
        export std::string toString(const Ratio& r) { ... }
        void normalize(Ratio& r) { ... } // not exported
      }
    \end{cppcode*}
  \end{exampleblock}
  \begin{exampleblock}{main.cpp}
    \begin{cppcode*}{}
      import <iostream>;
      import math;
      int main() {
        auto r = utils::Ratio{1, 3};
        std::cout << utils::toString(r);
      }
    \end{cppcode*}
  \end{exampleblock}
\end{frame}

\begin{frame}[fragile]
  \frametitlecpp[20]{Writing modules}
  \begin{block}{Exporting declarations}
    \begin{itemize}
      \item By default, everything inside a module is internal
      \item Declarations become visible when declared with \cppinline{export} in the primary module interface
      \item You can only export entities at namespace scope (including the global scope)
      \item You cannot export entities with internal linkage (declared static or inside anonymous namespace)
      \item You cannot export aliases to non-exported types
    \end{itemize}
  \end{block}
\end{frame}

\begin{frame}[fragile]
  \frametitlecpp[20]{Writing modules}
  \begin{exampleblock}{What to export}
    \begin{cppcode*}{}
      export module math;

      export void f();           // function
      export float pi = 3.14f;   // variable
      export namespace utils {   // namespace
        enum Status { Yes, No }; // enum (exported)
      }
      namespace utils {
        export struct Ratio { ... }; // class
        export using R = Ratio;      // alias
        float e = 2.71f;             // (not exported)
      }
      export { // content of export block
        void h();
      }
    \end{cppcode*}
  \end{exampleblock}
\end{frame}

\begin{frame}[fragile]
  \frametitlecpp[20]{Writing modules}
  \begin{alertblock}{What not to export}
    \begin{cppcode*}{}
      export module math;

      namespace {
        export void f();              // error
        export float pi = 3.14f;      // error
        export struct Ratio { ... };  // error
      }

      export static void f();         // error
      export static float pi = 3.14f; // error

      class C { ... };
      export using D = C;             // error
    \end{cppcode*}
  \end{alertblock}
\end{frame}

\begin{frame}[fragile]
  \frametitlecpp[20]{Visibility and reachability}
  \begin{exampleblock}{ratio.cpp}
    \begin{cppcode*}{}
      export module math;
      struct Ratio { int num, denom; };
      export Ratio golden() { ... } // golden is visible
    \end{cppcode*}
  \end{exampleblock}
  \begin{exampleblockGB}{main.cpp}{https://godbolt.org/z/n95rdxPdos}{Modules example}
    \begin{cppcode*}{}
      import math;
      import <iostream>;
      int main() {
        Ratio r = golden(); // error: Ratio not visible
        auto r = golden();  // OK: Ratio is reachable
        std::cout << r.num; // OK: members of reachable
                            // types are visible
        using R = decltype(r);
        R r2{1.f, 3.f};     // OK
      }
    \end{cppcode*}
  \end{exampleblockGB}
\end{frame}

\begin{frame}[fragile]
  \frametitlecpp[20]{Module structure}
  \begin{exampleblock}{}
    \small
    \begin{cppcode*}{}
      module;                  // global module fragment (opt.)
      #include <cassert>       //   only preprocessor
      // ...                   //   statements allowed

      export module math;      // module preamble starts with
                               //   exported module name.
      import <string>;         //   only imports allowed
      // ...                   //

      export struct S { ... }; // module purview started
      export S f();            //   at first non-import
      std::string h() { ... }

      module :private          // private module fragment (opt.)
      S f() { ... }            //   changes here trigger
                               //   no recompilation of BMI
    \end{cppcode*}
  \end{exampleblock}
\end{frame}

\begin{frame}[fragile]
  \frametitlecpp[20]{Module structure}
  \begin{alertblock}{Pitfall}
    \begin{cppcode*}{}
      export module math;
      #include <header.h>
      import <string>;
      // ...
    \end{cppcode*}
  \end{alertblock}
  \begin{block}{}
    \begin{itemize}
    \item Headers included after \cppinline{export module} add to the module's preamble and purview
    \item This is usually not what you want
    \end{itemize}
  \end{block}
  \begin{exampleblock}{Put includes into global module fragment}
    \begin{cppcode*}{}
      module;
      #include <vector>
      export module math;
      import <string>;
      // ...
    \end{cppcode*}
  \end{exampleblock}
\end{frame}

\begin{frame}[fragile,shrink=5]
  \frametitlecpp[20]{Module structure}
  \begin{block}{}
    Like with header files, we can split interface and implementation:
  \end{block}
  \begin{exampleblock}{Interface}
    \begin{cppcode*}{}
      export module math; // Module interface unit declaration

      export struct S { ... };
      export S f();
    \end{cppcode*}
  \end{exampleblock}
  \begin{exampleblock}{Implementation}
    \begin{cppcode*}{}
      module;
      #include <cassert>
      module math;        // Module implementation unit
      import <string>;

      S f() { ... }
      std::string b() { ... }
    \end{cppcode*}
  \end{exampleblock}
\end{frame}

\begin{frame}[fragile,shrink=5]
  \frametitlecpp[20]{Submodules and re-exporting}
  \begin{block}{}
    Modules can (re-)export other modules using \cppinline{export import}:
  \end{block}
  \vspace{-5mm}
  \begin{columns}[t]
    \begin{column}{.45\textwidth}
      \begin{exampleblock}{Module one}
        \begin{cppcode*}{}
          export module one;

          export struct S { ... };
        \end{cppcode*}
      \end{exampleblock}
    \end{column}
    \begin{column}{.45\textwidth}
      \begin{exampleblock}{Module two}
        \begin{cppcode*}{}
          export module two;
          import <string>;

          export std::string b()
            { ... }
        \end{cppcode*}
      \end{exampleblock}
    \end{column}
  \end{columns}
  \begin{columns}[t]
    \begin{column}{.45\textwidth}
      \begin{exampleblock}{Module three}
        \begin{cppcode*}{}
          export module three;
          import one;
          export import two;

          export S f() { ... }
        \end{cppcode*}
      \end{exampleblock}
    \end{column}
    \begin{column}{.45\textwidth}
      \begin{exampleblock}{Module four}
        \begin{cppcode*}{}
          export module four;
          import three;
          // b and f visible
          // S and std::string only
          //   reachable
        \end{cppcode*}
      \end{exampleblock}
    \end{column}
  \end{columns}
\end{frame}

\begin{frame}[fragile,shrink=5]
  \frametitlecpp[20]{Module partitions}
  \begin{block}{}
    We can split a module's \emph{interface} into multiple files:
  \end{block}
  \vspace{-5mm}
  \begin{columns}[t]
    \begin{column}{.4\textwidth}
      \begin{exampleblock}{Primary module interface}
        \begin{cppcode*}{}
          export module math;
          export import :structs;

          export S f();
        \end{cppcode*}
      \end{exampleblock}
    \end{column}
    \begin{column}{.5\textwidth}
      \begin{exampleblock}{Module interface partition}
        \begin{cppcode*}{}

          export module math:structs;

          export struct S { ... };
        \end{cppcode*}
      \end{exampleblock}
    \end{column}
  \end{columns}
  \begin{exampleblock}{Implementation}
    \begin{cppcode*}{}
      module math;
      import <string>;

      S f() { ... }
      std::string b() { ... }
    \end{cppcode*}
  \end{exampleblock}
\end{frame}

\begin{frame}[fragile,shrink=5]
  \frametitlecpp[20]{Module implementation units}
  \begin{block}{}
    We can split a module's \emph{implementation} into multiple files:
  \end{block}
  \begin{exampleblock}{Interface}
    \begin{cppcode*}{}
      export module math;
      import :foo; // no export
      import :bar; // no export
      import <string>;

      export struct S { ... };
      export S f();
      std::string b();
    \end{cppcode*}
  \end{exampleblock}
  \vspace{-5mm}
  \begin{columns}[t]
    \begin{column}{.45\textwidth}
      \begin{exampleblock}{Implementation unit foo}
        \begin{cppcode*}{}
          module math:foo;

          S f() { ... }
        \end{cppcode*}
      \end{exampleblock}
    \end{column}
    \begin{column}{.45\textwidth}
      \begin{exampleblock}{Implementation unit bar}
        \begin{cppcode*}{}
          module math:bar;

          std::string b() { ... }
        \end{cppcode*}
      \end{exampleblock}
    \end{column}
  \end{columns}
\end{frame}

\begin{frame}[fragile]
  \frametitlecpp[23]{Standard library modules}
  % see: Minimal module support for the standard library http://wg21.link/P2412
  \begin{block}{Standard library modules}
    \begin{itemize}
      \item Use \cppinline|import std;| for the entire C++ standard library.
      \begin{itemize}
      \item E.g.\ \cppinline{memcpy(a, b, n)}, \cppinline{sqrt(val)}, \cppinline{uint32_t}
      \item Note that you need to prefix with \cppinline{std::}
      \item Currently, support is not great (see later)
      \end{itemize}
      \item \cppinline|import std.compat;| for the C and C++
      \item No macros are exported. For these, you still need to include the corresponding headers.
      \begin{cppcode*}{}
        import std;        // everything C++
        #include <cassert> // for assert()
        #include <climits> // for e.g. CHAR_BIT
      \end{cppcode*}
      \item Definition of further, smaller modules of the standard library is in progress
    \end{itemize}
  \end{block}
\end{frame}

\begin{frame}[fragile]
    \frametitlecpp[20]{Header units}
    \begin{block}{Purpose}
      \begin{itemize}
      \item For easier transition from header-based projects to modules
      \item Allows to import any header file as if it was a module
      \item Can be mixed with regular header inclusion
      \end{itemize}
    \end{block}
    \begin{cppcode}
      #include <set>
      import <vector>;

      #include "header.hpp"
      import "otherheader.hpp";
    \end{cppcode}
\end{frame}

\begin{frame}[fragile]
    \frametitlecpp[20]{Header units}
    \begin{block}{Effect}
      \begin{itemize}
      \item All declarations in the imported header will be exported
      \item An imported header is not affected by the preprocessor state.
            Thus, you cannot ``configure'' the header by defining a macro before importing.
            This allows the header to be precompiled.
      \item An imported header may still provide preprocessor macros
      \end{itemize}
    \end{block}
    \begin{cppcode}
      #define MY_MACRO 1
      #include "header.hpp" // may be affected by MY_MACRO
      import "otherheader.hpp"; // ignores MY_MACRO
    \end{cppcode}
\end{frame}

\begin{frame}[fragile]
    \frametitlecpp[20]{Automatic include to import translation}
    \begin{block}{Importable headers}
      \begin{itemize}
      \item A header file which can either be included or imported is called ``importable''
      \begin{itemize}
        \item I.e.\ the header does not require setup of preprocessor state before inclusion
      \end{itemize}
      \item All \cpp standard library headers are importable
      \item C wrapper headers (\cppinline{<c*>}) are not
      \end{itemize}
    \end{block}
    \begin{block}{Include translation}
      \begin{itemize}
        \item Allows the compiler to automatically turn includes into imports
        \item Controlled via compiler flags and the build system
        \item Basically a standard replacement for precompiled headers
      \end{itemize}
    \end{block}
\end{frame}

\begin{frame}[fragile]
  \frametitle{Build system}
  \begin{columns}
    \begin{column}{.35\textwidth}
      \begin{block}{h.hpp}
        \begin{cppcode*}{linenos=false}
          #include <vector>
          ...
        \end{cppcode*}
      \end{block}
      \begin{block}{a.cpp}
        \begin{cppcode*}{linenos=false}
          #include "h.hpp"
          ...
        \end{cppcode*}
      \end{block}
      \begin{block}{b.cpp}
        \begin{cppcode*}{linenos=false}
          #include <vector>
          #include "h.hpp"
          ...
        \end{cppcode*}
      \end{block}
      \begin{block}{c.cpp}
        \begin{cppcode*}{linenos=false}
          #include <vector>
          ...
        \end{cppcode*}
      \end{block}
    \end{column}
    \begin{column}{.6\textwidth}
      \begin{block}{Traditional workflow}
        \center
        \begin{tikzpicture} [
            -Triangle,
            very thick
          ]
          \draw [-Latex](5,6) -- (5,0)  node [midway,right] {t};
          \path[thick]
            node (vec)  at(3,6)   [draw, rectangle] {\textless{}vector\textgreater{}}
            node (hpp)  at(1,4.5) [draw, rectangle] {h.hpp}
            node (acpp) at(0,3)   [draw, rectangle] {a.cpp}
            node (bcpp) at(2,3)   [draw, rectangle] {b.cpp}
            node (ccpp) at(4,3)   [draw, rectangle] {c.cpp}
            node (ao)   at(0,1.5) [draw, rectangle] {a.o}
            node (bo)   at(2,1.5) [draw, rectangle] {b.o}
            node (co)   at(4,1.5) [draw, rectangle] {c.o};

          \node[below= 0.5 cm of bo,fill=white,rounded rectangle] (link) {Linker};
          \node[below= 0.2 cm of link,rectangle,draw] (bin) {binary};

          \draw[very thick,dotted] (vec)  -- (hpp)  node [midway,fill=white,rounded rectangle] {\scriptsize \#};
          \draw[very thick,dotted] (vec)  -- (bcpp) node [midway,fill=white,rounded rectangle] {\scriptsize \#};
          \draw[very thick,dotted] (vec)  -- (ccpp) node [midway,fill=white,rounded rectangle] {\scriptsize \#};
          \draw[very thick,dotted] (hpp)  -- (acpp) node [midway,fill=white,rounded rectangle] {\scriptsize \#};
          \draw[very thick,dotted] (hpp)  -- (bcpp) node [midway,fill=white,rounded rectangle] {\scriptsize \#};
          \draw[very thick] (acpp) -- (ao)   node [pos=0.4,fill=white,rounded rectangle] {\scriptsize \texttt{Compiler}};
          \draw[very thick] (bcpp) -- (bo)   node [pos=0.4,fill=white,rounded rectangle] {\scriptsize \texttt{Compiler}};
          \draw[very thick] (ccpp) -- (co)   node [pos=0.4,fill=white,rounded rectangle] {\scriptsize \texttt{Compiler}};
          \draw[very thick] (ao)   -- (link);
          \draw[very thick] (bo)   -- (link);
          \draw[very thick] (co)   -- (link);
          \draw[thick] (link) -- (bin);

        \end{tikzpicture}
      \end{block}
    \end{column}
  \end{columns}
\end{frame}

\begin{frame}[fragile]
  \frametitlecpp[20]{Build system}
  \begin{columns}
    \begin{column}{.35\textwidth}
      \small
      \begin{block}{h.cpp}
        \begin{cppcode*}{linenos=false}
          export module foo;
          import <vector>;
          ...
        \end{cppcode*}
      \end{block}
      \begin{block}{a.cpp}
        \begin{cppcode*}{linenos=false}
          import foo;
          ...
        \end{cppcode*}
      \end{block}
      \begin{block}{b.cpp}
        \begin{cppcode*}{linenos=false}
          import <vector>;
          import foo;
          ...
        \end{cppcode*}
      \end{block}
      \begin{block}{c.cpp}
        \begin{cppcode*}{linenos=false}
          #include <vector>
          ...
        \end{cppcode*}
      \end{block}
    \end{column}
    \begin{column}{.6\textwidth}
      \begin{block}{New workflow}
        \center
        \resizebox{0.8\textwidth}{!}{%
        \begin{tikzpicture}[
              -Triangle,
              very thick
            ]
          \draw [-Latex](7,10) -- (7,0)  node [midway,right] {t};
          \draw[thick] node at(3, 10) [fill=white,rounded rectangle] {scan dependencies};
          \path[thick]
          node (vec)    at(5,9)   [rectangle,draw] {\textless{}vector\textgreater{}}
          node (vecbmi) at(4,7.5) [rectangle,draw] {vector.bmi}
          node (hcpp)   at(2,6)   [rectangle,draw] {h.cpp}
          node (ho)     at(0,4.5) [rectangle,draw] {h.o}
          node (hbmi)   at(2,4.5) [rectangle,draw] {foo.bmi}
          node (acpp)   at(2,3)   [rectangle,draw] {a.cpp}
          node (bcpp)   at(4,3)   [rectangle,draw] {b.cpp}
          node (ccpp)   at(6,6)   [rectangle,draw] {c.cpp}
          node (ao)     at(2,1.5) [rectangle,draw] {a.o}
          node (bo)     at(4,1.5) [rectangle,draw] {b.o}
          node (co)     at(6,4.5) [rectangle,draw] {c.o}
          node (link)   at(3,0.7) [fill=white, rounded rectangle] {\small Linker}
          node (bin)    at(3,0)   [rectangle,draw] {binary};
          \draw[very thick,dotted] (vec) -- (ccpp) node [midway,fill=white,rounded rectangle] {\scriptsize \#};
          \draw[very thick] (vec) -- (vecbmi) node [midway,fill=white,rounded rectangle] {\scriptsize Compiler};
          \draw[very thick,dotted] (vecbmi) -- (hcpp) node [midway,fill=white,rounded rectangle] {\scriptsize imp};
          \draw[very thick] (hcpp) -- (ho) node [pos=0.4] (hoLeft) {};
          \draw[very thick] (hcpp) -- (hbmi) node [pos=0.4] (hoRight) {};
          \path (hoLeft) -- (hoRight) node[midway,fill=white,rounded rectangle] {\scriptsize Compiler};
          \draw[very thick,dotted] (vecbmi) -- (bcpp) node [midway,fill=white,rounded rectangle] {\scriptsize imp};
          \draw[very thick,dotted] (hbmi) -- (acpp) node [pos=0.4,fill=white,rounded rectangle] {\scriptsize imp};
          \draw[very thick,dotted] (hbmi) -- (bcpp) node [pos=0.4,fill=white,rounded rectangle] {\scriptsize imp};
          \draw[very thick] (acpp) -- (ao) node [pos=0.4,fill=white,rounded rectangle] {\scriptsize \texttt{Compiler}};
          \draw[very thick] (bcpp) -- (bo) node [pos=0.4,fill=white,rounded rectangle] {\scriptsize \texttt{Compiler}};
          \draw[very thick] (ccpp) -- (co) node [pos=0.4,fill=white,rounded rectangle] {\scriptsize \texttt{Compiler}};
          \draw[very thick] (ho) to [bend right=40] (link);
          \draw[very thick] (ao) -- (link);
          \draw[very thick] (bo) -- (link);
          \draw[very thick] (co) to [bend left=40] (link);
          \draw[thick] (link) -- (bin);
        \end{tikzpicture}
        }
      \end{block}
    \end{column}
  \end{columns}
\end{frame}

\begin{frame}[fragile]
    \frametitle{Build system}
    \begin{block}{New challenges}
      \begin{itemize}
        \item Resolving a module name to the file name building it
        \item Translation units no longer independent
        \begin{itemize}
          \item Extra tool needs to infer dependency graph before building
          \begin{itemize}
            \item Called a ``dependency scanner''
            \item Needs to be communicated to the build system
            \item Using a new \href{https://wg21.link/P1689}{standard dependency file format}
          \end{itemize}
          \item Parallel and distributed builds need synchronization
        \end{itemize}
        \item Compilation of module translation units produces a binary module interface (BMI) \emph{and} an object file
        \begin{itemize}
          \item Need to manage BMIs between multiple compiler invocations
        \end{itemize}
        \item Tools beside the compiler need to build/read BMIs
        \begin{itemize}
          \item E.g.\ static analysis, language servers for auto completion, etc.
        \end{itemize}
        \item The \cpp standard specifies very little on how this works
        \begin{itemize}
          \item We may experience large implementation divergence
        \end{itemize}
      \end{itemize}
    \end{block}
\end{frame}

\begin{frame}[fragile]
    \frametitlecpp[20]{Build system}
    \begin{block}{Status}
      \begin{itemize}
        \item Build systems:
        \begin{itemize}
          \item Dependency scanner since clang 16: \texttt{clang-scan-deps}
          \item Cmake support in v3.28 (\href{https://cmake.org/cmake/help/latest/manual/cmake-cxxmodules.7.html}{documentation})
          \item MSBuild (Windows) fully handles dependency scanning
          \item \href{https://wg21.link/p1602}{Experimental extensions to GNU make}
        \end{itemize}
        \item \cppinline{g++ a.cpp b.cpp ...} ``just works'', must be one line though
        \item C++23: \cppinline{import std;} supported in MSVC, partially in clang and gcc
          \begin{itemize}
            \item The standard library module currently has to be compiled manually (\href{https://gcc.gnu.org/git/?p=gcc.git;a=commit;h=7db55c0ba1baaf0e323ef7f9ef8c9cda077d40e9}{see gcc commit in 2024},
            \href{https://0xstubs.org/using-the-c23-std-module-with-gcc-15/}{gcc-15 recipe}, \href{https://0xstubs.org/using-the-c23-std-module-with-clang-18/}{clang-18 recipe})
            \item GCC/clang/MSVC also plan support in C++20 % see: https://github.com/microsoft/STL/issues/3945
          \end{itemize}
        \item Header units unsupported (May 2023) % See talk: https://youtu.be/_LGR0U5Opdg?si=yQoCYD2yGFhi_vs6&t=3768
        \begin{itemize}
          \item Standard library header units possible in gcc11 and MSVC (2025)
        \end{itemize}
      \end{itemize}
    \end{block}
\end{frame}

\begin{frame}[fragile]
    \frametitlecpp[20]{Modules}
    \begin{exampleblock}{Guidance for today}
      \begin{itemize}
        \item Start writing importable headers (no macro dependencies)
        \begin{itemize}
          \item Dual use: \cppinline{#include} before \cpp20 and \cppinline{import} with \cpp20
        \end{itemize}
        \item Watch progress on module support in your build system
      \end{itemize}
    \end{exampleblock}
    \begin{exampleblock}{Guidance for tomorrow}
      \begin{itemize}
        \item Start writing modules when your users have \cpp20
        \item Maybe import standard headers when compilers fully support \cpp20
        \item Start using \cppinline{import std;} when available
        \item Future of header units unclear (implementation difficulties)
      \end{itemize}
    \end{exampleblock}
\end{frame}

\begin{frame}[fragile]
    \frametitlecpp[20]{Modules}
    \begin{block}{Resources}
      \small
      \begin{itemize}
      \item \href{https://www.youtube.com/watch?v=nP8QcvPpGeM}{A (Short) Tour of C++ Modules - Daniela Engert - CppCon 2021} (very technical)
      \item Understanding C++ Modules: \href{https://vector-of-bool.github.io/2019/03/10/modules-1.html}{Part1}, \href{https://vector-of-bool.github.io/2019/03/31/modules-2.html}{Part2} and \href{https://vector-of-bool.github.io/2019/10/07/modules-3.html}{Part3}.
      \item \href{https://www.youtube.com/watch?v=DJTEUFRslbI}{So, You Want to Use C++ Modules … Cross-Platform? - Daniela Engert - C++ on Sea 2023}
      \item \href{https://www.youtube.com/watch?v=c563KgO-uf4&t=686s}{import CMake: 2023 State of C++20 modules in CMake - Bill Hoffman - CppNow 2023}
      \end{itemize}
    \end{block}
\end{frame}

\begin{frame}[fragile]
  \frametitlecpp[20]{Modules}
  \begin{exercise}{Modules}
    \begin{itemize}
      \item go to \texttt{exercises/modules}
      \item convert the \texttt{Complex.hpp} header into a module named \texttt{math}
    \end{itemize}
  \end{exercise}
  \begin{exercise}{Header units}
    \begin{itemize}
      \item go to \texttt{exercises/header\_units}
      \item convert all \cppinline{#include}s into header unit \cppinline{import}s
    \end{itemize}
  \end{exercise}
\end{frame}
