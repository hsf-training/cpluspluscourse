\subsection[modules]{Modules}

\begin{frame}[fragile]
  \frametitlecpp[20]{Modules}
  \begin{block}{Motivation}
    \begin{itemize}
      \item Modules allow for sharing declarations and definitions across translation units
      \item Header files do the same, but modules have benefits:
      \begin{itemize}
        \item Better isolation. No cross-talk between multiple pieces of included code, and also not between included code and your code.
        \item Better control over public interface of your library. Avoid leaking library dependencies into user code.
        \item Faster compilation
      \end{itemize}
    \end{itemize}
  \end{block}
  \begin{alertblock}{Textual includes}
    \begin{cppcode}
      #include "math.h"
    \end{cppcode}
  \end{alertblock}
  \begin{exampleblock}{Module importation}
    \begin{cppcode}
      import math;
    \end{cppcode}
  \end{exampleblock}
\end{frame}

\begin{frame}[fragile]
  \frametitlecpp[20]{Writing modules}
  \begin{exampleblock}{ratio.cpp}
    \begin{cppcode*}{gobble=2}
      export module math;
      import <string>;
      namespace utils { // ns. and module names orthogonal
        export struct Ratio { float num, denom; };
        export std::string toString(const Ratio& r) { ... }
        void normalize(Ratio& r) { ... } // not exported
      }
    \end{cppcode*}
  \end{exampleblock}
  \begin{exampleblock}{main.cpp}
    \begin{cppcode*}{gobble=2}
      import <iostream>;
      import math;
      int main() {
        auto r = utils::Ratio{1, 3};
        std::cout << utils::toString(r);
      }
    \end{cppcode*}
  \end{exampleblock}
\end{frame}

\begin{frame}[fragile]
  \frametitlecpp[20]{Writing modules}
  \begin{block}{Exporting declarations}
    \begin{itemize}
      \item By default, everything inside a module is internal
      \item Declarations become visible when declared with \mintinline{cpp}{export} in the primary module interface
      \item You can only export entities at namespace scope (including the global scope)
      \item You cannot export entities with internal linkage (declared static or inside anonymous namespace)
      \item You cannot export aliases to non-exported types
    \end{itemize}
  \end{block}
\end{frame}

\begin{frame}[fragile]
  \frametitlecpp[20]{Writing modules}
  \begin{exampleblock}{What to export}
    \begin{cppcode*}{gobble=2}
      export module math;

      export void f();           // function
      export float pi = 3.14f;   // variable
      export namespace utils {   // namespace
        enum Status { Yes, No }; // enum (exported)
      }
      namespace utils {
        export struct Ratio { ... }; // class
        export using R = Ratio;      // alias
        float e = 2.71f;             // (not exported)
      }
      export { // content of export block
        void h();
      }
    \end{cppcode*}
  \end{exampleblock}
\end{frame}

\begin{frame}[fragile]
  \frametitlecpp[20]{Writing modules}
  \begin{alertblock}{What not to export}
    \begin{cppcode*}{gobble=2}
      export module math;

      namespace {
        export void f();              // error
        export float pi = 3.14f;      // error
        export struct Ratio { ... };  // error
      }

      export static void f();         // error
      export static float pi = 3.14f; // error

      class C { ... };
      export using D = C;             // error
    \end{cppcode*}
  \end{alertblock}
\end{frame}

\begin{frame}[fragile]
  \frametitlecpp[20]{Visibility and reachability}
  \begin{exampleblock}{ratio.cpp}
    \begin{cppcode*}{gobble=2}
      export module math;
      struct Ratio { float num, denom; };
      export Ratio golden() { ... } // golden is visible
    \end{cppcode*}
  \end{exampleblock}
  \begin{exampleblock}{main.cpp}
    \begin{cppcode*}{gobble=2}
      import math;
      import <iostream>;
      int main() {
        Ratio r = golden(); // error (Ratio not visible)
        auto r = golden();  // ok: (Ratio is reachable)
        std::cout << r.num; // ok: members of reachable
                            // class types become visible
        using R = decltype(r);
        R r2{1.f, 3.f};     // ok
      }
    \end{cppcode*}
  \end{exampleblock}
\end{frame}

\begin{frame}[fragile]
  \frametitlecpp[20]{Module structure}
  \begin{exampleblock}{}
    \begin{cppcode*}{gobble=2}
      module;
      #include <cassert>       // global module fragment
      // ...                   //   only preprocessor
                               //   statements allowed

      export module math;      // exported module name

      import <string>;         // module preamble
      // ...                   //   only imports allowed

      export struct S { ... }; // module purview
      export S f() { ... }     //  starts at first
      std::string h() { ... }  //  non-import
    \end{cppcode*}
  \end{exampleblock}
\end{frame}

\begin{frame}[fragile]
  \frametitlecpp[20]{Module structure}
  \begin{alertblock}{Pitfall}
    \begin{cppcode*}{gobble=2}
      export module math;
      #include <header.h>
      import <string>;
      // ...
    \end{cppcode*}
  \end{alertblock}
  \begin{block}{}
    \begin{itemize}
    \item Headers included after \mintinline{cpp}{export module} add to the module's preamble and purview
    \item This is usually not what you want
    \end{itemize}
  \end{block}
  \begin{exampleblock}{Put includes into global module fragment}
    \begin{cppcode*}{gobble=2}
      module;
      #include <vector>
      export module math;
      import <string>;
      // ...
    \end{cppcode*}
  \end{exampleblock}
\end{frame}

\begin{frame}[fragile,shrink=5]
  \frametitlecpp[20]{Module structure}
  \begin{block}{}
    Like with header files, we can split interface and implementation:
  \end{block}
  \begin{exampleblock}{Interface}
    \begin{cppcode*}{gobble=2}
      export module math;

      export struct S { ... };
      export S f();
    \end{cppcode*}
  \end{exampleblock}
  \begin{exampleblock}{Implementation}
    \begin{cppcode*}{gobble=2}
      module;
      #include <cassert>
      module math;
      import <string>;

      S f() { ... }
      std::string b() { ... }
    \end{cppcode*}
  \end{exampleblock}
\end{frame}

\begin{frame}[fragile,shrink=5]
  \frametitlecpp[20]{Module partitions}
  \begin{block}{}
    We can split a module's interface into multiple files:
  \end{block}
  \vspace{-5mm}
  \begin{columns}[t]
    \begin{column}{.4\textwidth}
      \begin{exampleblock}{Primary module interface}
        \begin{cppcode*}{gobble=6}
          export module math;
          export import :structs;

          export S f();
        \end{cppcode*}
      \end{exampleblock}
    \end{column}
    \begin{column}{.5\textwidth}
      \begin{exampleblock}{Module interface partition}
        \begin{cppcode*}{gobble=6}

          export module math:structs;

          export struct S { ... };
        \end{cppcode*}
      \end{exampleblock}
    \end{column}
  \end{columns}
  \begin{exampleblock}{Implementation}
    \begin{cppcode*}{gobble=2}
      module math;
      import <string>;

      S f() { ... }
      std::string b() { ... }
    \end{cppcode*}
  \end{exampleblock}
\end{frame}

\begin{frame}[fragile,shrink=5]
  \frametitlecpp[20]{Module implementation units}
  \begin{block}{}
    We can split a module's implementation into multiple files:
  \end{block}
  \begin{exampleblock}{Interface}
    \begin{cppcode*}{gobble=2}
      export module math;
      import :foo; // no export
      import :bar; // no export
      import <string>;

      export struct S { ... };
      export S f();
      std::string b();
    \end{cppcode*}
  \end{exampleblock}
  \vspace{-5mm}
  \begin{columns}[t]
    \begin{column}{.45\textwidth}
      \begin{exampleblock}{Implementation unit foo}
        \begin{cppcode*}{gobble=6}
          module math:foo;

          S f() { ... }
        \end{cppcode*}
      \end{exampleblock}
    \end{column}
    \begin{column}{.45\textwidth}
      \begin{exampleblock}{Implementation unit bar}
        \begin{cppcode*}{gobble=6}
          module math:bar;

          std::string b() { ... }
        \end{cppcode*}
      \end{exampleblock}
    \end{column}
  \end{columns}
\end{frame}

\begin{frame}[fragile,shrink=5]
  \frametitlecpp[20]{Submodules and re-exporting}
  \begin{block}{}
    Modules can (re-)export other modules:
  \end{block}
  \vspace{-5mm}
  \begin{columns}[t]
    \begin{column}{.45\textwidth}
      \begin{exampleblock}{Module base}
        \begin{cppcode*}{gobble=6}
          export module base;

          export struct S { ... };
        \end{cppcode*}
      \end{exampleblock}
    \end{column}
    \begin{column}{.45\textwidth}
      \begin{exampleblock}{Module bar}
        \begin{cppcode*}{gobble=6}
          export module bar;
          import <string>;

          export std::string b()
            { ... }
        \end{cppcode*}
      \end{exampleblock}
    \end{column}
  \end{columns}
  \begin{columns}[t]
    \begin{column}{.45\textwidth}
      \begin{exampleblock}{Module foo}
        \begin{cppcode*}{gobble=6}
          export module foo;
          import export module base;
          import bar;

          export S f() { ... }
        \end{cppcode*}
      \end{exampleblock}
    \end{column}
    \begin{column}{.45\textwidth}
      \begin{exampleblock}{Module math}
        \begin{cppcode*}{gobble=6}
          export module math;
          export import foo;
        \end{cppcode*}
      \end{exampleblock}
    \end{column}
  \end{columns}
\end{frame}

\begin{frame}[fragile]
  \frametitlecpp[23]{Standard library modules}
  % see: Minimal module support for the standard library http://wg21.link/P2412
  \begin{block}{Standard library modules}
    \begin{itemize}
      \item Use \mintinline{cpp}|import std;| for the entire C++ standard library. Basically everything in namespace \mintinline{cpp}|std|.
      \begin{itemize}
        \item E.g.\ \mintinline{cpp}{memcpy(a, b, n);}, \mintinline{cpp}{sqrt(val);}, \mintinline{cpp}{uint32_t i;} will not compile. You need to prefix \mintinline{cpp}{std::}.
      \end{itemize}
      \item Use \mintinline{cpp}|import std.compat;| for the entire C and C++ standard library
      \item No macros are exported. For these, you still need to include the corresponding headers.
      \begin{cppcode*}{gobble=4}
        import std;        // everything C++
        #include <cassert> // for assert()
        #include <climits> // for e.g. CHAR_BIT
      \end{cppcode*}
      \item Definition of further, smaller modules of the standard library is in progress (e.g.\ fundamentals, containers, networking, etc.)
    \end{itemize}
  \end{block}
\end{frame}

\begin{frame}[fragile]
    \frametitlecpp[20]{Header units}
    \begin{block}{Purpose}
      \begin{itemize}
      \item For easier transition from header-based projects to modules
      \item Allows to import any header file as if it was a module
      \item Can be mixed with regular header inclusion
      \end{itemize}
    \end{block}
    \begin{cppcode}
      #include <set>
      import <vector>;

      #include "header.hpp"
      import "otherheader.hpp";
    \end{cppcode}
\end{frame}

\begin{frame}[fragile]
    \frametitlecpp[20]{Header units}
    \begin{block}{Effect}
      \begin{itemize}
      \item All declarations in the imported header will be exported
      \item An imported header is not affected by the preprocessor state.
            Thus, you cannot ``configure'' the header by defining a macro before importing.
            This allows the header to be precompiled.
      \item An imported header may still provide preprocessor macros
      \end{itemize}
    \end{block}
    \begin{cppcode}
      #define MY_MACRO 1
      #include "header.hpp" // may be affected by MY_MACRO
      import "otherheader.hpp"; // ignores MY_MACRO
    \end{cppcode}
\end{frame}

\begin{frame}[fragile]
    \frametitlecpp[20]{Automatic include to import translation}
    \begin{block}{Importable headers}
      \begin{itemize}
      \item A header file which can successfully be either included or imported is called ``importable''
      \begin{itemize}
        \item I.e.\ the header does not require setup of preprocessor state before inclusion
      \end{itemize}
      \item All \cpp standard library headers are importable
      \item C wrapper headers (\mintinline{cpp}{<c*>}) are not
      \end{itemize}
    \end{block}
    \begin{block}{Include translation}
      \begin{itemize}
        \item Allows the compiler to automatically turn includes into imports
        \item Controlled via compiler flags and the build system
        \item Basically a standard replacement for precompiled headers
      \end{itemize}
    \end{block}
\end{frame}

\begin{frame}[fragile]
  \frametitle{Build system}
  \begin{columns}
    \begin{column}{.35\textwidth}
      \begin{block}{h.hpp}
        \begin{cppcode*}{gobble=6,linenos=false}
          #include <vector>
          ...
        \end{cppcode*}
      \end{block}
      \begin{block}{a.cpp}
        \begin{cppcode*}{gobble=6,linenos=false}
          #include "h.hpp"
          ...
        \end{cppcode*}
      \end{block}
      \begin{block}{b.cpp}
        \begin{cppcode*}{gobble=6,linenos=false}
          #include <vector>
          #include "h.hpp"
          ...
        \end{cppcode*}
      \end{block}
      \begin{block}{c.cpp}
        \begin{cppcode*}{gobble=6,linenos=false}
          #include <vector>
          ...
        \end{cppcode*}
      \end{block}
    \end{column}
    \begin{column}{.6\textwidth}
      \begin{block}{Traditional workflow}
        \center
        \begin{tikzpicture} [
            -Triangle,
            very thick
          ]
          \draw [-Latex](5,6) -- (5,0)  node [midway,right] {t};
          \path[thick]
            node (vec)  at(3,6)   [draw, rectangle] {\textless{}vector\textgreater{}}
            node (hpp)  at(1,4.5) [draw, rectangle] {h.hpp}
            node (acpp) at(0,3)   [draw, rectangle] {a.cpp}
            node (bcpp) at(2,3)   [draw, rectangle] {b.cpp}
            node (ccpp) at(4,3)   [draw, rectangle] {c.cpp}
            node (ao)   at(0,1.5) [draw, rectangle] {a.o}
            node (bo)   at(2,1.5) [draw, rectangle] {b.o}
            node (co)   at(4,1.5) [draw, rectangle] {c.o};

          \node[below= 0.5 cm of bo,fill=white,rounded rectangle] (link) {Linker};
          \node[below= 0.2 cm of link,rectangle,draw] (bin) {binary};

          \draw[very thick,dotted] (vec)  -- (hpp)  node [midway,fill=white,rounded rectangle] {\scriptsize \#};
          \draw[very thick,dotted] (vec)  -- (bcpp) node [midway,fill=white,rounded rectangle] {\scriptsize \#};
          \draw[very thick,dotted] (vec)  -- (ccpp) node [midway,fill=white,rounded rectangle] {\scriptsize \#};
          \draw[very thick,dotted] (hpp)  -- (acpp) node [midway,fill=white,rounded rectangle] {\scriptsize \#};
          \draw[very thick,dotted] (hpp)  -- (bcpp) node [midway,fill=white,rounded rectangle] {\scriptsize \#};
          \draw[very thick] (acpp) -- (ao)   node [pos=0.4,fill=white,rounded rectangle] {\scriptsize \texttt{Compiler}};
          \draw[very thick] (bcpp) -- (bo)   node [pos=0.4,fill=white,rounded rectangle] {\scriptsize \texttt{Compiler}};
          \draw[very thick] (ccpp) -- (co)   node [pos=0.4,fill=white,rounded rectangle] {\scriptsize \texttt{Compiler}};
          \draw[very thick] (ao)   -- (link);
          \draw[very thick] (bo)   -- (link);
          \draw[very thick] (co)   -- (link);
          \draw[thick] (link) -- (bin);

        \end{tikzpicture}
      \end{block}
    \end{column}
  \end{columns}
\end{frame}

\begin{frame}[fragile]
  \frametitlecpp[20]{Build system}
  \begin{columns}
    \begin{column}{.35\textwidth}
      \small
      \begin{block}{h.cpp}
        \begin{cppcode*}{gobble=6,linenos=false}
          export module foo;
          import <vector>;
          ...
        \end{cppcode*}
      \end{block}
      \begin{block}{a.cpp}
        \begin{cppcode*}{gobble=6,linenos=false}
          import foo;
          ...
        \end{cppcode*}
      \end{block}
      \begin{block}{b.cpp}
        \begin{cppcode*}{gobble=6,linenos=false}
          import <vector>;
          import foo;
          ...
        \end{cppcode*}
      \end{block}
      \begin{block}{c.cpp}
        \begin{cppcode*}{gobble=6,linenos=false}
          #include <vector>
          ...
        \end{cppcode*}
      \end{block}
    \end{column}
    \begin{column}{.6\textwidth}
      \begin{block}{New workflow}
        \center
        \resizebox{0.8\textwidth}{!}{%
        \begin{tikzpicture}[
              -Triangle,
              very thick
            ]
          \draw [-Latex](7,10) -- (7,0)  node [midway,right] {t};
          \draw[thick] node at(3, 10) [fill=white,rounded rectangle] {build dependencies};
          \path[thick]
          node (vec)    at(5,9)   [rectangle,draw] {\textless{}vector\textgreater{}}
          node (vecbmi) at(4,7.5) [rectangle,draw] {vector.bmi}
          node (hcpp)   at(2,6)   [rectangle,draw] {h.cpp}
          node (ho)     at(0,4.5) [rectangle,draw] {h.o}
          node (hbmi)   at(2,4.5) [rectangle,draw] {foo.bmi}
          node (acpp)   at(2,3)   [rectangle,draw] {a.cpp}
          node (bcpp)   at(4,3)   [rectangle,draw] {b.cpp}
          node (ccpp)   at(6,6)   [rectangle,draw] {c.cpp}
          node (ao)     at(2,1.5) [rectangle,draw] {a.o}
          node (bo)     at(4,1.5) [rectangle,draw] {b.o}
          node (co)     at(6,4.5) [rectangle,draw] {c.o}
          node (link)   at(3,0.7) [fill=white, rounded rectangle] {\small Linker}
          node (bin)    at(3,0)   [rectangle,draw] {binary};
          \draw[very thick,dotted] (vec) -- (ccpp) node [midway,fill=white,rounded rectangle] {\scriptsize \#};
          \draw[very thick] (vec) -- (vecbmi) node [midway,fill=white,rounded rectangle] {\scriptsize Compiler};
          \draw[very thick,dotted] (vecbmi) -- (hcpp) node [midway,fill=white,rounded rectangle] {\scriptsize imp};
          \draw[very thick] (hcpp) -- (ho) node [pos=0.4] (hoLeft) {};
          \draw[very thick] (hcpp) -- (hbmi) node [pos=0.4] (hoRight) {};
          \path (hoLeft) -- (hoRight) node[midway,fill=white,rounded rectangle] {\scriptsize Compiler};
          \draw[very thick,dotted] (vecbmi) -- (bcpp) node [midway,fill=white,rounded rectangle] {\scriptsize imp};
          \draw[very thick,dotted] (hbmi) -- (acpp) node [pos=0.4,fill=white,rounded rectangle] {\scriptsize imp};
          \draw[very thick,dotted] (hbmi) -- (bcpp) node [pos=0.4,fill=white,rounded rectangle] {\scriptsize imp};
          \draw[very thick] (acpp) -- (ao) node [pos=0.4,fill=white,rounded rectangle] {\scriptsize \texttt{Compiler}};
          \draw[very thick] (bcpp) -- (bo) node [pos=0.4,fill=white,rounded rectangle] {\scriptsize \texttt{Compiler}};
          \draw[very thick] (ccpp) -- (co) node [pos=0.4,fill=white,rounded rectangle] {\scriptsize \texttt{Compiler}};
          \draw[very thick] (ho) to [bend right=40] (link);
          \draw[very thick] (ao) -- (link);
          \draw[very thick] (bo) -- (link);
          \draw[very thick] (co) to [bend left=40] (link);
          \draw[thick] (link) -- (bin);
        \end{tikzpicture}
        }
      \end{block}
    \end{column}
  \end{columns}
\end{frame}

\begin{frame}[fragile]
    \frametitle{Build system}
    \begin{block}{New challenges}
      \begin{itemize}
        \item Resolving a module name to the file name building it
        \item Translation units no longer independent
        \begin{itemize}
          \item Extra tool needs to infer dependency graph before building
          \item This graph needs to be communicated to the build system
          \item We will likely need a \href{https://wg21.link/P1689}{standard dependency file format}
          \item Parallel and distributed builds need synchronization
        \end{itemize}
        \item Compilation of module translation units produces binary module interface (BMI) and object file
        \begin{itemize}
          \item BMIs need to be maintained/shared between multiple compiler instances
        \end{itemize}
        \item Tools beside the compiler need to build/read binary module interface (static analysis, auto completion, etc.)
        \item The C++ standard specifies very little on how this works
        \begin{itemize}
          \item We may experience large implementation divergence
        \end{itemize}
      \end{itemize}
    \end{block}
\end{frame}

\begin{frame}[fragile]
    \frametitlecpp[20]{Build system}
    \begin{block}{Status}
      \begin{itemize}
        \item build2 supports everything g++ does
        \item MSBuild figures out dependencies and build order automatically (Windows/Visual Studio only)
        \item \mintinline{cpp}{g++ a.cpp b.cpp ...} ``just works'', must be one line though
        \item GCC uses a \href{https://wg21.link/P1184}{module mapper} to resolve imports, which specifies a protocol and uses a central server for module maintenance
        \item \href{https://wg21.link/p1602}{Experimental extensions to GNU make} exist to grow dependency graph on the fly while modules are discovered
        \item No support in cmake (3.23) yet
      \end{itemize}
    \end{block}
\end{frame}

\begin{frame}[fragile]
    \frametitlecpp[20]{Building modules (g++)}
    \begin{block}{Case study: g++}
      \begin{itemize}
        \item By default, g++ caches compiled module interfaces (CMI, i.e.\ BMI) in subdirectory \texttt{./gcm.cache}
        \item Each module needs to be built before it can be imported
        \begin{itemize}
          \item {\footnotesize \mintinline{cpp}{g++ -std=c++20 -fmodules-ts -c ratio.cpp -o ratio.o}}
          \item Generates \texttt{ratio.o} and \texttt{./gcm.cache/ratio.gcm} (CMI)
        \end{itemize}
        \item Each header unit needs to be built before it can be imported
        \begin{itemize}
          \item {\footnotesize \mintinline{cpp}{g++ -std=c++20 -fmodules-ts -x c++-system-header vector}}
          \item Generates e.g.\ \texttt{./gcm.cache/usr/include/c++/11/vector.gcm} (CMI)
          \item {\footnotesize \mintinline{cpp}{g++ -std=c++20 -fmodules-ts -x c++-header ratio.h}}
          \item Generates e.g.\ \texttt{./gcm.cache/,/ratio.h.gcm} (CMI)
        \end{itemize}
      \end{itemize}
    \end{block}
\end{frame}

\begin{frame}[fragile]
    \frametitlecpp[20]{Modules}
    \begin{exampleblock}{Guidance for today}
      \begin{itemize}
        \item Start writing importable headers (no macro dependencies)
        \begin{itemize}
          \item This allows dual use: \mintinline{cpp}{#include} before \cpp20 and \mintinline{cpp}{import} with \cpp20
        \end{itemize}
        \item Watch progress on module support in your build system
      \end{itemize}
    \end{exampleblock}
    \begin{exampleblock}{Guidance for tomorrow}
      \begin{itemize}
        \item Start importing headers when \cpp20 is available
        \item Start writing modules when all users of your code have \cpp20
        \item Start using \mintinline{cpp}{import std;} when \cpp23 is available
      \end{itemize}
    \end{exampleblock}
\end{frame}

\begin{frame}[fragile]
    \frametitlecpp[20]{Modules}
    \begin{block}{Resources}
      \begin{itemize}
      \item \href{https://www.youtube.com/watch?v=szHV6RdQdg8}{Practical C++ Modules - Boris Kolpackov - CppCon 2019}
      \item \href{https://www.youtube.com/watch?v=nP8QcvPpGeM}{A (Short) Tour of C++ Modules - Daniela Engert - CppCon 2021} (very technical)
      \item Understanding C++ Modules: \href{https://vector-of-bool.github.io/2019/03/10/modules-1.html}{Part1}, \href{https://vector-of-bool.github.io/2019/03/31/modules-2.html}{Part2} and \href{https://vector-of-bool.github.io/2019/10/07/modules-3.html}{Part3}.
      \end{itemize}
    \end{block}
\end{frame}

\begin{frame}[fragile]
  \frametitlecpp[20]{Modules}
  \begin{exercise}{Modules}
    \begin{itemize}
      \item go to \texttt{code/modules}
      \item convert the \texttt{Complex.hpp} header into a module named \texttt{math}
    \end{itemize}
  \end{exercise}
  \begin{exercise}{Header units}
    \begin{itemize}
      \item go to \texttt{code/header\_units}
      \item convert all \mintinline{cpp}{#include}s into header unit \mintinline{cpp}{import}s
    \end{itemize}
  \end{exercise}
\end{frame}
