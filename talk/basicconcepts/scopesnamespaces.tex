\subsection[NS]{Scopes / namespaces}

\begin{frame}[fragile]
  \frametitlecpp[98]{Scope}
  \begin{block}{Definition}
    Portion of the source code where a given name is valid \\
    Typically :
    \begin{itemize}
    \item simple block of code, within \cppinline{{}}
    \item function, class, namespace
    \item the global scope, i.e.\ translation unit (.cpp file + all includes)
    \end{itemize}
  \end{block}
  \begin{exampleblock}{Example}
    \begin{cppcode*}{}
      { int a;
        { int b;
        } // end of b scope
      } // end of a scope
    \end{cppcode*}
  \end{exampleblock}
\end{frame}

\begin{frame}[fragile]
  \frametitlecpp[98]{Scope and lifetime of variables}
  \begin{block}{Variable life time}
    \begin{itemize}
      \item Variables are (statically) allocated when defined
      \item Variables are freed at the end of a scope
    \end{itemize}
  \end{block}
  \begin{goodpractice}{Initialisation}
    \begin{itemize}
      \item Initialise variables when allocating them!
      \item This prevents bugs reading uninitialised memory
    \end{itemize}
  \end{goodpractice}
  \begin{multicols}{2}
    % the code has to be repeated n times here. Anyone finding a better solution
    % is welcome, but it's not a trivial task, due to the verbatim nature of minted
    \begin{overprint}[\columnwidth]
    \onslide<1>
      \begin{minted}[linenos,highlightlines=1]{cpp}
int a = 1;
{
  int b[4];
  b[0] = a;
}
// Doesn't compile here:
// b[1] = a + 1;
      \end{minted}
    \onslide<2>
      \begin{minted}[linenos,highlightlines=3]{cpp}
int a = 1;
{
  int b[4];
  b[0] = a;
}
// Doesn't compile here:
// b[1] = a + 1;
      \end{minted}
    \onslide<3>
      \begin{minted}[linenos,highlightlines=4]{cpp}
int a = 1;
{
  int b[4];
  b[0] = a;
}
// Doesn't compile here:
// b[1] = a + 1;
      \end{minted}
    \onslide<4>
      \begin{minted}[linenos,highlightlines=7]{cpp}
int a = 1;
{
  int b[4];
  b[0] = a;
}
// Doesn't compile here:
// b[1] = a + 1;
      \end{minted}
    \end{overprint}

    \columnbreak

    \begin{tikzpicture}
      \memorystack[word size=1, block size=4, nb blocks=5, size x = 0.5\columnwidth]
      \onslide<1-> {
        \memorypush{a = 1}
      }
      \onslide<2>{
        \memorypush{b[0] = ?}
      }
      \memorygoto{2}
      \onslide<3>{
        \memorypush{b[0] = 1}
      }
      \memorygoto{3}
      \onslide<2-3>{
        \memorypush{b[1] = ?}
        \memorypush{b[2] = ?}
        \memorypush{b[3] = ?}
      }

      \memorygoto{2}
      \onslide<4>{
        \memorypush{\color{gray} 1}
        \memorypush{\color{gray} ?}
        \memorypush{\color{gray} ?}
        \memorypush{\color{gray} ?}
        }

    \end{tikzpicture}

  \end{multicols}
\end{frame}

\begin{frame}[fragile]
  \frametitlecpp[98]{Namespaces}
  \begin{itemize}
  \item Namespaces allow to segment your code to avoid name clashes
  \item They can be embedded to create hierarchies (separator is '::')
  \end{itemize}
  \begin{multicols}{2}
    \begin{cppcode*}{gobble=2}
      int a;
      namespace n {
        int a;   // no clash
      }
      namespace p {
        int a;   // no clash
        namespace inner {
          int a; // no clash
        }
      }
      void f() {
        n::a = 3;
      }
    \end{cppcode*}
    \columnbreak
    \begin{cppcode*}{gobble=2,firstnumber=14}
      namespace p { // reopen p
        void f() {
          p::a = 6;
          a = 6;  //same as above
          ::a = 1;
          p::inner::a = 8;
          inner::a = 8;
          n::a = 3;
        }
      }
      using namespace p::inner;
      void g() {
        a = -1; // err: ambiguous
      }
  \end{cppcode*}
  \end{multicols}
\end{frame}

\begin{frame}[fragile]
  \frametitlecpp[17]{Nested namespaces}
  Easier way to declare nested namespaces
  \begin{alertblock}{\cpp98}
    \begin{cppcode*}{}
      namespace A {
        namespace B {
          namespace C {
            //...
          }
        }
      }
    \end{cppcode*}
  \end{alertblock}
  \begin{exampleblock}{\cpp17}
    \begin{cppcode*}{}
      namespace A::B::C {
        //...
      }
    \end{cppcode*}
  \end{exampleblock}
\end{frame}

\begin{advanced}
\begin{frame}[fragile]
  \frametitlecpp[98]{Unnamed / anonymous namespaces}
  \begin{exampleblock}{A namespace without a name!}
    \begin{cppcode*}{}
      namespace {
        int localVar;
      }
    \end{cppcode*}
  \end{exampleblock}
  \begin{block}{Purpose}
    \begin{itemize}
    \item groups a number of declarations
    \item visible only in the current translation unit
    \item but not reusable outside
    \item allows much better compiler optimizations and checking
      \begin{itemize}
      \item e.g. unused function warning
      \item context dependent optimizations
      \end{itemize}
    \end{itemize}
  \end{block}
  \begin{alertblock}{Supersedes static}
    \begin{cppcode*}{gobble=2,firstnumber=4}
      static int localVar; // equivalent C code
    \end{cppcode*}
  \end{alertblock}
\end{frame}

\begin{frame}[fragile]
  \frametitlecpp[98]{Using namespace directives}
  \begin{alertblock}{Avoid ``using namespace'' directives}
    \begin{itemize}
      \item Make all members of a namespace visible in current scope
      \item Risk of name clashes or ambiguities
    \end{itemize}
    \begin{cppcode*}{}
      using namespace std;
      cout << "We can print now\n"; // uses std::cout
    \end{cppcode*}
  \end{alertblock}
  \begin{alertblock}{Never use in headers at global scope!}
    \begin{cppcode*}{gobble=2}
      #include "PoorlyWritten.h" // using namespace std;
      struct array { ... };
      array a;  // Error: name clash with std::array
    \end{cppcode*}
  \end{alertblock}
  \begin{block}{What to do instead}
    \begin{itemize}
      \item Qualify names: \cppinline{std::vector}, \cppinline{std::cout}, \ldots
      \item Put things that belong together in the same namespace
      \item Use \textit{using declarations} in local scopes: \cppinline{using std::cout;}
    \end{itemize}
  \end{block}
\end{frame}

\end{advanced}
