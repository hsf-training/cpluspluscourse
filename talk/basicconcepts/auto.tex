\subsection[auto]{Auto keyword}

\begin{frame}[fragile]
  \frametitlecpp[11]{Auto keyword}
  \begin{block}{Reason of being}
    \begin{itemize}
    \item Many type declarations are redundant
    \item They are often a source for compiler warnings and errors
    \item Using auto prevents unwanted/unnecessary type conversions
    \end{itemize}
    \begin{cppcode*}{}
      std::vector<int> v;
      int a = v[3];
      int b = v.size();  // bug ? unsigned to signed
    \end{cppcode*}
  \end{block}
  \pause
  \begin{block}{Practical usage}
    \begin{cppcode*}{}
      std::vector<int> v;
      auto a = v[3];
      const auto b = v.size();
      int sum{0};
      for (auto n : v) { sum += n; }
    \end{cppcode*}
  \end{block}
\end{frame}

\begin{frame}[fragile]
  \frametitlecpp[98]{Loops, references, auto}
  \begin{exerciseWithShortcut}{Loops, references, auto}{Loops, refs, auto}
    Familiarise yourself with range-based for loops and references
    \begin{itemize}
      \item Go to \texttt{code/loopsRefsAuto}
      \item Look at \texttt{loopsRefsAuto.cpp}
      \item Compile it (\texttt{make}) and run the program (\texttt{./loopsRefsAuto})
      \item Work on the tasks that you find in \texttt{loopsRefsAuto.cpp}
    \end{itemize}
  \end{exerciseWithShortcut}
\end{frame}
