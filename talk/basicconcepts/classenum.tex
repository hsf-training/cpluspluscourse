\subsection[Class/Enum]{Class and enum types}

\begin{frame}[fragile]
  \frametitlecpp[98]{struct}
  \begin{mdframed}[style=simplebox]
    \center ``members'' grouped together under one name
  \end{mdframed}
  \begin{multicols}{2}
    \begin{cppcode*}{gobble=2}
      struct Individual {
        unsigned char age;
        float weight;
      };

      Individual student;
      student.age = 25;
      student.weight = 78.5f;

      Individual teacher = {
        45, 67.0f
      };
    \end{cppcode*}
    \columnbreak
    \begin{cppcode*}{gobble=2,firstnumber=14}
      Individual *ptr = &student;
      ptr->age = 25;
      // same as: (*ptr).age = 25;
    \end{cppcode*}
    \pause
    \vfill
    \hspace{-1.5cm}
    \begin{tikzpicture}
      \memorystack[nb blocks=5]
      \onslide<3-> {
        \memorypush{25,?,?,?}
        \memorypushwidevalue{78.5}
        \memorystruct{1}{2}{\scriptsize student}
      }
      \onslide<4-> {
        \memorypush{45,?,?,?}
        \memorypushwidevalue{67.0}
        \memorystruct{3}{4}{\scriptsize teacher}
      }
    \end{tikzpicture}
    \vfill \null
  \end{multicols}
\end{frame}

\begin{frame}[fragile]
  \frametitlecpp[98]{union}
  \begin{mdframed}[style=simplebox]
    \center ``members'' packed together at same memory location
  \end{mdframed}
  \begin{multicols}{2}
    \begin{cppcode*}{gobble=2}
      union Duration {
        int seconds;
        short hours;
        char days;
      };
      Duration d1, d2, d3;
      d1.seconds = 259200;
      d2.hours = 72;
      d3.days = 3;
      d1.days = 3; // d1.seconds overwritten
      int a = d1.seconds; // d1.seconds is garbage
    \end{cppcode*}
    \pause
    \columnbreak
    \null \vfill
    \begin{tikzpicture}
      \clip (0,0) rectangle (6cm, 3cm);
      \memorystack[word size=4,nb blocks=4]
      \visible<3-5>{\memorypushwidevalue{d1 259200}}
      \onslide<4->{\memorypushhalfvalue{d2 72}}
      \memorygoto{2}
      \onslide<4->{\memorypush{,,?,?}}
      \onslide<5->{\memorypush{d3 3,?,?,?}}
      \memorygoto{1}
      \onslide<6->{\memorypush{d1 3,?,?,?}}
    \end{tikzpicture}
    \vfill \null
  \end{multicols}
  \onslide<7->{
  \begin{alertblock}{}
    Starting with \cpp17: prefer \mintinline{cpp}{std::variant}
  \end{alertblock}
  }
\end{frame}

\begin{frame}[fragile]
  \frametitlecpp[98]{Enums}
  \begin{block}{}
    \begin{itemize}
        \item use to declare a list of related constants (enumerators)
        \item has an underlying integral type
        \item enumerator names leak into enclosing scope
    \end{itemize}
  \end{block}
  \begin{multicols}{2}
    \begin{cppcode*}{gobble=2}
      enum VehicleType {

        BIKE,  // 0
        CAR,   // 1
        BUS,   // 2
      };
      VehicleType t = CAR;
    \end{cppcode*}
    \columnbreak
    \begin{cppcode*}{gobble=2}
      enum VehicleType
        : int { // C++11
        BIKE = 3,
        CAR = 5,
        BUS = 7,
      };
      VehicleType t2 = BUS;
    \end{cppcode*}
  \end{multicols}
\end{frame}

\begin{frame}[fragile]
  \frametitlecpp[11]{Scoped enumeration, aka enum class}
  \begin{block}{Same syntax as enum, with scope}
    \begin{cppcode*}{}
      enum class VehicleType { Bus, Car };
      VehicleType t = VehicleType::Car;
    \end{cppcode*}
  \end{block}
  \pause
  \begin{exampleblock}{Only advantages}
    \begin{itemize}
    \item scopes enumerator names, avoids name clashes
    \item strong typing, no automatic conversion to int
    \end{itemize}
    \small
    \begin{cppcode*}{}
      enum VType { Bus, Car }; enum Color { Red, Blue };
      VType t = Bus;
      if (t == Red) { /* We do enter */ }
      int a = 5 * Car; // Ok, a = 5

      enum class VT { Bus, Car }; enum class Col { Red, Blue };
      VT t = VT::Bus;
      if (t == Col::Red) { /* Compiler error */ }
      int a = t * 5;       // Compiler error
    \end{cppcode*}
  \end{exampleblock}
\end{frame}

\begin{frame}[fragile]
  \frametitlecpp[98]{More sensible example}
  \begin{multicols}{2}
    \begin{cppcode*}{gobble=2}
      enum class ShapeType {
        Circle,
        Rectangle
      };

      struct Rectangle {
        float width;
        float height;
      };
    \end{cppcode*}
    \columnbreak
    \pause
    \begin{cppcode*}{gobble=2,firstnumber=10}
      struct Shape {
        ShapeType type;
        union {
          float radius;
          Rectangle rect;
        };
      };
    \end{cppcode*}
  \end{multicols}
  \pause
  \begin{multicols}{2}
    \begin{cppcode*}{gobble=2,firstnumber=17}
      Shape s;
      s.type =
        ShapeType::Circle;
      s.radius = 3.4;

    \end{cppcode*}
    \columnbreak
    \begin{cppcode*}{gobble=2,firstnumber=20}
      Shape t;
      t.type =
        Shapetype::Rectangle;
      t.rect.width = 3;
      t.rect.height = 4;
    \end{cppcode*}
  \end{multicols}
\end{frame}

\begin{frame}[fragile]
  \frametitle{typedef and using \hfill \cpp98 / \cpp11}
  Used to create type aliases
  \begin{alertblock}{\cpp98}
    \begin{cppcode*}{gobble=2}
      typedef uint64_t myint;
      myint toto = 17;
      typedef int pos[3];
    \end{cppcode*}
  \end{alertblock}
  \begin{exampleblock}{\cpp11}
    \begin{cppcode*}{gobble=2}
      using myint = uint64_t;
      myint toto = 17;
      using pos = int[3];

      template <typename T> using myvec = std::vector<T>;
      myvec<int> titi;
    \end{cppcode*}
  \end{exampleblock}
\end{frame}
