\subsection[Control]{Control structures}

\begin{frame}[fragile]
  \frametitlecpp[98]{Control structures: if}
  \begin{block}{if syntax}
    \begin{cppcode*}{}
      if (condition1) {
        Statement1; Statement2;
      } else if (condition2)
        OnlyOneStatement;
      else {
        Statement3;
        Statement4;
      }
    \end{cppcode*}
    \begin{itemize}
      \item \mintinline{cpp}{else} and \mintinline{cpp}{else if} clause are optional
      \item \mintinline{cpp}{else if} clause can be repeated
      \item braces are optional if there is a single statement
    \end{itemize}
  \end{block}
\end{frame}

\begin{frame}[fragile]
  \frametitlecpp[98]{Control structures: if}
  \begin{exampleblock}{Practical example}
    \begin{cppcode*}{}
      int collatz(int a) {
        if (a <= 0) {
          std::cout << "not supported";
          return 0;
        } else if (a == 1) {
          return 1;
        } else if (a%2 == 0) {
          return collatz(a/2);
        } else {
          return collatz(3*a+1);
        }
      }
    \end{cppcode*}
  \end{exampleblock}
\end{frame}

\begin{frame}[fragile]
  \frametitlecpp[98]{Control structures: conditional operator}
  \begin{block}{Syntax}
    \begin{cppcode*}{linenos=false}
      test ? expression1 : expression2;
    \end{cppcode*}
    \vspace{-0.2cm}
    \begin{itemize}
      \item if test is \mintinline{cpp}{true} expression1 is returned
      \item else expression2 is returned
    \end{itemize}
  \end{block}
  \pause
  \begin{exampleblock}{Practical example}
    \begin{cppcode*}{}
      int collatz(int a) {
        return a==1 ? 1 : collatz(a%2==0 ? a/2 : 3*a+1);
      }
    \end{cppcode*}
  \end{exampleblock}
  \pause
  \begin{alertblock}{Do not abuse}
    \begin{itemize}
      \item explicit ifs are easier to read \\
      \item use only when obvious and not nested
    \end{itemize}
  \end{alertblock}
\end{frame}

\begin{frame}[fragile]
  \frametitlecpp[98]{Control structures: switch}
  \begin{block}{Syntax}
    \begin{cppcode*}{gobble=2}
      switch(identifier) {
        case c1 : statements1; break;
        case c2 : statements2; break;
        case c3 : statements3; break;
        ...
        default : instructiond; break;
      }
    \end{cppcode*}
    \begin{itemize}
      \item \mintinline{cpp}{break} is not mandatory but...
      \item cases are entry points, not independent pieces
      \item execution falls through to the next case without a \mintinline{cpp}{break}!
      \item \mintinline{cpp}{default} may be omitted
    \end{itemize}
  \end{block}
  \pause
  \begin{alertblock}{Use break}
    Do not try to make use of non breaking cases
  \end{alertblock}
\end{frame}

\begin{frame}[fragile]
  \frametitlecpp[98]{Control structures: switch}
  \begin{exampleblock}{Practical example}
    \begin{cppcode*}{}
      enum class Lang { French, German, English, Other };
      ...
      switch (language) {
      case Lang::French:
        std::cout << "Bonjour";
        break;
       case Lang::German:
        std::cout << "Guten Tag";
        break;
      case Lang::English:
        std::cout << "Good morning";
        break;
      default:
        std::cout << "I do not speak your language";
      }
    \end{cppcode*}
  \end{exampleblock}
\end{frame}

\AtBeginEnvironment{minted}{\renewcommand{\fcolorbox}[4][]{#4}}

\begin{frame}[fragile]
  \frametitlecpp[17]{\texttt{[[fallthrough]]} attribute}
  \begin{block}{New compiler warning}
    Since \cpp17, compilers are encouraged to warn on fall-through
  \end{block}
  \begin{exampleblock}{\cpp17}
    \begin{cppcode*}{}
      switch (c) {
        case 'a':
          f();    // Warning emitted
        case 'b': // Warning emitted
        case 'c':
          g();
          [[fallthrough]]; // Warning suppressed
        case 'd':
          h();
      }
    \end{cppcode*}
  \end{exampleblock}
\end{frame}

\begin{frame}[fragile]
  \frametitlecpp[17]{init-statements for if and switch}
  \begin{block}{}
    Allows to limit variable scope in \mintinline{cpp}{if} and \mintinline{cpp}{switch} statements
  \end{block}
  \begin{exampleblock}{\cpp17}
    \begin{cppcode*}{}
      if (Value val = GetValue(); condition(val)) {
        f(val);
      } else {
        g(val);
      }
      h(val); // compile error
    \end{cppcode*}
  \end{exampleblock}
  \pause
  \begin{alertblock}{\cpp98}
    Don't confuse with a variable declaration as condition:
    \begin{cppcode*}{}
      if (Value* val = GetValuePtr())
        f(*val);
    \end{cppcode*}
  \end{alertblock}
\end{frame}

\begin{frame}[fragile]
  \frametitlecpp[98]{Control structures: for loop}
  \begin{block}{for loop syntax}
    \begin{cppcode*}{}
      for(initializations; condition; increments) {
        statements;
      }
    \end{cppcode*}
    \vspace{-0.2cm}
    \begin{itemize}
      \item initializations and increments are comma separated
      \item initializations can contain declarations
      \item braces are optional if loop body is a single statement
    \end{itemize}
  \end{block}
  \pause
  \begin{exampleblock}{Practical example}
    \begin{cppcode*}{}
      for(int i = 0, j = 0 ; i < 10 ; i++, j = i*i) {
        std::cout << i << "^2 is " << j << '\n';
      }
    \end{cppcode*}
  \end{exampleblock}
  \pause
  \begin{goodpractice}{Don't abuse the \texttt{for} syntax}
    \begin{itemize}
      \item The \mintinline{cpp}{for} loop head should fit in 1-3 lines
    \end{itemize}
  \end{goodpractice}
\end{frame}

\begin{frame}[fragile]
  \frametitlecpp[11]{Range-based loops}
  \begin{block}{Reason of being}
    \begin{itemize}
    \item simplifies loops over ``ranges'' tremendously
    \item especially with STL containers
    \end{itemize}
  \end{block}
  \begin{block}{Syntax}
    \begin{cppcode*}{}
      for ( type iteration_variable : range ) {
        // body using iteration_variable
      }
    \end{cppcode*}
  \end{block}
  \begin{exampleblock}{Example code}
    \begin{cppcode*}{}
      int v[4] = {1,2,3,4};
      int sum = 0;
      for (int a : v) { sum += a; }
    \end{cppcode*}
  \end{exampleblock}
\end{frame}

\begin{frame}[fragile]
  \frametitlecpp[98]{Control structures: while loop}
  \begin{block}{while loop syntax}
    \begin{cppcode*}{}
      while(condition) {
        statements;
      }
      do {
        statements;
      } while(condition);
    \end{cppcode*}
    \begin{itemize}
      \item braces are optional if the body is a single statement
    \end{itemize}
  \end{block}
  \pause
  \begin{alertblock}{Bad example}
    \begin{cppcode*}{}
      while (n != 1)
        if (0 == n%2) n /= 2;
        else n = 3 * n + 1;
    \end{cppcode*}
  \end{alertblock}
\end{frame}

\begin{frame}[fragile]
  \frametitlecpp[98]{Control structures: jump statements}
  \begin{block}{}
    \begin{description}
    \item[break] exits the loop and continues after it
    \item[continue] goes immediately to next loop iteration
    \item[return] exists the current function
    \item[goto] can jump anywhere inside a function, don't use!
    \end{description}
  \end{block}
  \pause
  \begin{alertblock}{Bad example}
    \begin{cppcode*}{}
      while (1) {
        if (n == 1) break;
        if (0 == n%2) {
          std::cout << n << '\n';
          n /= 2;
          continue;
        }
        n = 3 * n + 1;
      }
    \end{cppcode*}
  \end{alertblock}
\end{frame}

\begin{frame}[fragile]
  \frametitlecpp[11]{Control structures}
  \begin{exerciseWithShortcut}{Control structures}{Control structs}
    Familiarise yourself with different kinds of control structures. Re-implement them in different ways.
    \begin{itemize}
      \item Go to \texttt{code/control}
      \item Look at \texttt{control.cpp}
      \item Compile it (\texttt{make}) and run the program (\texttt{./control})
      \item Work on the tasks that you find in \texttt{README.md}
    \end{itemize}
  \end{exerciseWithShortcut}
\end{frame}
