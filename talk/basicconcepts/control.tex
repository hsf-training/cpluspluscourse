\subsection[Control]{Control instructions}

\begin{frame}[fragile]
  \frametitlecpp[98]{Control instructions : if}
  \begin{block}{if syntax}
    \begin{cppcode*}{}
      if (condition1) {
        Instructions1;
      } else if (condition2) {
        Instructions2;
      } else {
        Instructions3;
      }
    \end{cppcode*}
    \begin{itemize}
      \item {\it else} and {\it else if} part are optional
      \item {\it else if} part can be repeated
      \item braces are optional if there is a single instruction
    \end{itemize}
  \end{block}
\end{frame}

\begin{frame}[fragile]
  \frametitlecpp[98]{Control instructions : if}
  \begin{exampleblock}{Practical example}
    \begin{cppcode*}{}
      int collatz(int a) {
        if (a <= 0) {
          std::cout << "not supported";
          return 0;
        } else if (a == 1) {
          return 1;
        } else if (a%2 == 0) {
          return collatz(a/2);
        } else {
          return collatz(3*a+1);
        }
      }
    \end{cppcode*}
  \end{exampleblock}
\end{frame}

\begin{frame}[fragile]
  \frametitlecpp[98]{Control instructions : conditional operator}
  \begin{block}{Syntax}
    \begin{cppcode*}{linenos=false}
      test ? expression1 : expression2;
    \end{cppcode*}
    \vspace{-0.3cm}
    \begin{itemize}
      \item if test is {\it true} expression1 is returned
      \item else expression 2 is returned
    \end{itemize}
  \end{block}
  \pause
  \begin{exampleblock}{Practical example}
    \begin{cppcode*}{}
      int collatz(int a) {
        return a==1 ? 1 : collatz(a%2 ? 3*a+1 : a/2);
      }
    \end{cppcode*}
  \end{exampleblock}
  \pause
  \begin{alertblock}{Do not abuse}
    explicit ifs are easier to read \\
    to be used only when obvious and not nested
  \end{alertblock}
\end{frame}

\begin{frame}[fragile]
  \frametitlecpp[98]{Control instructions : switch}
  \begin{block}{Syntax}
    \begin{cppcode*}{gobble=2}
      switch(identifier) {
        case c1 : instructions1; break;
        case c2 : instructions2; break;
        case c3 : instructions3; break;
        ...
        default : instructiond; break;
      }
    \end{cppcode*}
    \begin{itemize}
      \item {\it break} is not mandatory but...
      \item cases are entry points, not independent pieces
      \item execution carries on with the next case if no {\it break} is present !
      \item {\it default} may be omitted
    \end{itemize}
  \end{block}
  \pause
  \begin{alertblock}{Use break}
    Do not try to make use of non breaking cases
  \end{alertblock}
\end{frame}

\begin{frame}[fragile]
  \frametitlecpp[98]{Control instructions : switch}
  \begin{exampleblock}{Practical example}
    \begin{cppcode*}{}
      enum class Lang { FRENCH, GERMAN, ENGLISH, OTHER };
      ...
      switch (language) {
      case Lang::FRENCH:
        printf("Bonjour");
        break;
       case Lang::GERMAN:
        printf("Guten Tag");
        break;
      case Lang::ENGLISH:
        printf("Good morning");
        break;
      default:
        printf("I do not speak your language");
      }
    \end{cppcode*}
  \end{exampleblock}
\end{frame}

\AtBeginEnvironment{minted}{\renewcommand{\fcolorbox}[4][]{#4}}

\begin{frame}[fragile]
  \frametitlecpp[17]{\texttt{[[fallthrough]]} attribute}
  \begin{alertblock}{\cpp14}
    \begin{cppcode}
      switch (c) {
        case 'a':
          f(); // Warning emitted
        case 'c':
          h();
      }
    \end{cppcode}
  \end{alertblock}
  \begin{exampleblock}{\cpp17}
    \begin{cppcode*}{}
      switch (c) {
        case 'a':
          f();
          [[fallthrough]]; // Warning suppressed
        case 'c':
          h();
      }
    \end{cppcode*}
  \end{exampleblock}
\end{frame}

\begin{frame}[fragile]
  \frametitlecpp[17]{init-statements for if and switch}
  Allows to simplify if and switch statements
  \begin{alertblock}{\cpp14}
    \begin{cppcode*}{}
      Value val = GetValue();
      if (condition(val)) {
        // on success
      } else {
        // on false...
      }
    \end{cppcode*}
  \end{alertblock}
  \begin{exampleblock}{\cpp17}
    \begin{cppcode*}{}
      if (Value val = GetValue(); condition(val)) {
        // on success
      } else {
        // on false...
      }
    \end{cppcode*}
    \vspace{-.1cm}
    val is visible only inside the if and else statements
  \end{exampleblock}
\end{frame}

\begin{frame}[fragile]
  \frametitlecpp[98]{Control instructions : for loop}
  \begin{block}{for loop syntax}
    \begin{cppcode*}{}
      for(initializations; condition; increments) {
        instructions;
      }
    \end{cppcode*}
    \vspace{-0.2cm}
    \begin{itemize}
      \item initializations and increments are comma separated
      \item initializations can contain declarations
      \item braces are optional if there is a single instruction
    \end{itemize}
  \end{block}
  \pause
  \begin{exampleblock}{Practical example}
    \begin{cppcode*}{}
      for(int i = 0, j = 0 ; i < 10 ; i++, j = i*i) {
        std::cout << i << "^2 is " << j << '\n';
      }
    \end{cppcode*}
  \end{exampleblock}
  \pause
  \begin{alertblock}{Do not abuse the syntax}
    The for statement should fit in 1-3 lines
  \end{alertblock}
\end{frame}

\begin{frame}[fragile]
  \frametitlecpp[11]{Range based loops}
  \begin{block}{Reason of being}
    \begin{itemize}
    \item simplifies loops tremendously
    \item especially with STL containers
    \end{itemize}
  \end{block}
  \begin{block}{Syntax}
    \begin{cppcode*}{}
      for ( type iteration_variable : container ) {
        // body using iteration_variable
      }
    \end{cppcode*}
  \end{block}
  \begin{exampleblock}{Example code}
    \begin{cppcode*}{}
      int v[4] = {1,2,3,4};
      int sum = 0;
      for (int a : v) { sum += a; }
    \end{cppcode*}
  \end{exampleblock}
\end{frame}

\begin{frame}[fragile]
  \frametitlecpp[98]{Control instructions : while loop}
  \begin{block}{while loop syntax}
    \begin{cppcode*}{}
      while(condition) {
        instructions;
      }
      do {
        Instructions;
      } while(condition);
    \end{cppcode*}
    \vspace{-0.3cm}
    \begin{itemize}
      \item braces are optional if there is a single instruction
    \end{itemize}
  \end{block}
  \pause
  \begin{alertblock}{Practical example}
    \begin{cppcode*}{}
      while (n != 1)
        if (0 == n%2) n /= 2;
        else n = 3 * n + 1;
    \end{cppcode*}
  \end{alertblock}
\end{frame}

\begin{frame}[fragile]
  \frametitlecpp[98]{Control instructions : commands}
  \begin{block}{control commands}
    \begin{description}
    \item[break] goes out of the loop
    \item[continue] goes immediately to next iteration
    \item[return] goes out of current function
    \end{description}
  \end{block}
  \pause
  \begin{alertblock}{Practical example}
    \begin{cppcode*}{}
      while (1) {
        if (n == 1) break;
        if (0 == n%2) {
          std::cout << n << '\n';
          n /= 2;
          continue;
        }
        n = 3 * n + 1;
      }
    \end{cppcode*}
  \end{alertblock}
\end{frame}
