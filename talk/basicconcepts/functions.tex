\subsection[$f()$]{Functions}

\begin{frame}[fragile]
  \frametitlecpp[98]{Functions}
  \begin{multicols}{2}
    \begin{cppcode*}{}
      // with return type
      int square(int a) {
        return a * a;
      }

      // multiple parameters
      int mult(int a,
               int b) {
        return a * b;
      }
    \end{cppcode*}
    \columnbreak
    \begin{cppcode*}{firstnumber=11}
      // no return
      void log(char* msg) {
        std::cout << msg;
      }

      // no parameter
      void hello() {
        std::cout << "Hello World";
      }
    \end{cppcode*}
  \end{multicols}

  \pause

  \begin{exampleblock}{Functions and references to returned values}
    \begin{cppcode}
      int result = square(2);
      int & temp = square(2);        // Not allowed
      int const & temp2 = square(2); // OK
    \end{cppcode}
  \end{exampleblock}
\end{frame}

\begin{frame}[fragile]
  \frametitlecpp[98]{Function default arguments}
  \begin{multicols}{2}
    \begin{cppcode*}{}
      // must be the trailing
      // argument
      int add(int a,
              int b = 2) {
        return a + b;
      }
      // add(1) == 3
      // add(3,4) == 7

    \end{cppcode*}
    \columnbreak
    \begin{cppcode*}{firstnumber=11}
      // multiple default
      // arguments are possible
      int add(int a = 2,
              int b = 2) {
        return a + b;
      }
      // add() == 4
      // add(3) == 5
    \end{cppcode*}
  \end{multicols}
\end{frame}


\begin{frame}[fragile]
  \frametitlecpp[98]{Functions: parameters are passed by value}
  \begin{multicols}{2}
    % the code has to be repeated n times here. Anyone finding a better solution
    % is welcome, but it's not a trivial task, due to the verbatim nature of minted
    \begin{overprint}[\columnwidth]
      \onslide<1-2>
      \begin{minted}[linenos,highlightlines=2]{cpp}
struct BigStruct {...};
BigStruct s;

// parameter by value
void printVal(BigStruct p) {
  ...
}
printVal(s); // copy

// parameter by reference
void printRef(BigStruct &q) {
  ...
}
printRef(s); // no copy
      \end{minted}
      \onslide<3>
      \begin{minted}[linenos,highlightlines={5,8}]{cpp}
struct BigStruct {...};
BigStruct s;

// parameter by value
void printVal(BigStruct p) {
  ...
}
printVal(s); // copy

// parameter by reference
void printRef(BigStruct &q) {
  ...
}
printRef(s); // no copy
      \end{minted}
      \onslide<4->
      \begin{minted}[linenos,highlightlines={11,14}]{cpp}
struct BigStruct {...};
BigStruct s;

// parameter by value
void printVal(BigStruct p) {
  ...
}
printVal(s); // copy

// parameter by reference
void printRef(BigStruct &q) {
  ...
}
printRef(s); // no copy
      \end{minted}
    \end{overprint}
    \columnbreak
    \null \vfill
    \begin{tikzpicture}
      \memorystack[word size=1, block size=80, nb blocks=7, size x=3cm]
      \onslide<2-> {
        \memorypush{s1}
        \memorypush{...}
        \memorypush{sn}
        \memorystruct{1}{3}{s}
      }
      \onslide<3> {
        \memorypush{p1 = s1}
        \memorypush{...}
        \memorypush{pn = sn}
        \memorystruct{4}{6}{p}
      }
      \memorygoto{4}
      \onslide<4> {
        \memorypushpointer[q =]{1}
      }
    \end{tikzpicture}
    \vfill \null
  \end{multicols}
\end{frame}

\begin{frame}[fragile]
  \frametitlecpp[98]{Functions: pass by value or reference?}
  \begin{multicols}{2}
    % the code has to be repeated n times here. Anyone finding a better solution
    % is welcome, but it's not a trivial task, due to the verbatim nature of minted
    \begin{overprint}[\columnwidth]
      \onslide<1>
      \begin{minted}[linenos]{cpp}
struct SmallStruct {int a;};
SmallStruct s = {1};

void changeVal(SmallStruct p) {
  p.a = 2;
}
changeVal(s);
// s.a == 1

void changeRef(SmallStruct &q) {
  q.a = 2;
}
changeRef(s);
// s.a == 2
      \end{minted}
      \onslide<2>
      \begin{minted}[linenos,highlightlines=2]{cpp}
struct SmallStruct {int a;};
SmallStruct s = {1};

void changeVal(SmallStruct p) {
  p.a = 2;
}
changeVal(s);
// s.a == 1

void changeRef(SmallStruct &q) {
  q.a = 2;
}
changeRef(s);
// s.a == 2
      \end{minted}
      \onslide<3>
      \begin{minted}[linenos,highlightlines={4,7}]{cpp}
struct SmallStruct {int a;};
SmallStruct s = {1};

void changeVal(SmallStruct p) {
  p.a = 2;
}
changeVal(s);
// s.a == 1

void changeRef(SmallStruct &q) {
  q.a = 2;
}
changeRef(s);
// s.a == 2
      \end{minted}
      \onslide<4>
      \begin{minted}[linenos,highlightlines=5]{cpp}
struct SmallStruct {int a;};
SmallStruct s = {1};

void changeVal(SmallStruct p) {
  p.a = 2;
}
changeVal(s);
// s.a == 1

void changeRef(SmallStruct &q) {
  q.a = 2;
}
changeRef(s);
// s.a == 2
      \end{minted}
      \onslide<5>
      \begin{minted}[linenos,highlightlines=8]{cpp}
struct SmallStruct {int a;};
SmallStruct s = {1};

void changeVal(SmallStruct p) {
  p.a = 2;
}
changeVal(s);
// s.a == 1

void changeRef(SmallStruct &q) {
  q.a = 2;
}
changeRef(s);
// s.a == 2
      \end{minted}
      \onslide<6>
      \begin{minted}[linenos,highlightlines={10,13}]{cpp}
struct SmallStruct {int a;};
SmallStruct s = {1};

void changeVal(SmallStruct p) {
  p.a = 2;
}
changeVal(s);
// s.a == 1

void changeRef(SmallStruct &q) {
  q.a = 2;
}
changeRef(s);
// s.a == 2
      \end{minted}
      \onslide<7>
      \begin{minted}[linenos,highlightlines=11]{cpp}
struct SmallStruct {int a;};
SmallStruct s = {1};

void changeVal(SmallStruct p) {
  p.a = 2;
}
changeVal(s);
// s.a == 1

void changeRef(SmallStruct &q) {
  q.a = 2;
}
changeRef(s);
// s.a == 2
      \end{minted}
      \onslide<8>
      \begin{minted}[linenos,highlightlines=14]{cpp}
struct SmallStruct {int a;};
SmallStruct s = {1};

void changeVal(SmallStruct p) {
  p.a = 2;
}
changeVal(s);
// s.a == 1

void changeRef(SmallStruct &q) {
  q.a = 2;
}
changeRef(s);
// s.a == 2
      \end{minted}
    \end{overprint}
    \columnbreak
    \null \vfill
    \begin{tikzpicture}
      \memorystack[word size=1, block size=4, nb blocks=3, size x=3cm]
      \onslide<2-6> {
        \memorypush{s.a = 1}
      }
      \memorygoto{1}
      \onslide<7-> {
        \memorypush{s.a = 2}
      }

      \memorygoto{2}
      \onslide<3> {
        \memorypush{p.a = 1}
      }
      \memorygoto{2}
      \onslide<4> {
        \memorypush{p.a = 2}
      }
      \memorygoto{2}
      \onslide<6-7> {
        \memorypushpointer[q =]{1}
      }
      \end{tikzpicture}
    \vfill \null
  \end{multicols}
\end{frame}

\begin{frame}[fragile]
  \frametitlecpp[98]{Pass by value, reference or pointer}
  \begin{block}{Different ways to pass arguments to a function}
    \begin{itemize}
    \item By default, arguments are passed by value (= copy) \\
          good for small types, e.g.\ numbers
    \item Use references for parameters to avoid copies \\
          good for large types, e.g.\ objects
    \item Use \cppinline{const} for safety and readability whenever possible
    \end{itemize}
  \end{block}
  \pause
  \begin{block}{Syntax}
    \begin{cppcode*}{escapeinside=||}
struct T {...}; T a;
void fVal(T value);        fVal(a);   // by value
void fRef(const T &value); fRef(a);   // by reference
void fPtr(const T *value); fPtr(|{\setlength{\fboxsep}{0pt}\color{gray}\colorbox{yellow}{\textsc{&}}}|a);  // by pointer
void fWrite(T &value);     fWrite(a); // non-const ref
    \end{cppcode*}
  \end{block}
\end{frame}

\begin{frame}[fragile]
    \frametitlecpp[98]{Overloading}
    \begin{block}{Overloading}
        \begin{itemize}
            \item We can have multiple functions with the same name
            \begin{itemize}
                \item Must have different parameter lists
                \item A different return type alone is not allowed
                \item Form a so-called ``overload set''
            \end{itemize}
            \item Default arguments can cause ambiguities
        \end{itemize}
    \end{block}
    \begin{exampleblock}{}
        \begin{cppcode*}{}
            int sum(int b);             // 1
            int sum(int b, int c);      // 2, ok, overload
            // float sum(int b, int c); // disallowed
            sum(42); // calls 1
            sum(42, 43); // calls 2
            int sum(int b, int c, int d = 4); // 3, overload
            sum(42, 43, 44); // calls 3
            sum(42, 43);     // error: ambiguous, 2 or 3
        \end{cppcode*}
    \end{exampleblock}
\end{frame}

\begin{frame}[fragile]
  \frametitlecpp[98]{Functions}
  \begin{exercise}{Functions}
    Familiarise yourself with pass by value / pass by reference.
    \begin{itemize}
      \item Go to \texttt{exercises/functions}
      \item Look at \texttt{functions.cpp}
      \item Compile it (\texttt{make}) and run the program (\texttt{./functions})
      \item Work on the tasks that you find in \texttt{functions.cpp}
    \end{itemize}
  \end{exercise}
\end{frame}

\begin{frame}[fragile]
  \frametitlecpp[98]{Functions: good practices}
  \begin{onlyenv}<1>
    \begin{goodpractice}{Write readable functions}
      \begin{itemize}
        \item Keep functions short
        \item Do one logical thing (single-responsibility principle)
        \item Use expressive names
        \item Document non-trivial functions
      \end{itemize}
    \end{goodpractice}
    \begin{exampleblock}{Example: Good}
      \begin{cppcode*}{}
        /// Count number of dilepton events in data.
        /// \param d Dataset to search.
        unsigned int countDileptons(Data &d) {
          selectEventsWithMuons(d);
          selectEventsWithElectrons(d);
          return d.size();
        }
      \end{cppcode*}
    \end{exampleblock}
  \end{onlyenv}
  \begin{onlyenv}<2->
    \begin{alertblock}{Example: don't! Everything in one long function}
      \begin{multicols}{2}
        \begin{cppcode*}{}
          unsigned int runJob() {
            // Step 1: data
            Data data;
            data.resize(123456);
            data.fill(...);

            // Step 2: muons
            for (....) {
              if (...) {
                data.erase(...);
              }
            }
            // Step 3: electrons
            for (....) {
        \end{cppcode*}
        \columnbreak
        \begin{cppcode*}{firstnumber=last}
              if (...) {
                data.erase(...);
              }
            }

            // Step 4: dileptons
            int counter = 0;
            for (....) {
              if (...) {
                counter++;
              }
            }

            return counter;
          }
        \end{cppcode*}
      \end{multicols}
    \end{alertblock}
  \end{onlyenv}
\end{frame}
