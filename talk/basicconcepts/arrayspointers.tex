\subsection[Ptr]{Arrays and Pointers}

\begin{frame}[fragile]
  \frametitlecpp[98]{Static arrays}
  \begin{cppcode}
    int ai[4] = {1,2,3,4};
    int ai[] = {1,2,3,4};  // identical

    char ac[3] = {'a','b','c'};   // char array
    char ac[4] = "abc";           // valid C string
    char ac[4] = {'a','b','c',0}; // same valid string

    int i = ai[2];  // i = 3
    char c = ac[8]; // at best garbage, may segfault
    int i = ai[4];  // also garbage !
  \end{cppcode}
\end{frame}

\Scontents*[store-cmd=code_arrays]{
int i = 4;
int *pi = &i;
int j = *pi + 1;

int ai[] = {1,2,3};
int *pai = ai; // decay to ptr
int *paj = pai + 1;
int k = *paj + 1;

// compile error
int *pak = k;

// seg fault !
int *pak = (int*)k;
int l = *pak;
}
\begin{frame}[fragile]
  \frametitlecpp[98]{Pointers}
  \begin{multicols}{2}
  \begin{overprint}[\columnwidth]
    \onslide<1> \highlightCppCode{}{code_arrays}
    \onslide<2> \highlightCppCode{1}{code_arrays}
    \onslide<3> \highlightCppCode{2}{code_arrays}
    \onslide<4> \highlightCppCode{3}{code_arrays}
    \onslide<5> \highlightCppCode{5}{code_arrays}
    \onslide<6> \highlightCppCode{6}{code_arrays}
    \onslide<7> \highlightCppCode{7}{code_arrays}
    \onslide<8> \highlightCppCode{8}{code_arrays}
    \onslide<9> \highlightCppCode{11}{code_arrays}
  \end{overprint}
  \columnbreak
    \onslide<2->{
      \begin{tikzpicture}
        \memorystack[size x=3cm,word size=1,block size=4,nb blocks=11]
        \onslide<2-> {\memorypush{i = 4}}
        \onslide<3-> {\memorypushpointer[pi =]{1}}
        \onslide<4-> {\memorypush{j = 5}}
        \onslide<5-> {\memorypush{ai[0] = 1}}
        \onslide<5-> {\memorypush{ai[1] = 2}}
        \onslide<5-> {\memorypush{ai[2] = 3}}
        \onslide<6-> {\memorypushpointer[pai =]{4}}
        \onslide<7-> {\memorypushpointer[paj =]{5}}
        \onslide<8-> {\memorypush{k = 3}}
        \onslide<9-> {\memorypush{pak = 3}}
        \onslide<9-> {\draw[\stackcolor!80,->] (stack10-1.west) -- +(-0.5cm,0)
          node [anchor=east] {??};}
      \end{tikzpicture}
    }
  \end{multicols}
\end{frame}

\begin{frame}[fragile]
  \frametitlecpp[11]{nullptr}
  \begin{block}{A pointer to nothing}
    \begin{itemize}
    \item if a pointer doesn't point to anything, set it to \cppinline{nullptr}
    \begin{itemize}
      \item useful to e.g.\ mark the end of a linked data structure
      \item or absence of an optional function argument (pointer)
    \end{itemize}
    \item same as setting it to 0 or \cppinline{NULL} (before \cpp11)
    \item triggers compilation error when assigned to integer
    \end{itemize}
  \end{block}
  \pause
  \begin{exampleblock}{Example code}
    \begin{cppcode*}{}
      int* ip = nullptr;
      int i = NULL;      // compiles, bug?
      int i = nullptr;   // ERROR
    \end{cppcode*}
  \end{exampleblock}
\end{frame}

\begin{frame}[fragile]
  \frametitlecpp[98]{Dynamic arrays using C}
  \begin{cppcode}
    #include <cstdlib>
    #include <cstring>

    int *bad;          // pointer to random address
    int *ai = nullptr; // better, deterministic, testable

    // allocate array of 10 ints (uninitialized)
    ai = (int*) malloc(10*sizeof(int));
    memset(ai, 0, 10*sizeof(int)); // and set them to 0

    ai = (int*) calloc(10, sizeof(int)); // both in one go

    free(ai); // release memory
  \end{cppcode}
  \begin{goodpracticeWithShortcut}{Don't use C's memory management}{C's memory management}
    Use \cppinline{std::vector} and friends or smart pointers
  \end{goodpracticeWithShortcut}
\end{frame}

\begin{frame}[fragile]
	\frametitlecpp[98]{Manual dynamic arrays using \cpp}
	\begin{cppcode}
		#include <cstdlib>
		#include <cstring>

		// allocate array of 10 ints
		int* ai = new int[10];   // uninitialized
		int* ai = new int[10]{}; // zero-initialized

		delete[] ai; // release array memory

		// allocate a single int
		int* pi = new int;
		int* pi = new int{};
		delete pi; // release scalar memory
	\end{cppcode}
	\begin{goodpracticeWithShortcut}{Don't use manual memory management}{Manual memory management}
		Use \cppinline{std::vector} and friends or smart pointers
	\end{goodpracticeWithShortcut}
\end{frame}
