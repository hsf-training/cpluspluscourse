\section[conc]{Concurrency}

\subsection[thr]{Threads and async}

\begin{frame}[fragile]
  \frametitleii{Basic concurrency}
  \begin{block}{Threading}
    \begin{itemize}
    \item new object std::thread in \textless{}thread\textgreater{} header
    \item takes a function as argument of its constructor
    \item must be detached or joined before the main thread terminates
    \item \cpp20: std::jthread automatically joins at destruction
    \end{itemize}
  \end{block}
  \pause
  \begin{exampleblock}{Example code}
    \begin{cppcode*}{}
      void doSth() {...}
      void doSthElse() {...}
      int main() {
        std::thread t1(doSth);
        std::thread t2(doSthElse);
        for (auto t: {&t1,&t2}) t->join();
      }
    \end{cppcode*}
  \end{exampleblock}
\end{frame}

\begin{frame}[fragile]
  \frametitleii{The thread constructor}
  \begin{exampleblock}{Can take a function and its arguments}
    \begin{cppcode*}{}
      void function(int j, double j) {...};
      std::thread t1(function, 1, 2.0);
    \end{cppcode*}
  \end{exampleblock}
  \pause
  \begin{exampleblock}{Can take any function like object}
    \begin{cppcode*}{}
      struct AdderFunctor {
        AdderFunctor(int i): m_i(i) {}
        int operator() (int j) { return i+j; };
        int m_i;
      };
      std::thread t2(AdderFunctor(2), 5);
      int a;
      std::thread t3([](int i) { return i+2; }, a);
      std::thread t4([a]       { return a+2; });
    \end{cppcode*}
  \end{exampleblock}
\end{frame}

\begin{frame}[fragile]
  \frametitleii{Basic asynchronicity}
  \begin{block}{Concept}
    \begin{itemize}
    \item separation of the specification of what should be done and the retrieval of the results
    \item ``start working on this, and ping me when it's ready''
    \end{itemize}
  \end{block}
  \pause
  \begin{block}{Practically}
    \begin{itemize}
    \item std::async function launches an asynchronous task
    \item std::future template allows to handle the result
    \end{itemize}
  \end{block}
  \pause
  \begin{exampleblock}{Example code}
    \begin{cppcode*}{}
      int computeSth() {...}
      std::future<int> res = std::async(computeSth);
      std::cout << res->get() << std::endl;
    \end{cppcode*}
  \end{exampleblock}
\end{frame}

\begin{frame}[fragile]
  \frametitleii{Mixing the two}
  \begin{block}{Is async running concurrent code ?}
    \begin{itemize}
    \item it depends !
    \item you can control this with a launch policy argument
      \begin{description}
      \item[std::launch::async] spawns a thread for immediate execution
      \item[std::launch::deferred] causes lazy execution in current thread
      \end{description}
      \begin{itemize}
      \item execution starts when get() is called
      \end{itemize}
    \item default is not specified !
    \end{itemize}
  \end{block}
  \pause
  \begin{exampleblock}{Usage}
    \begin{cppcode*}{}
      int computeSth() {...}
      auto res = std::async(std::launch::async,
                            computeSth);
      auto res2 = std::async(std::launch::deferred,
                             computeSth);
    \end{cppcode*}
  \end{exampleblock}
\end{frame}

\begin{frame}[fragile]
  \frametitleii{Fine grained control on asynchronous execution}
  \begin{block}{std::packaged\_task template}
    \begin{itemize}
    \item creates an asynchronous version of any function like object
      \begin{itemize}
      \item identical arguments
      \item returns a std::future
      \end{itemize}
    \item provides access to the returned future
    \item associated with threads, gives full control on execution
    \end{itemize}
  \end{block}
  \pause
  \begin{exampleblock}{Usage}
    \begin{cppcode*}{}
      int task() { return 42; }
      std::packaged_task<int()> pckd_task(task);
      auto future = pckd_task.get_future();
      pckd_task();
      std::cout << future.get() << std::endl;
    \end{cppcode*}
  \end{exampleblock}
\end{frame}

\subsection[mutex]{Mutexes}

\begin{frame}[fragile]
  \frametitleii{Races}
  \begin{exampleblock}{Example code}
    \begin{cppcode*}{}
      int a = 0;
      void inc() { a++; };
      void inc100() {
        for (int i=0; i < 100; i++) inc();
      };
      int main() {
        std::thread t1(inc100);
        std::thread t2(inc100);
        for (auto t: {&t1,&t2}) t->join();
        std::cout << a << std::endl;
      }
    \end{cppcode*}
  \end{exampleblock}
  \pause
  \begin{block}{What do you expect ? Try it in code/race}
    \pause
    Anything between 100 and 200 !!!
  \end{block}
\end{frame}

\begin{frame}[fragile]
  \frametitleii{Atomicity}
  \begin{exampleblock}{Definition (wikipedia)}
    \begin{itemize}
    \item an operation (or set of operations) is atomic if it appears to the rest of the system to occur instantaneously
    \end{itemize}
  \end{exampleblock}
  \begin{block}{Practically}
    \begin{itemize}
    \item an operation that won't run concurrently to another one
    \item an operation that will have a stable environment during execution
    \end{itemize}
  \end{block}
  \pause
  \begin{alertblock}{Is ++ operator atomic ?}
    \pause
    Usually not. It behaves like :
    \begin{cppcode*}{}
      eax = a       // memory to register copy
      increase eax  // increase (atomic CPU instruction)
      a = eax       // copy back to memory
    \end{cppcode*}
  \end{alertblock}
\end{frame}

\begin{frame}[fragile]
  \frametitleii{Timing}
  \begin{exampleblock}{Code}
    \begin{cppcode*}{}
      eax = a       // memory to register copy
      increase eax  // increase (atomic CPU instruction)
      a = eax       // copy back to memory
    \end{cppcode*}
  \end{exampleblock}
  \begin{block}{For 2 threads}
    \begin{tikzpicture}
      \begin{umlseqdiag}
        \umlobject[x=0, class=eax]{Thread 1}
        \umlobject[x=3, class=a, fill=blue!20]{Memory}
        \umlobject[x=6, class=eax]{Thread 2}
        \begin{umlcall}[op=read, type=synchron, return=0]{Thread 1}{Memory}
        \end{umlcall}
        \begin{umlcall}[padding=3, op=read, type=synchron, return=0]{Thread 2}{Memory}
        \end{umlcall}
        \begin{umlcallself}[op=incr, type=synchron]{Thread 1}
        \end{umlcallself}
        \begin{umlcallself}[op=incr, type=synchron]{Thread 2}
        \end{umlcallself}
        \begin{umlcall}[op=write 1]{Thread 2}{Memory}
        \end{umlcall}
        \begin{umlcall}[padding=3, op=write 1]{Thread 1}{Memory}
        \end{umlcall}
      \end{umlseqdiag}
      \draw[-triangle 60](8.5,0) -- (8.5,-4) node[right, pos=0.5]{time};
    \end{tikzpicture}
  \end{block}
\end{frame}

\begin{frame}[fragile]
  \frametitleii{Mutexes}
  \begin{block}{Concept}
    \begin{itemize}
    \item a lock to serialize access to a non atomic piece of code
    \end{itemize}
  \end{block}
  \pause
  \begin{block}{The objects}
    \begin{description}[labelwidth=1.8cm]
    \item[std::mutex] in the mutex header
    \item[std::lock\_guard] for an RAII version of it
    \item[std::unique\_lock] same and can be released/relocked
    \end{description}
  \end{block}
  \pause
  \begin{exampleblock}{Practically}
    \begin{cppcode*}{}
      int a = 0;
      std::mutex m;
      void inc() {
        std::lock_guard<std::mutex> guard(m);
        a++;
      };
    \end{cppcode*}
  \end{exampleblock}
\end{frame}

\begin{frame}[fragile]
  \frametitleii{Mutexes}
  \begin{alertblock}{Exercise Time}
    \begin{itemize}
    \item Go to code/race
    \item Look at the code and try it\\
      See that it has a race condition
    \item Use a mutex to fix the issue
    \item See the difference in execution time
    \end{itemize}
  \end{alertblock}
\end{frame}

\begin{frame}[fragile]
  \frametitleii{Dead lock}
  \begin{exampleblock}{Scenario}
    \begin{itemize}
    \item 2 mutexes, 2 threads
    \item locking order different in the 2 threads
    \end{itemize}
  \end{exampleblock}
  \pause
  \begin{block}{Sequence diagram}
    \begin{tikzpicture}
      \begin{umlseqdiag}
        \umlobject[x=0]{Thread 1}
        \umlobject[x=2.5, fill=blue!20]{Mutex A}
        \umlobject[x=5, fill=blue!20]{Mutex B}
        \umlobject[x=7.5]{Thread 2}
        \begin{umlcall}[op=lock]{Thread 1}{Mutex A}
        \end{umlcall}
        \begin{umlcall}[op=lock, dt=6]{Thread 2}{Mutex B}
        \end{umlcall}
        \begin{umlcall}[op=lock (block), dt=6]{Thread 1}{Mutex B}
        \end{umlcall}
        \begin{umlcall}[op=lock (block), dt=12]{Thread 2}{Mutex A}
        \end{umlcall}
      \end{umlseqdiag}
      \draw[-triangle 60](9,0) -- (9,-4) node[right, pos=0.5]{time};
    \end{tikzpicture}
  \end{block}
\end{frame}

\begin{frame}[fragile]
  \frametitleii{How to avoid dead locks}
  \begin{block}{Possible solutions}
    \begin{itemize}
    \item Never take several locks
      \begin{itemize}
      \item Or add master lock protecting the locking phase
      \end{itemize}
    \item Respect a strict order in the locking across all threads
    \item Do not use locks
      \begin{itemize}
      \item Use other techniques, e.g. queues
      \end{itemize}
    \end{itemize}
  \end{block}
\end{frame}

\begin{frame}[fragile]
  \frametitleii{Condition variables}
  \begin{block}{How to express thread dependencies}
    \begin{itemize}
    \item Allows a thread to sleep until a given condition is satisfied
    \item std::condition\_variable object from condition\_variable header
    \end{itemize}
  \end{block}
  \pause
  \begin{block}{Usage}
    \begin{itemize}
    \item wraps an RAII lock around a mutex
    \item wait() will hang until the condition is met
      \begin{itemize}
      \item you can have several waiters sharing the same mutex
      \end{itemize}
    \item notify\_one() will wake up one waiter
    \item notify\_all() will wake up all waiters
    \end{itemize}
  \end{block}
\end{frame}

\begin{frame}[fragile]
  \frametitleii{Condition variable usage}
  \begin{exampleblock}{Example code}
    \begin{cppcode*}{}
      int value = -1;
      std::mutex mutex;
      std::condition_variable cond;
      auto t1 = std::thread([&cond] () {
        value = ... long process ...;
        cond.notify_all();
      });
      auto t2 = std::thread([&cond,&mutex,&value] () {
        std::unique_lock<std::mutex> lock{mutex};
        cond.wait(lock, [&value] { return value > 0; });
        ... use value ...
      });
      { std::unique_lock<std::mutex> lock{mutex};
        cond.wait(lock, [&value] { return value > 0; });
        std::cout << value << std::endl; }
    \end{cppcode*}
  \end{exampleblock}
\end{frame}
