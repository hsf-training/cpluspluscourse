\subsection[doxygen]{Doxygen}

\begin{frame}[fragile]
  \frametitle{Doxygen}
  \begin{block}{Doxygen}
    \begin{itemize}
      \item Generates documentation from source code comments
      \item Output formats: HTML, LaTeX, XML, ...
      \begin{itemize}
        \item May be input for further generators, e.g.\ Sphinx
      \end{itemize}
      \item Doxygen uses a config file, usually called Doxyfile
      \item Run \mintinline{bash}{doxygen -g} to generate an initial Doxyfile
      \item Edit Doxyfile (e.g.\ output format, source code location, etc.)
      \item Run \mintinline{bash}{doxygen} to (re-)generate documentation
      \item View e.g.\ HTML documentation using a standard web browser
      \item More on the \href{https://doxygen.nl/manual/starting.html}{doxygen website}
    \end{itemize}
  \end{block}
\end{frame}

\begin{frame}[fragile]
  \frametitle{Doxygen}
  \begin{block}{Comment blocks}
    \begin{columns}
      \begin{column}{.35\textwidth}
        \begin{cppcode*}{}
          /**
           * text/directives ...
           */
          some_cpp_entity

          /*!
           * text/directives ...
           */
          some_cpp_entity
        \end{cppcode*}
      \end{column}
      \begin{column}{.35\textwidth}
        \begin{cppcode*}{firstnumber=10}
          ///
          /// text/directives ...
          ///
          some_cpp_entity

          //!
          //! text/directives ...
          //!
          some_cpp_entity
        \end{cppcode*}
      \end{column}
    \end{columns}
    \begin{itemize}
      \item More details available \href{https://www.doxygen.nl/manual/docblocks.html}{here}
    \end{itemize}
  \end{block}
  \begin{block}{Comment blocks for members}
    \begin{cppcode*}{linenos=false}
      int a_class_member; ///< trailing documentation
    \end{cppcode*}
  \end{block}

\end{frame}

\begin{frame}[fragile]
  \frametitle{Doxygen}
  \begin{block}{Comment directives}
    \begin{cppcode*}{}
      /**
       * Long description here. May include HTML etc.
       *
       * \brief Checks whether i is odd
       * \tparam Integral The integral type of the input
       * \param i Input value
       * \return True if i is odd, otherwise false
       * \throw std::out_of_range if i is larger than 100
       * \see isEven
       */
      template <typename Integral>
      bool isOdd(Integral i);
    \end{cppcode*}
    \begin{itemize}
      \item All directives can also start with @ instead of \textbackslash
      \item A list of all commands can be found \href{https://doxygen.nl/manual/commands.html}{here}
    \end{itemize}
  \end{block}
\end{frame}
