\subsection[advOO]{Advanced OO}

\begin{frame}[fragile]
  \frametitlecpp[98]{Polymorphism}
  \begin{block}{the concept}
    \begin{itemize}
    \item objects actually have multiple types simultaneously
    \item and can be used as any of them
    \end{itemize}
  \end{block}
  \begin{multicols}{2}
    \begin{cppcode*}{gobble=2}
      Polygon p;

      int f(Drawable & d) {...}
      f(p);  //ok

      try {
        throw p;
      } catch (Shape & e) {
        // will be caught
      }
    \end{cppcode*}
    \columnbreak
    \center
    \begin{tikzpicture}[node distance=1.5cm]
      \classbox{Drawable}{}
      \classbox[below of=Drawable]{Shape}{}
      \classbox[below of=Shape]{Polygon}{}
      \draw[very thick,-Triangle] (Polygon) -- (Shape);
      \draw[very thick,-Triangle] (Shape) -- (Drawable);
    \end{tikzpicture}
  \end{multicols}
\end{frame}


\begin{frame}[fragile]
  \frametitlecpp[98]{Inheritance privacy and polymorphism}
  \begin{block}{Only public inheritance is visible to code outside the class}
    \begin{itemize}
    \item private and protected are not
    \item this may restrict usage of polymorphism
    \end{itemize}
  \end{block}
  \begin{multicols}{2}
    \begin{cppcode*}{gobble=2}
      Polygon p;

      int f(Drawable & d) {...}
      f(p);  // Not ok anymore

      try {
        throw p;
      } catch (Shape & e) {
        // ok, will be caught
      }
    \end{cppcode*}
    \columnbreak
    \center
    \begin{tikzpicture}[node distance=1.5cm]
      \classbox{Drawable}{}
      \classbox[below of=Drawable]{Shape}{}
      \classbox[below of=Shape]{Polygon}{}
      \draw[very thick,-Triangle] (Polygon) -- (Shape) node[midway,right] {public};
      \draw[very thick,-Triangle] (Shape) -- (Drawable) node[midway,right] {private};
    \end{tikzpicture}
  \end{multicols}
\end{frame}

\begin{frame}[fragile]
  \frametitlecpp[98]{Method overriding}
  \begin{block}{the problem}
    \begin{itemize}
    \item a given method of the parent can be overridden in a child
    \item but which one is called?
    \end{itemize}
  \end{block}
  \begin{multicols}{2}
    \begin{cppcode*}{gobble=2}
      Polygon p;
      p.draw(); // ?

      Shape & s = p;
      s.draw(); // ?
    \end{cppcode*}
    \columnbreak
    \center
    \begin{tikzpicture}[node distance=1.5cm]
      \classbox{Drawable}{
        void draw();
      }
      \classbox[below of=Drawable]{Shape}{}
      \classbox[below of=Shape]{Polygon}{
        void draw();
      }
      \draw[very thick,-Triangle] (Polygon) -- (Shape);
      \draw[very thick,-Triangle] (Shape) -- (Drawable);
    \end{tikzpicture}
  \end{multicols}
\end{frame}

\begin{frame}[fragile]
  \frametitlecpp[98]{Virtual methods}
  \begin{block}{the concept}
    \begin{itemize}
    \item methods can be declared \mintinline{cpp}{virtual}
    \item for these, the most derived object is always considered
    \item for others, the type of the variable decides
    \end{itemize}
  \end{block}
  \begin{overprint}
  \onslide<2>
  \begin{multicols}{2}
    \begin{cppcode*}{gobble=2}
      Polygon p;
      p.draw(); // Polygon.draw

      Shape & s = p;
      s.draw(); // Drawable.draw
    \end{cppcode*}
    \columnbreak
    \center
    \begin{tikzpicture}[node distance=1.5cm]
      \classbox{Drawable}{
        \mintinline{cpp}{void draw();}
      }
      \classbox[below of=Drawable]{Shape}{}
      \classbox[below of=Shape]{Polygon}{
        \mintinline{cpp}{void draw();}
      }
      \draw[very thick,-Triangle] (Polygon) -- (Shape);
      \draw[very thick,-Triangle] (Shape) -- (Drawable);
    \end{tikzpicture}
  \end{multicols}

  \onslide<3>
    \begin{multicols}{2}
    \begin{cppcode*}{gobble=2,highlightlines=5}
      Polygon p;
      p.draw(); // Polygon.draw

      Shape & s = p;
      s.draw(); // Polygon.draw
    \end{cppcode*}
    \columnbreak
    \center
    \begin{tikzpicture}[node distance=1.5cm]
      \classbox{Drawable}{
        \mintinline{cpp}{virtual void draw();}
      }
      \classbox[below of=Drawable]{Shape}{}
      \classbox[below of=Shape]{Polygon}{
        \mintinline{cpp}{void draw();}
      }
      \draw[very thick,-Triangle] (Polygon) -- (Shape);
      \draw[very thick,-Triangle] (Shape) -- (Drawable);
    \end{tikzpicture}
  \end{multicols}
  \end{overprint}
\end{frame}

\begin{frame}[fragile]
  \frametitlecpp[11]{Virtual methods - implications}
  \begin{block}{Mechanics}
    \begin{itemize}
    \item virtual methods are dispatched at run time
      \begin{itemize}
      \item while non-virtual methods are bound at compile time
      \end{itemize}
    \item they also imply extra storage and an extra indirection
      \begin{itemize}
      \item practically the object stores a pointer to the correct method
      \item in a so-called ``virtual table'' (``vtable'')
      \end{itemize}
    \end{itemize}
  \end{block}
  \begin{alertblock}{Consequences}
    \begin{itemize}
    \item virtual methods are ``slower'' than standard ones
    \item and they can rarely be inlined
    \item templates are an alternative for performance-critical cases
    \end{itemize}
  \end{alertblock}
\end{frame}

\begin{frame}[fragile]
  \frametitlecpp[11]{{\texttt override} keyword}
  \begin{block}{Principle}
    \begin{itemize}
    \item when overriding a virtual method
    \item the \mintinline{cpp}|override| keyword should be used
    \item the \mintinline{cpp}|virtual| keyword is then optional
    \end{itemize}
  \end{block}
  \begin{exampleblock}{Practically}
    \begin{cppcode}
      struct Base {
        virtual void some_func(float);
      };
      struct Derived : Base {
        void some_func(float) override;
      };
    \end{cppcode}
  \end{exampleblock}
\end{frame}

\begin{frame}[fragile]
  \frametitlecpp[11]{Why was {\texttt override} keyword introduced?}
  To detect the mistake in the following code :
  \begin{block}{Without {\texttt override} (\cpp98)}
    \begin{cppcode}
      struct Base {
        virtual void some_func(float);
      };
      struct Derived : Base {
        void some_func(double); // oops !
      };
    \end{cppcode}
  \end{block}
  \begin{itemize}
  \item with \mintinline{cpp}|override|, you would get a compiler error
  \item if you forget \mintinline{cpp}|override| when you should have it, you get a compiler warning
  \end{itemize}
\end{frame}

\begin{frame}[fragile]
  \frametitlecpp[11]{{\texttt final} keyword}
  \begin{block}{Idea}
    \begin{itemize}
    \item make sure you cannot further override a given virtual method
    \item by declaring it final
    \end{itemize}
  \end{block}
  \begin{exampleblock}{Practically}
    \begin{cppcode}
      struct Base {
        virtual void some_func(float);
      };
      struct Intermediate : Base {
        void some_func(float) final;
      };
      struct Derived : Intermediate {
        void some_func(float) override; // error
      };
    \end{cppcode}
  \end{exampleblock}
\end{frame}

\begin{frame}[fragile]
  \frametitlecpp[98]{Pure Virtual methods}
  \begin{block}{Concept}
    \begin{itemize}
    \item unimplemented methods that must be overridden
    \item marked by \mintinline{cpp}{= 0} in the declaration
    \item makes their class abstract
    \item only non-abstract classes can be instantiated
    \end{itemize}
  \end{block}
  \pause
  \begin{multicols}{2}
    \begin{cppcode*}{gobble=2}
      // Error : abstract class
      Shape s;

      // ok, draw has been implemented
      Polygon p;

      // Shape type still usable
      Shape & s = p;
      s.draw();
    \end{cppcode*}
    \columnbreak
    \begin{tikzpicture}[node distance=1.5cm]
      \classbox{Drawable}{
        \mintinline{cpp}{virtual void draw() = 0;}
      }
      \classbox[below of=Drawable]{Shape}{}
      \classbox[below of=Shape]{Polygon}{
        \mintinline{cpp}{void draw() override;}
      }
      \draw[very thick,-Triangle] (Polygon) -- (Shape);
      \draw[very thick,-Triangle] (Shape) -- (Drawable);
    \end{tikzpicture}
  \end{multicols}
\end{frame}

\begin{frame}[fragile]
  \frametitlecpp[98]{Pure Abstract Class aka Interface}
  \begin{block}{Definition of pure abstract class}
    \begin{itemize}
    \item a class that has
      \begin{itemize}
        \item no data members
        \item all its methods pure virtual
      \end{itemize}
    \item the equivalent of an Interface in Java
    \end{itemize}
  \end{block}
  \begin{multicols}{2}
    \begin{cppcode*}{gobble=2}
      struct Drawable {
        virtual void draw() = 0;
      }
    \end{cppcode*}
    \columnbreak
    \center
    \begin{tikzpicture}[node distance=1.5cm]
      \classbox{Drawable}{
        virtual void draw() = 0;
      }
    \end{tikzpicture}
  \end{multicols}
\end{frame}

\begin{frame}[fragile]
  \frametitlecpp[98]{Overriding overloaded methods}
  \begin{block}{Concept}
    \begin{itemize}
    \item overriding an overloaded method will hide the others
    \item unless you inherit them using \mintinline{cpp}{using}
    \end{itemize}
  \end{block}
  \begin{cppcode*}{gobble=0}
    struct BaseClass {
      int foo(std::string);
      int foo(int);
    };
    struct DerivedClass : BaseClass {
      using BaseClass::foo;
      int foo(std::string);
    };
    DerivedClass dc;
    dc.foo(4);      // error if no using
    \end{cppcode*}
\end{frame}

\begin{frame}[fragile]
  \frametitlecpp[98]{Polymorphism}
  \begin{alertblock}{Exercise Time}
    \begin{itemize}
    \item go to code/polymorphism
    \item look at the code
    \item open trypoly.cpp
    \item create a Pentagon, call its perimeter method
    \item create a Hexagon, call its perimeter method
    \item create a Hexagon, call its parent's perimeter method
    \item retry with virtual methods
    \end{itemize}
  \end{alertblock}
\end{frame}

\begin{frame}[fragile]
  \frametitlecpp[98]{Multiple Inheritance}
  \begin{block}{Concept}
    \begin{itemize}
    \item one class can inherit from multiple parents
    \end{itemize}
  \end{block}
  \begin{multicols}{2}
    \begin{tikzpicture}[]
      \classbox[]{Polygon}{}
      \classbox[below of=Polygon,node distance=1.5cm]{Rectangle}{}
      \classbox[right of=Rectangle,node distance=3cm]{Text}{}
      \classbox[below right of=Rectangle,node distance=2cm]{TextBox}{}
      \draw[very thick,Triangle-] (Polygon) -- (Rectangle);
      \draw[very thick,Triangle-] (Rectangle) -- (TextBox);
      \draw[very thick,Triangle-] (Text) -- (TextBox);
    \end{tikzpicture}
    \columnbreak
    \vspace{2cm}
    \begin{cppcode*}{gobble=2}
      class TextBox :
        public Rectangle, Text {
        // inherits from both
        // publicly from Rectangle
        // privately from Text
      }
    \end{cppcode*}
  \end{multicols}
\end{frame}

\begin{frame}[fragile]
  \frametitlecpp[98]{The diamond shape}
  \begin{block}{Definition}
    \begin{itemize}
    \item situation when one class inherits several times from a given grand parent
    \end{itemize}
  \end{block}
  \begin{alertblock}{Problem}
    \begin{itemize}
    \item are the members of the grand parent replicated?
    \end{itemize}
  \end{alertblock}
  \vfill
  \hspace{2.5cm}
  \begin{tikzpicture}[]
    \classbox[]{Drawable}{}
    \classbox[below left of=Drawable,node distance=2cm]{Rectangle}{}
    \classbox[right of=Rectangle,node distance=3cm]{Text}{}
    \classbox[below right of=Rectangle,node distance=2cm]{TextBox}{}
    \draw[very thick,Triangle-] (Drawable) -- (Rectangle);
    \draw[very thick,Triangle-] (Drawable) -- (Text);
    \draw[very thick,Triangle-] (Rectangle) -- (TextBox);
    \draw[very thick,Triangle-] (Text) -- (TextBox);
  \end{tikzpicture}
\end{frame}

\begin{frame}[fragile]
  \frametitlecpp[98]{Virtual inheritance}
  \begin{block}{Solution}
    \begin{itemize}
    \item inheritance can be {\it virtual} or not
    \item {\it virtual} inheritance will ``share'' parents
    \item standard inheritance will replicate them
    \end{itemize}
    \begin{cppcode}
      class Text : public virtual Drawable {...};
      class Rectangle : public virtual Drawable {...};
    \end{cppcode}
  \end{block}
  \begin{multicols}{2}
    \begin{tikzpicture}[]
      \classbox[below of=title]{Drawable}{}
      \classbox[below left of=Drawable,node distance=2cm]{Rectangle}{}
      \classbox[right of=Rectangle,node distance=3cm]{Text}{}
      \classbox[below right of=Rectangle,node distance=2cm]{TextBox}{}
      \draw[very thick,Triangle-] (Drawable) -- node[below,pos=0.35,sloped] {\scriptsize virtual} (Rectangle);
      \draw[very thick,Triangle-] (Drawable) -- node[below,pos=0.45,sloped] {\scriptsize virtual} (Text);
      \draw[very thick,Triangle-] (Rectangle) -- (TextBox);
      \draw[very thick,Triangle-] (Text) -- (TextBox);
    \end{tikzpicture}
    \columnbreak
    \begin{tikzpicture}[]
      \classbox[]{Drawable1}{}
      \classbox[below of=Drawable1,node distance=1.5cm]{Rectangle}{}
      \draw[very thick,Triangle-] (Drawable1) -- (Rectangle);
      \classbox[right of=Drawable1,node distance=3cm]{Drawable2}{}
      \classbox[below of=Drawable2,node distance=1.5cm]{Text}{}
      \draw[very thick,Triangle-] (Drawable2) -- (Text);
      \classbox[below right of=Rectangle,node distance=2cm]{TextBox}{}
      \draw[very thick,Triangle-] (Rectangle) -- (TextBox);
      \draw[very thick,Triangle-] (Text) -- (TextBox);
    \end{tikzpicture}
  \end{multicols}
\end{frame}

\begin{frame}[fragile]
  \frametitlecpp[98]{Multiple inheritance advice}
  \begin{block}{Do not use multiple inheritance}
    \begin{itemize}
    \item Except for inheriting from interfaces
    \item and for rare special cases
    \end{itemize}
  \end{block}
  \pause
  \begin{alertblock}{Do not use diamond shapes}
    \begin{itemize}
    \item This is a sign that your architecture is not correct
    \item In case you are tempted, think twice and change your mind
    \end{itemize}
  \end{alertblock}
\end{frame}

\begin{frame}[fragile]
  \frametitlecpp[98]{Virtual inheritance}
  \begin{alertblock}{Exercise Time}
    \begin{itemize}
    \item go to code/virtual\_inheritance
    \item look at the code
    \item open trymultiherit.cpp
    \item create a TextBox and call draw
    \item Fix the code to call both draws by using types
    \item retry with virtual inheritance
    \end{itemize}
  \end{alertblock}
\end{frame}

\begin{frame}[fragile]
  \frametitlecpp[98]{Virtual inheritance}
  \begin{exampleblock}{Good practice}
    if you write a class and expect users to inherit from it, declare its destructor \mintinline{cpp}{virtual}
  \end{exampleblock}
  \begin{alertblock}{Warning}
    in case of virtual inheritance it is the most derived class that calls the virtual base class's constructor
  \end{alertblock}
\end{frame}
