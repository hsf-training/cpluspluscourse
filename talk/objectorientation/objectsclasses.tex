\subsection[OO]{Objects and Classes}

\begin{frame}[fragile]
  \frametitlecpp[98]{What are classes and objects}
  \begin{block}{Classes (or ``user-defined types'')}
    C structs on steroids
    \begin{itemize}
    \item with inheritance
    \item with access control
    \item including methods (aka.\ member functions)
    \end{itemize}
  \end{block}
  \begin{block}{Objects}
    \begin{itemize}
    \item instances of classes
    \end{itemize}
  \end{block}
  \begin{block}{A class encapsulates state and behavior of ``something''}
    \begin{itemize}
    \item shows an interface
    \item provides its implementation
      \begin{itemize}
      \item status, properties
      \item possible interactions
      \item construction and destruction
      \end{itemize}
    \end{itemize}
  \end{block}
\end{frame}


\begin{frame}[fragile]
  \frametitlecpp[98]{My first class}
  \begin{multicols}{2}
    \begin{cppcode*}{}
      struct MyFirstClass {
        int a;
        void squareA() {
          a *= a;
        }
        int sum(int b) {
          return a + b;
        }
      };

      MyFirstClass myObj;
      myObj.a = 2;

      // let's square a
      myObj.squareA();
    \end{cppcode*}
    \columnbreak
    \center
    \null \vfill
    \begin{tikzpicture}
      \classbox{MyFirstClass}{
        int a; \\
        void squareA(); \\
        int sum(int b);
      }
    \end{tikzpicture}
    \vfill \null
  \end{multicols}
\end{frame}

\begin{frame}[fragile]
  \frametitlecpp[98]{Separating the interface}
  \begin{columns}[t]
    \begin{column}{.45\textwidth}
    \begin{block}{Header: MyClass.hpp}
      \begin{cppcode*}{}
        #pragma once
        struct MyClass {
          int a;
          void squareA();
        };
      \end{cppcode*}
    \end{block}
    \begin{block}{Implementation: MyClass.cpp}
      \begin{cppcode*}{}
        #include "MyClass.hpp"
        void MyClass::squareA() {
          a *= a;
        }
      \end{cppcode*}
    \end{block}
    \end{column}
    \begin{column}{.45\textwidth}
    \begin{block}{User 1: main.cpp}
      \begin{cppcode*}{}
        #include "MyClass.hpp"
        int main() {
          MyClass mc;
          ...
        }
      \end{cppcode*}
    \end{block}
    \begin{block}{User 2: fun.cpp}
      \begin{cppcode*}{}
        #include "MyClass.hpp"
        void f(MyClass& mc) {
          mc.squareA();
        }
      \end{cppcode*}
    \end{block}
    \end{column}
  \end{columns}
\end{frame}

\begin{frame}[fragile]
  \frametitlecpp[98]{Implementing methods}
  \begin{goodpractice}{Implementing methods}
    \begin{itemize}
    \item usually in .cpp, outside of class declaration
    \item using the class name as ``namespace''
    \item short member functions can be in the header
    \item some functions (templates, \cppinline{constexpr}) must be in the header
    \end{itemize}
  \end{goodpractice}
  \begin{block}{}
    \begin{cppcode}
    #include "MyFirstClass.hpp"

    void MyFirstClass::squareA() {
      a *= a;
    }
    int MyFirstClass::sum(int b) {
      return a + b;
    }
    \end{cppcode}
  \end{block}
\end{frame}

\begin{frame}[fragile]
  \frametitlecpp[98]{Method overloading}
  \begin{block}{The rules in \cpp}
    \begin{itemize}
    \item overloading is authorized and welcome
    \item signature is part of the method identity
    \item but not the return type
    \end{itemize}
  \end{block}
  \begin{cppcode}
    struct MyFirstClass {
      int a;
      int sum(int b);
      int sum(int b, int c);
    };

    int MyFirstClass::sum(int b) { return a + b; }

    int MyFirstClass::sum(int b, int c) {
      return a + b + c;
    }
  \end{cppcode}
\end{frame}
