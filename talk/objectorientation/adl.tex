\subsection[ADL]{Name Lookups}

\begin{frame}[fragile]
  \frametitlecpp[98]{Basics of name lookup}
  \begin{exampleblock}{Example code}
    \begin{cppcode}
      std::cout << std::endl;
    \end{cppcode}
  \end{exampleblock}
  \begin{block}{How to find the declaration of a name ?}
    Mainly 2 cases :
    \begin{itemize}
    \item qualified lookup, for names preceded by `::'
      \begin{itemize}
      \item here \mintinline{cpp}{cout} and \mintinline{cpp}{endl}
      \item name is only looked for in given class/namespace/enum class
      \end{itemize}
    \item unqualified lookup
      \begin{itemize}
      \item here for \mintinline{cpp}{std} and \mintinline{cpp}{operator<<}
      \item name is looked for in a sequence of scopes until found
      \item remaining scopes are not examined
      \end{itemize}
    \end{itemize}
  \end{block}
\end{frame}

\begin{frame}
  \frametitlecpp[98]{Unqualified name lookup and ADL}
  \begin{block}{Ordered list of scopes (simplified)}
    \begin{itemize}
    \item file (only for global level usage)
    \item current namespace/block, enclosing namespaces/blocks, etc...
    \item current class if any, base classes if any, etc...
    \item for a call expression (e.g.\ \mintinline{cpp}{f(a, b)} or \mintinline{cpp}{a + b}), Argument Dependent Lookup (ADL)
    \end{itemize}
  \end{block}
  \begin{exampleblock}{Argument Dependent Lookup (simplified)}
    To find a function name (including operators), the compiler also examines the arguments. For each argument, it searches:
    \begin{itemize}
    \item class, if any
    \item direct and indirect base classes, if any
    \item enclosing namespace
    \end{itemize}
  \end{exampleblock}
\end{frame}

\begin{frame}[fragile]
  \frametitlecpp[98]{ADL consequences (1)}
  \begin{block}{Use standalone/non-member functions}
    When a method is not accessing the private part of a class, make it a function in the same namespace
    \vspace{-1mm}
    \begin{columns}[T]
      \begin{column}{.35\textwidth}
        Don't write :
        \vspace{-1mm}
        \begin{cppcode*}{gobble=6}
          namespace MyNS {
            struct A {
              T func(...);
            };
          }
        \end{cppcode*}
      \end{column}
      \begin{column}{.35\textwidth}
        Prefer :
        \vspace{-1mm}
        \begin{cppcode*}{gobble=6,firstnumber=6}
          namespace MyNS {
            struct A { ... };
            T func(const A&, ..);
          }
        \end{cppcode*}
      \end{column}
    \end{columns}
    \vspace{.2cm}
    Advantages :
    \begin{itemize}
    \item minimal change in user code, func still feels part of A class
    \item makes sure func does not touch internal state of A
    \end{itemize}
    Notes :
    \begin{itemize}
    \item non-member func has to be in same namespace as A
    \item please avoid global namespace
    \end{itemize}
  \end{block}
\end{frame}

\begin{frame}[fragile]
  \frametitlecpp[17]{ADL consequences (2)}
  \begin{block}{Prefer nested namespaces to \texttt{using}}
    Don't write :
    \begin{cppcode}
      using namespace MyProject;
      funcFromMyProject(...);
    \end{cppcode}
    Prefer :
    \begin{cppcode*}{firstnumber=3}
      namespace MyProject::MySubpart {
        funcFromMyProject(...);
      }
    \end{cppcode*}
    And let the compiler lookup (more easily) the function name
  \end{block}
\end{frame}

\begin{frame}[fragile]
  \frametitlecpp[98]{ADL consequences (3)}
  \begin{block}{Customization points and \texttt{using}}
    \begin{columns}[t]
      \begin{column}{.35\textwidth}
        Don't write :
        \begin{cppcode*}{gobble=6}
          N::A a,b;
          std::swap(a, b);
        \end{cppcode*}
      \end{column}
      \begin{column}{.35\textwidth}
        Prefer :
        \begin{cppcode*}{gobble=6,firstnumber=3}
          N::A a,b;
          using std::swap;
          swap(a, b);
        \end{cppcode*}
      \end{column}
    \end{columns}
    \vspace{.2cm}
    Advantages :
    \begin{itemize}
    \item allows to use \mintinline{cpp}{std::swap} by default
    \item but also to benefit from any dedicated specialization
    \end{itemize}
    \begin{columns}
      \begin{column}{.7\textwidth}
        \begin{cppcode*}{gobble = 6,firstnumber=6}
          namespace N {
            class A { ... };
            // optimized swap for A
            void swap(A&, A&);
          }
        \end{cppcode*}
      \end{column}
    \end{columns}
  \end{block}
\end{frame}

\begin{frame}[fragile]
  \frametitlecpp[98]{ADL and Hidden Friends}
  \begin{block}{Hidden friend declarations participate in ADL}
    \begin{itemize}
      \item Friend functions defined in class scope are ``hidden friends''
      \item They are \emph{not} member functions (no access to \mintinline{cpp}{this})
      \item They participate in ADL, e.g.\ \mintinline{cpp}{std::cout << complex}
    \end{itemize}
  \end{block}
  \begin{exampleblock}{\texttt{operator<<} as hidden friend}
    \begin{cppcode*}{gobble=2}
      class Complex {
        float m_real, m_imag; public: Complex(float, float);
        friend
        std::ostream & operator<<(std::ostream & os,
                                  Complex const & c) {
          return os << c.m_real << " + " << c.m_imag << "i";
        }
      };
      std::cout << Complex{2.f, 3.f}; // Prints 2 + 3i
    \end{cppcode*}
  \end{exampleblock}
\end{frame}
